%% Copernicus Publications Manuscript Preparation Template for LaTeX Submissions
%% ---------------------------------
%% This template should be used for copernicus.cls
%% The class file and some style files are bundled in the Copernicus Latex Package, which can be downloaded from the different journal webpages.
%% For further assistance please contact Copernicus Publications at: production@copernicus.org
%% https://publications.copernicus.org/for_authors/manuscript_preparation.html

%% copernicus_rticles_template (flag for rticles template detection - do not remove!)

%% Please use the following documentclass and journal abbreviations for discussion papers and final revised papers.

%% 2-column papers and discussion papers
\documentclass[, manuscript]{copernicus}



%% Journal abbreviations (please use the same for preprints and final revised papers)

% Advances in Geosciences (adgeo)
% Advances in Radio Science (ars)
% Advances in Science and Research (asr)
% Advances in Statistical Climatology, Meteorology and Oceanography (ascmo)
% Aerosol Research (ar)
% Annales Geophysicae (angeo)
% Archives Animal Breeding (aab)
% Atmospheric Chemistry and Physics (acp)
% Atmospheric Measurement Techniques (amt)
% Biogeosciences (bg)
% Climate of the Past (cp)
% DEUQUA Special Publications (deuquasp)
% Earth Surface Dynamics (esurf)
% Earth System Dynamics (esd)
% Earth System Science Data (essd)
% E&G Quaternary Science Journal (egqsj)
% EGUsphere (egusphere) | This is only for EGUsphere preprints submitted without relation to an EGU journal.
% European Journal of Mineralogy (ejm)
% Fossil Record (fr)
% Geochronology (gchron)
% Geographica Helvetica (gh)
% Geoscience Communication (gc)
% Geoscientific Instrumentation, Methods and Data Systems (gi)
% Geoscientific Model Development (gmd)
% History of Geo- and Space Sciences (hgss)
% Hydrology and Earth System Sciences (hess)
% Journal of Bone and Joint Infection (jbji)
% Journal of Micropalaeontology (jm)
% Journal of Sensors and Sensor Systems (jsss)
% Magnetic Resonance (mr)
% Mechanical Sciences (ms)
% Natural Hazards and Earth System Sciences (nhess)
% Nonlinear Processes in Geophysics (npg)
% Ocean Science (os)
% Polarforschung - Journal of the German Society for Polar Research (polf)
% Primate Biology (pb)
% Proceedings of the International Association of Hydrological Sciences (piahs)
% Safety of Nuclear Waste Disposal (sand)
% Scientific Drilling (sd)
% SOIL (soil)
% Solid Earth (se)
% State of the Planet (sp)
% The Cryosphere (tc)
% Weather and Climate Dynamics (wcd)
% Web Ecology (we)
% Wind Energy Science (wes)

% Pandoc citation processing

% The "Technical instructions for LaTex" by Copernicus require _not_ to insert any additional packages.
% 
% tightlist command for lists without linebreak
\providecommand{\tightlist}{%
  \setlength{\itemsep}{0pt}\setlength{\parskip}{0pt}}


%%\usepackage{booktabs}
\usepackage{longtable}
\usepackage{array}
\usepackage{multirow}
\usepackage{wrapfig}
\usepackage{float}
\usepackage{colortbl}
\usepackage{pdflscape}
\usepackage{tabu}
\usepackage{threeparttable}
\usepackage{threeparttablex}
\usepackage[normalem]{ulem}
\usepackage{makecell}
\usepackage{xcolor}
%
%% \usepackage commands included in the copernicus.cls:
%\usepackage[german, english]{babel}
%\usepackage{tabularx}
%\usepackage{cancel}
%\usepackage{multirow}
%\usepackage{supertabular}
%\usepackage{algorithmic}
%\usepackage{algorithm}
%\usepackage{amsthm}
%\usepackage{float}
%\usepackage{subfig}
%\usepackage{rotating}

\begin{document}


\title{Incorporating observed fire severity in refined emissions
estimates for boreal and temperate forest fires in the carbon budget
model CBM-CFS3}


\Author[1]{Dan K.}{Thompson}
\Author[2]{Ellen}{Whitman}
\Author[3]{Hafer}{Mark}
\Author[2]{Oleksandra}{Hararuk}
\Author[1]{Chelene}{Hanes}
\Author[3]{Vinicius}{Manvailer Goncalves}
\Author[3]{Ben}{Hudson}


\affil[1]{Natural Resources Canada, Canadian Forest Service, Great Lakes
Forestry Centre, Sault Ste. Marie, Canada}
\affil[2]{Natural Resources Canada, Canadian Forest Service, Northern
Forestry Centre, Edmonton, Canada}
\affil[3]{Natural Resources Canada, Canadian Forest Service, Pacific
Forestry Centre, Victoria, Canada}

\runningtitle{Fire Severity and Canadian Forest Carbon Budget Models}

\runningauthor{Thompson et al.}





\received{}
\pubdiscuss{} %% only important for two-stage journals
\revised{}
\accepted{}
\published{}

%% These dates will be inserted by Copernicus Publications during the typesetting process.


\firstpage{1}

\maketitle


\begin{abstract}
Among the many natural disturbances that affect Canada's boreal and
temperate forest biomes, wildfire has the greatest impact on forest
productivity, landscape structure, timber supply and greenhouse gas
emissions. Current representations of fire impact on forest carbon
stocks is limited to a single parameterization of fire severity
(i.e.~the fate of biomass including consumption) that assumes only high
severity fires, despite a large and increasing evidence base of
widespread mixed-severity wildfire. In this sub-model of the National
Forest Carbon Monitoring Accounting and Reporting System for Canada
(NFC-MARS), field measurements of biomass consumption, as related to
satellite-derived burn severity maps, are interpreted from a fire
physics and ecology perspective to derive algorithms to describe forest
carbon fluxes in the immediate aftermath of fires. Model outputs
indicate total carbon emissions range from a 11 t C/ha in Boreal Shield
West forests of Saskatchewan following low severity fire to over 60 t
C/ha in Pacific Maritime forests of British Columbia under high severity
fire. Shifting from a low drought condition, low-severity burn to a high
drought condition, high severity burn typically doubles total carbon
emissions. The existing approach to emissions in NFC-MARS yields CO2e
emissions that are 10-20\% percent higher than this new method, owing to
lower overall canopy consumption with mixed fires, which is only
partially offset by increased estimates forest floor consumption in this
new approach. Comparisons against directly observed fire plume emissions
ratios as well as against annualized carbon emissions for Canada's 2023
fire season show good model agreement with observations.
\end{abstract}


\copyrightstatement{His Majesty the King in Right of Canada as
represented by the Minister of Natural Resources Canada. This work is
distributed under the Creative Commons Attribution 4.0 License.}


\section{Introduction}

Wildfire is on par with insects as the largest stand-replacing
disturbance process in Canada's forest, impacting \textasciitilde1-3 Mha
of Canada's 367 Mha forested area in a typical year \citep{hanes2019}.
In Canada's reliable 53-year burned area record, nine years have
exceeded 4 Mha of burned area (or approximately 1\% of Canada's forest
area) \citep{skakun2022}. The 2023 fire season in Canada burned a
remarkable 15 Mha owing to extreme drought, severe fire weather
conditions, and a prolonged fire season length
\citep{jainDriversImpactsRecordBreaking2024}.

Of this, publicly-owned managed forest under long or short-term tenure
accounts for 23\% of the area burned between 1972 and 2024;
publicly-owned managed forest under long-term license to private timber
companies forms 40\% of Canada's forest area \citep{stinson2019}.
Privately-owned forest constitutes only 6\% of the forest area, but only
0.5\% of area burned. The remainder of the forest area burned in Canada
being a mix of formally protected areas, remote unmanaged forest,
Indigenous reserve lands and other uses without large-scale harvesting.
Managed northern forest areas adjacent to communities have historically
shown some meaningful local fire suppression effects with a bias towards
older forest nearby boreal forest communities in Canada, though the
effect is largely limited to a 25 km radius between these widely
dispersed communities \citep{parisien2020}. More southern boreal forest
with extensive suppression activities has historically seen evidence of
a reduction in observed vs potential area burned
\citep{cummingEffectiveFireSuppression2005} though this effect has
likely been erased given the increasing frequency of extreme burning
conditions \citet{wangCriticalFireWeather2023} and corresponding record
area burned \citep{jainDriversImpactsRecordBreaking2024}.

Burned area in Canada is dominated by a relatively small number of very
large fires, with 3\% of fires constituting 97\% of the burned area
\citep{stocks2002}. Lightning-caused fires account for approximately
half of all ignitions and approximately 80\% of burned area, but no
distinction is made between human and lightning ignition for forest
sector GHG reporting purposes. Annual burned area mapping for GHG
reporting in Canada is conducted using a composite of satellite and
aerial mapping at 30 metre resolution; the relatively small number of
large fires, and their slow vegetation regeneration \citep{white2017}
allows for reliable mapping using multi-spectral imagery such as Landsat
within one year of the fire \citep{whitman2018}.

Canada reports on GHG emission and removals from the forest sector in
Canada's National GHG Inventory Report (NIR). Carbon stocks and stock
changes are modelled using CBM-CFS3
\citep{kurzCBMCFS3ModelCarbondynamics2009} parameterized and driven by
spatial and non-spatial data describing forest inventory, annual yield
and disturbance \citep{stinsonInventorybasedAnalysisCanadas2011}.
Following international good practice guidance to focus on human-caused
emissions in NIRs, Canada focuses emission reporting on managed forest
lands, which make up approximately 225 ha of Canada's 367 ha total
forest area. In CBM-CFS3, forest inventory is represented by spatially
referenced stand lists representing large homogeneous stands that fall
within spatial units (e.g.~forest management areas) and broad regions
called Reconciliation Units (RU) that are the intersection of ecozones
with provincial/territorial boundaries \citep{kurz2002}. Within CBM-CFS3
precisely mapped burned areas \citep{hall2020} are summarized at the
level of spatial units and applied randomly to stands within that unit
Years with larger burned areas will improve the representative nature of
random stands being assigned fire disturbance within a year, as a wider
variety of stand types will be selected. Furthermore, large,
drought-driven fire years such as 2023 have been observed to burn at
more equal rates across the diversity of local ecosite flammability
\citep{parks2018, whitman2024} and thus provides an ideal test case for
assessing changes in emissions estimates for a spatially referenced
system. Importantly, these years with very large area burned still show
a similar natural range in variability of burn severity
\citep{whitman2018}.

CBM-CFS3 currently assesses fire impacts to carbon pools only as
representing high-severity fire, which is the most common of the widely
adopted three severity class approach for assessing of burned forest in
Canada \citep{hallRemoteSensingBurn2008} and the conterminous US
\citep{parksNewMetricQuantifying2014}. Currently, biomass consumption
estimates are based on the assumption of complete crown mortality, with
additional biomass consumption following a spatially-referenced
aggregated estimate at the RU of annualized drought conditions.
Quantification of the change in carbon stock in CBM-CFS3 is made via a
``Disturbance Matrix'' (hereafter referred to as a DM) which is simply a
matrix of the proportional mass flux of carbon between each pair of
pools in the model for a site experiencing a given disturbance. This
proportional mass flux is independent of the size of the C pool. Pools
in the DM include all the above and below-ground pools tracked in
CBM-CFS3, as well as an atmospheric sink pool. Importantly, unlike
emissions-only fire models commonly used in air quality modelling in
Canada \citep{chen2019}, DM definitions in CBM-CFS3 also track the
transfer of live biomass to dead, but still uncombusted pools, such as
the transfer from live stemwood pools (which are killed but not burned)
to standing deadwood, also known as snags.

In Canada's forests, a combination of disturbance history, soils, and
less frequently topographic variables determine leading tree species; at
local scales (1--100 ha), tree species plays a major role in determining
ecosystem susceptibility to fire \citep{bernier2016} where older,
conifer-dominated forests burn at very high rates relative to adjacent
deciduous or mixed stands. Even when deciduous and mixed forests burn at
rates more similar to older conifer stands during large drought-driven
fire years \citep{parksNewMetricQuantifying2014}, they do so at
consistently lower severity compared to all but the most xeric conifer
forests \citep{whitman2018}. Thus, important local biases in fire
activity and severity towards older and moderate to poorly-drained
forests are not resolved in the spatially-referenced CBM-CFS3 when a
uniform fire severity is applied.

To support recent advances in operational burn severity mapping for
Canada \citep{whitman2020} alongside multi-decade reliable burned area
records that provide certainty on fire start and end dates
\citep{hall2020}, this paper describes a local scale (30-m) method for
defining a per-pixel proportional carbon flux measurement via a locally
calculated fire DM. In this document, we outline the evidence-based fire
Disturbance Matrices (DM) updated and designed for a planned,
spatially-explicit update to the CBM-CFS3, anchored in a three severity
class paradigm. These fire carbon flux models are built from a blend of
aggregated field data linked to remotely sensed severity, as well as
insights from fire physics and experimental fires. Key knowledge gaps
are also highlighted, with interim solutions presented until further
quantification can be done in field studies, such as from further
wildfire observations, experimental fires, or prescribed fires.

Simplified fire DMs (i.e.~a scalar reduction on 100\% mortality
assumption) have been used in valuable scenario exercises using a
spatial explicit version of CBM-CFS3 for assessment of future fire and
harvest scenarios\citep{smyth2022}; the algorithm development shown here
provides an important data-driven and regionally-adjusted framework that
better reflects the ecological nuances of moderate and low-severity fire
across Canada's diverse ecozones.

\section{Methods}

\subsection{Carbon Modelling}

\subsection{Carbon Modelling Spatial Units}

Carbon reporting in Canada's forests is broken down by Reconciliation
Unit (RU), which is the intersection of terrestrial ecozones and
provincial/territorial boundaries, as shown in Figure \ref{fig:RU-map}.
Forest carbon stocks in Canada are influenced by species composition,
age class distribution, natural disturbance history, human-influenced
management regime and climatic variables, among other factors, which are
well-represented by the RU framework supporting Canada's NIR. Due to the
current spatial imprecision of certain key model data
(e.g.~deforestation rates), analysis finer than the RU-level tends to
misrepresent regional-variation in emission estimates.

\begin{figure}
\centering
\includegraphics{GMD-Thompson-et-al_files/figure-latex/Fig1-RU-ecozones-1.pdf}
\caption{\label{fig:RU-map}Reconciliation Units used in the CBM-CFS3
framework.}
\end{figure}

\subsection{Biomass pools of the CBM-CFS3 Model}

\begin{table}
\centering
\caption{\label{tab:CBM-pools-simple-table}Key components of the CBM-CFS3 model relevant to fire behaviour and fuel consumption.}
\centering
\begin{tabular}[t]{>{\centering\arraybackslash}p{3cm}|>{\centering\arraybackslash}p{6cm}|>{\centering\arraybackslash}p{3cm}}
\hline
Component & Fire Behaviour & CBM-CFS3 Pool\\
\hline
Stemwood & The main stem of the tree can either survive or be killed and converted to a snag, depending on severity. & Merchantable\\
\hline
Foliage & Susceptible to combustion at high severity and mortality at all severities. & Foliage\\
\hline
Branches & Smaller diameter classes susceptible to combustion at high severity and mortality at all severities. & Other\\
\hline
Coarse woody debris & Susceptible to combustion at all severities. After fire, uncombusted snags fall down into this pool. & Medium DOM\\
\hline
Forest Floor & Combustion fraction dependent on drought condition; lower susceptibility in F, H, O soil horizons (i.e. Slow pool). After fire at any severity, uncombusted foliage and dead branches transfer to these pools. Uncombusted roots can be transferred into these pools for high severity fires when the main stem experiences mortality. & Above Ground Very Fast (Litter); Above Ground Fast; Above Ground Slow DOM (F,H,O)\\
\hline
Snags & Higher combustion susceptibility at higher severity & Stem, Branch Snag\\
\hline
\end{tabular}
\end{table}

The carbon pools represented in CBM-CFS3 can be broadly categorized as
live biomass, dead organic matter (DOM) and soil carbon and are
described in detail in Kurz et al.~2009. Key components that interact
most with wildfire are summarized in Table
\ref{tab:CBM-pools-simple-table}. Though designed for forest GHG
reporting, the pool structure of CBM-CFS3 does distinguish between
softwood (conifer) and hardwood (broadleaf) that also have large
contrasts in both foliage flammability
\citep{alexanderSurfaceFireSpread2010} and ability to survive low to
moderate intensity surface fires \citep{heonResistanceBorealForest2014}.
Tree biomass pools are further divided into merchantable tree biomass
pools (foliage, stemwood, branches), with separate hardwood and softwood
pools for smaller, subcanopy trees. Representative pool values (in t
C/ha) per Reconciliation Unit are produced and updated annually based on
updated forest inventory and disturbance data. Pool size information by
RU is provided in Appendix A. Upon the detection of a forest disturbance
such as fire, pool C biomass (both dead and alive) can be transferred to
other pools including an atmosphere pool (or remaining the same pool) at
proportions prescribed by a Disturbance Matrix. Any live or dead biomass
pools combusted immediately are tallied in the direct emissions sums
reported here. Live biomass that is killed but not combusted by the fire
is tracked, but experiences delayed decomposition on the order of years
to decades, and therefore is not tallied within the direct emissions
modelling shown below.

\subsection{Axioms of forest carbon budget after fire}

To simplify the process of creating DMs as a distillation of the
complexities of fire severity and combustion patterns, the following
logical axioms are proposed and maintained throughout:

\begin{enumerate}
\def\labelenumi{\arabic{enumi}.}
\item
  Disturbance matrices are to be in terms of mortality, not survival.
  Mortality here is defined as tree death by the end of the calendar
  year of the fire's occurrence. Tree mortality in subsequent years is
  not modelled here.
\item
  Crown Fraction Burned (CFB) is a mass-based estimate of the portion of
  foliage consumed in the flaming passage of a fire, and is inclusive of
  merchantable and submerchantable trees, both broadleaf and needleleaf.
  Needles that are heat-killed but otherwise not consumed in the fire
  are not considered part of CFB, and are instead considered part of the
  foliage to litter biomass transfer.
\item
  The heat-killed but unconsumed fraction of the canopy is equal to
  (mortality - CFB).
\item
  In submerchantable trees, mortality == CFB.
\item
  In merchantable trees, CFB \textless= mortality.
\item
  Snags are inclusive of both those killed by prior fire as well as
  those killed by all other causes.
\end{enumerate}

Of these, Crown Fraction Burned (CFB) is an important concept used
primarily in fire behaviour science but not carbon accounting nor fire
ecology. CFB was introduced in the 1992 Fire Behaviour Prediction System
documentation \citep{forestrycanadafiredangerratinggroup1992}, and
provides a simple continuous 0-100 variable for only the consumption of
foliage (inclusive of both conifer and broadleaf). For our purposes, CFB
is the desirable metric as opposed to ordinal and less precise systems
like Canopy Fire Severity Index \citep{kasischke2000} that allows the
user to specify which pools of canopy biomass are consumed, but not the
precise fraction of each given pool that is consumed.

\subsection{Combustion gas emission ratios}

Certain variables, like the partitioning of
CO\textsubscript{2}:CH\textsubscript{4}:CO gas emissions, are constant
throughout ecozones, but vary by flaming vs smouldering combustion
modes. The precise emissions ratios vary slightly between models and
field studies, but for this initial algorithm assessment, we define
these emissions ratios as being identical to those used in Canada's
operational wildfire smoke forecasting system, FireWork
\citep{chen2019}. They are defined as:

\begin{table}
\centering
\caption{\label{tab:tab-globalVarsEmissions}Emissions factors for flaming and smouldering used in this model.  Values are the molar fraction of carbon in the unburned biomass and the emitted species, e.g. 87 percent of C in biomass is converted to C as CO2 in flaming combustion.}
\centering
\begin{tabular}[t]{l|r|r}
\hline
Spp & Flaming & Smouldering\\
\hline
CO2 & 0.868 & 0.703\\
\hline
CO & 0.070 & 0.161\\
\hline
PM10 & 0.022 & 0.048\\
\hline
NMOG & 0.016 & 0.035\\
\hline
PM25 & 0.019 & 0.040\\
\hline
CH4 & 0.005 & 0.013\\
\hline
BC & 0.000 & 0.000\\
\hline
\end{tabular}
\end{table}

where CO\textsubscript{2} is responsible for 86.8\% of emissions in the
flaming phase, but only 70.3\% of emissions in the smouldering phase,
with a doubling of CO emissions and tripling of CH\textsubscript{4}
emissions (Table \ref{tab:tab-globalVarsEmissions}). With a Global
Warming Potential of CO equal to 1.0 and CH\textsubscript{4} of 25, the
Global Warming Potential per unit of biomass consumption in the
smouldering phase is 1.18 times higher in global warming potential
compared to flaming, not including differential aerosol production and
injection heights, however. With flaming and smouldering each
contributing roughly equally to wildfire emissions, these distinct
flaming and smouldering emissions ratios correspond well prior emissions
factors used in CBM-CFS3. Note that as currently described, the sum of
CO\textsubscript{2}, CH\textsubscript{4}, and CO emissions from
wildfires only represent approximately 95\% of the fire carbon mass
emitted to the atmosphere, with 0.5-2.0\% of biomass emitted as
particulate matter (e.g.~PM2.5, but also PM1 and PM10 classes of
particulates at 1 and 10 um diameters, respectively), and an additional
3\% \citep{hayden2022} to as little as 1\% \citep{simon2010} composed of
non-methane organic gases that have a large range in global warming
potentials as compared to CH\textsubscript{4}.

\subsection{Ground plot and remotely sensed fire severity data}

While remotely sensed fire severity metrics are capable of determining
if a specific location falls within a low, moderate, or high severity
fire area \citep{hall2008}, remotely sensed fire severity metrics alone
do not inform estimates of fuel consumption down to specific biomass
pools. Instead, the change in biomass pools is empirically related to
per-pixel satellite spectral reflectance indices (i.e.~dNBR, RBR), using
detailed and semi-standardized methods of burn severity ground plots
with unburned controls
\citep[e.g.][]{cockeComparisonBurnSeverity2005, hall2008}. Burn severity
ground plot measurements are often aggregated into a single ``Composite
Burn Index'' with a weighting scheme originating in southwestern U.S.
forests
\citep{keyLandscapeAssessment2006, parksGivingEcologicalMeaning2019};
plot level measurements of individual forest biomass pools (e.g.~conifer
overstory mortality \% and canopy consumption \%) are recorded as part
of the CBI methodology. In this methodology we do not use the CBI
aggregated metric itself to infer biomass consumption, but rather
compilations of burn severity ground plot measurements that include all
the individual biomass pool measurements. Individual burn severity plots
were assigned a severity class based on ground observations, and an
ecozone-level median biomass pool consumption fraction was computed for
each severity class for each biomass pool (see Figure
\ref{fig:Fig-CBI-ecozones-map}). For ecozones without ground plot
measurements, an adjacent and ecologically similar ecozone with
observations was used.

\begin{figure}
\includegraphics[width=0.75\linewidth]{../CBIPlots_Canada_Ecozones} \caption{\label{fig:Fig-CBI-ecozones-map} Locations of field-based burn severity plots.}\label{fig:Fig-CBI-ecozones-map}
\end{figure}

The axioms listed above, combined with field survey data on biomass
combustion and mortality rates as stratified by severity class, forms a
improved data-driven and ecologically-informed approach to fire direct C
emissions estimations here termed FireDMs (Fire Disturbance
Matrix-severity).

\subsubsection{Overstory tree mortality and consumption}

\begin{table}
\centering
\caption{\label{tab:tab-MortByEcozone}Softwood fractional mortality by ecozone, as derived from median values from field studies}
\centering
\begin{tabular}[t]{l|r|r|r}
\hline
Ecozone & Low & Mod & High\\
\hline
BSW & 0.45 & 0.81 & 1.00\\
\hline
TP & 0.45 & 0.81 & 1.00\\
\hline
TSW & 0.10 & 0.81 & 1.00\\
\hline
BP & 0.45 & 0.81 & 1.00\\
\hline
BC & 0.24 & 0.65 & 0.98\\
\hline
BSE & 0.45 & 0.81 & 1.00\\
\hline
TSE & 0.10 & 0.81 & 1.00\\
\hline
MC & 0.28 & 0.74 & 0.98\\
\hline
HP & 0.45 & 0.81 & 1.00\\
\hline
TC & 0.24 & 0.65 & 0.98\\
\hline
PM & 0.13 & 0.38 & 0.97\\
\hline
AM & 0.28 & 0.34 & 0.95\\
\hline
MP & 0.28 & 0.34 & 0.95\\
\hline
P & 0.45 & 0.81 & 1.00\\
\hline
\end{tabular}
\end{table}

Numerous process-driven \citep{michaletz2006} or empirical
\citep{hood2017} tree mortality models are present and show significant
skill in predicting tree mortality based on fire behaviour (i.e.~flame
length, rate of spread). Since the driving data in this model is
satellite-derived fire severity over the landscape scale, fire behaviour
metrics such as flame length or scorching height of bark are not
available as a continuous mapped product. Instead, softwood and hardwood
overstory mortality is calculated per ecozone as a function of
satellite-observed fire severity using aggregated ground plot data
(Table \ref{tab:tab-MortByEcozone}).

And since large-diameter, live trees killed by fire do not experience
significant live stemwood consumption, the entirety of the live stemwood
biomass pool that is killed is transferred to the snag pool. Note that
the field data and disturbance modelling undertaken here only accounts
for tree mortality within the calendar year of the fire, and delayed
mortality of over one year has been documented in boreal low and
moderate severity fires \citep{angers2011}where less than half of total
mortality occurs after the year of the fire. Thus, the modelling here
does not account for delayed mortality that may extend upwards of 5
years after fire.

Crown Fraction Burned (CFB) speaks to the fraction of the live canopy
that is itself consumed in the flaming front. The alternate outcomes
being survival of the foliage, or the mortality of the tree without
canopy consumption, resulting in the dropping of foliage onto the forest
floor. From the axioms stated earlier, the CFB must be lower than or
equal to the mortality rate, using field studies that show any partial
crown consumption is likely sufficient to result in high rates if not
complete mortality \citep{hood2017}, which is the case in Canada's trees
with primarily thin bark. Due to the structure of CBM-CFS3, all High
Severity fires have their mortality in the merchantable and smaller
trees set to exactly 1.0, which is no more than a 5\% variance from
observed values. Ecozone-specific CFB values from field studies of burn
severity are summarized in Table \ref{tab:tab-CFBByEcozone}.

\begin{table}
\centering
\caption{\label{tab:tab-CFBByEcozone}Softwood crown fraction burned by ecozone, as derived from median values from field studies in each ecozone.}
\centering
\begin{tabular}[t]{l|r|r|r}
\hline
Ecozone & Low & Mod & High\\
\hline
BSW & 0.0 & 0.81 & 1.00\\
\hline
TP & 0.0 & 0.81 & 1.00\\
\hline
TSW & 0.1 & 0.81 & 1.00\\
\hline
BP & 0.0 & 0.81 & 1.00\\
\hline
BC & 0.0 & 0.65 & 0.98\\
\hline
BSE & 0.0 & 0.81 & 1.00\\
\hline
TSE & 0.1 & 0.81 & 1.00\\
\hline
MC & 0.0 & 0.74 & 1.00\\
\hline
HP & 0.0 & 0.81 & 1.00\\
\hline
TC & 0.0 & 0.65 & 1.00\\
\hline
PM & 0.0 & 0.38 & 0.97\\
\hline
AM & 0.0 & 0.34 & 0.95\\
\hline
MP & 0.0 & 0.34 & 0.95\\
\hline
P & 0.0 & 0.81 & 1.00\\
\hline
\end{tabular}
\end{table}

The consumption of live bark biomass is a pool in the model, and
consumption rates can be defined by severity class. At the moment,
lacking robust field data on bark biomass consumption rates across
ecozones and severity classes (which are a small portion of the overall
biomass), the bark proportional consumption rate is set to 34\% of the
overstory mortality rate, based only on a single set well-observed high
severity fires in the Taiga Plains by \citep{santín2015}.

\begin{table}
\centering
\caption{\label{tab:tab-ReSproutByEcozone}Ecozone-level average fraction of hardwood overstory species that do not suffer extensive belowground biomass mortality after fire.}
\centering
\begin{tabular}[t]{l|r}
\hline
Ecozone & Resprout Fraction\\
\hline
BSW & 0.75\\
\hline
TP & 0.75\\
\hline
TSW & 0.94\\
\hline
BP & 0.99\\
\hline
BC & 0.76\\
\hline
BSE & 0.67\\
\hline
TSE & 0.78\\
\hline
MC & 0.97\\
\hline
HP & 0.80\\
\hline
TC & 0.27\\
\hline
PM & 0.39\\
\hline
AM & 0.76\\
\hline
MP & 0.32\\
\hline
P & 0.99\\
\hline
\end{tabular}
\end{table}

A major distinction is made between softwood and hardwood trees, where
in Canada's boreal forests, a large fraction of hardwood trees (see
Appendix E) are able to resprout even when the main stem has been killed
by an intense forest fire \citep{brown1987}. Accordingly, the root
mortality rates differ greatly between softwoods and hardwoods, with
softwood root mortality equal precisely to stem mortality, while in
resprouting hardwoods, little root mortality is observed even after
intense fire \citep{pérez-izquierdo2019}. Though spatially explicit
versions of CBM-CFS3 can resolve a species list down to the pixel level,
currently an ecozone-level regional average composition of hardwood
species with resprouting traits is used and is shown in Table
\ref{tab:tab-ReSproutByEcozone}.

The fraction of fine roots contained within the combustible forest floor
layers (SW.Prop.Fine.Root.duff) can be a close to or exceeding 50\% of
the fine root biomass \citep{strong1985}, and burns alongside the
organic soils \citep{benscoter2011}. Fine root allocation is between the
organic soil (termed ``Aboveground'' or AG in CBM-CFS3 vs
``belowground'' or BG for mineral soils) is a parameter that can be RU
level and is distinct for hardwoods (HW) vs softwoods (SW). As a result,
the calculation for softwood fine root consumption and mortality rate
(SW.Mort) are as follows, using softwood as an example:

\begin{equation}
SWFineRootConsump = SW.Prop.Fine.Root.duff \times Duff.Consump.Fract
\end{equation}

\begin{equation}
SWFineRootMort.AG = SW.Mort \times SW.Prop.Fine.Root.duff \times (1-Duff.Consump.Fract) \times (1-ReSproutFactor)
\end{equation}

\begin{equation}
SWFineRootMort.BG = SW.Mort \times (1-SW.Prop.Fine.Root.duff)
\end{equation}

In contrast, the larger diameter of the coarse root biomass pool
prevents its consumption during any smouldering of the duff layer, and
the mortality rate of coarse roots is simply proportional to that of the
stemwood overall.

\subsubsection{Understory tree mortality and consumption}

Understory (or small diameter overstory) tree mortality is defined
separately in the model, but given the lack of data on diameter classes
in the severity data, robust field data on differing mortality rates of
smaller diameter trees is not available, and so the understory tree
mortality rate is set equal to the overstory rate as defined in the
table above. Note that trees with a top height less than 1.4 m are not
considered in this pool, and instead are lumped into the ``other'' pool.

\subsubsection{Litter layer area-wise consumption by severity class}

\begin{table}
\centering
\caption{\label{tab:tab-UnBurnedForestFloorByEcozone}Unburned litter area by ecozone and severity class.  The majority of the data comes from studies in the Boreal Plains and Boreal Shield West, and so values are extrapolated from those two well-observed ecozones to all others.  Ecozones are sorted in order from highest to lowest average annual burned area.}
\centering
\begin{tabular}[t]{l|r|r|r}
\hline
Ecozone & Low & Mod & High\\
\hline
BSW & 0.20 & 0.08 & 0.05\\
\hline
TP & 0.14 & 0.16 & 0.03\\
\hline
TSW & 0.20 & 0.08 & 0.05\\
\hline
BP & 0.14 & 0.06 & 0.02\\
\hline
BC & 0.14 & 0.06 & 0.02\\
\hline
BSE & 0.20 & 0.08 & 0.05\\
\hline
TSE & 0.20 & 0.08 & 0.05\\
\hline
MC & 0.14 & 0.06 & 0.02\\
\hline
HP & 0.20 & 0.08 & 0.05\\
\hline
TC & 0.14 & 0.06 & 0.02\\
\hline
PM & 0.14 & 0.06 & 0.02\\
\hline
AM & 0.14 & 0.06 & 0.02\\
\hline
MP & 0.14 & 0.06 & 0.02\\
\hline
P & 0.14 & 0.06 & 0.02\\
\hline
\end{tabular}
\end{table}

The litter layer forms the first biomass pool in which a spreading fire
consumes fuel. In low-severity fires, the litter layer may be consumed
little to no underlying duff material consumed, nor any tree mortality
\citep{hessburg2019}. Logically, since litter consumption is largely
required for the ignition of the underlying duff layer, this litter
area-wise fractional consumption also informs and constrains duff
consumption. Unburned litter area from CBI plots summarized by severity
class and ecozone is given in Table
\ref{tab:tab-UnBurnedForestFloorByEcozone}.

\subsubsection{Snag and stump consumption}

Dead standing trees and branches on average is a biomass pool much
smaller than organic soils or coarse woody debris on the ground, but in
areas of extensive insect killed trees, recent fires, and blowdown, is a
substantial biomass pool. In this initial algorithm description, only
typical boreal forest stands where snags are a small fraction of total
stems are considered. Compared to live stemwood of the same diameter,
the low moisture content of snags and snag branches allows for much
greater consumption during the passage of an intense flaming front
\citep{stocks2004}. The snag branch pool experiences almost complete
combustion in high severity fires \citep{talucci2019}, and is highly
correlated with live foliage consumption (i.e Crown Fraction Burned)
\citep{degroot2022}. The largest biomass pool of the main standing dead
stemwood remains at most approximately 50-60\% consumed. In the absence
of extensive field data from CBI plots to quantify snag and snag branch
consumption, snag (dead stemwood) consumption fraction is estimated as:

\begin{equation}
Snag Consumption fraction = \left( 0.5 \times Crown Fraction Burned \right) + 0.05
\end{equation}

And for snag branches:

\begin{equation}
Snag Branch Consumption = \left( 0.9 \times Crown Fraction Burned \right)
\end{equation}

Stumps are tracked as part of the Other pool in CBM-CFS3. As they are
typically in contact with the upper forest floor, they are assumed to be
consumed at the same rate as coarse woody debris.

\subsection{Forest Floor Consumption}

While consumption of fine fuels in the litter layer of the forest floor
is nearly complete for any given fire intensity, consumption of deeper
organic soil horizons (F+H layers in upland forests and upper peat
layers in wetlands) is more drought dependent. In the fire literature in
Canada, the soil organic layer is sometimes termed the Forest Floor Fuel
Load (\emph{FFFL}) \citep{letang2012} and is dominated by the equivalent
organic soil Slow pool (Aboveground Slow Dead Organic Matter, or
AGSlowDOM) in CBM-CFS3 Typically attention has been paid to the absolute
value of Forest Floor Fuel Consumption (\emph{FFFC}); however in the
case of carbon modelling, it is the relative fraction of consumption
(\emph{FFFC/FFFL}) that is of interest. In this scheme, we utilize a
composite of wildfire data from \citep{degroot2009} alongside the ABoVE
duff consumption data \citep{walker2020}, with an alternative modelling
approach to compute the relative amount of consumption (scalar from 0 to
1) rather than directly modelling an absolute value in kg
m\textsuperscript{-2} or cm as otherwise done in the literature. A logit
transform is used on the scalar data to make it suitable for the fitted
non-linear least-squares modelling:

\begin{equation}
logit \left( \frac{FFFC}{FFFL}\right) = [3.91(1 - e^{(-0.008BUI)})] + (-0.53log_{e}(AGSlow))
\end{equation}

where BUI is the Fire Weather Index System's Buildup Index, and AGSlow
in the CBM-CFS3 (given in Mg C/ha in this equation), and also synonymous
with the the Forest Floor Fuel Load (with ecozone averages given in
\citep{letang2012} or site-level data where observed). Assuming an
average carbon content of 50\%, CBM-CFS3 values in Mg C/ha are converted
to fire behaviour units of kg of biomass per squared metre by dividing
by 5.

\begin{table}
\centering
\caption{\label{tab:tab-DuffConsumpTable} Fire Weather, fuel loading, and duff consumption values per ecozone.  Note that FFFL values 
 are from Letang et al., (2012) and not directly from the Reconciliation Unit values provided in the National GHG Inventory Report. 
 Median Buildup Index of burning is from Barber et al., (2024)}
\centering
\begin{tabular}[t]{l|l|l|l|l}
\hline
Ecozone & Median Buildup Index of Burning & Median FFFL kg m-2 & FFFC kg m-2 & \% consumption\\
\hline
BSW & 58 & 6.864 & 2.71 & 0.39\\
\hline
TP & 79 & 11.97 & 4.97 & 0.42\\
\hline
TSW & 72 & 1.7556 & 1.12 & 0.64\\
\hline
BP & 67 & 7.1932 & 3.1 & 0.43\\
\hline
BC & 60 & 7.67844 & 2.99 & 0.39\\
\hline
BSE & 40 & 9.3522 & 2.55 & 0.27\\
\hline
TSE & 35 & 4.9764 & 1.59 & 0.32\\
\hline
MC & 112 & 4.32 & 2.88 & 0.67\\
\hline
HP & 52 & 6.1462 & 2.33 & 0.38\\
\hline
TC & 59 & 7.7616 & 2.98 & 0.38\\
\hline
PM & 60 & 13.5584 & 4.34 & 0.32\\
\hline
AM & 39 & 6.33 & 1.97 & 0.31\\
\hline
MP & 40 & 9.3522 & 2.55 & 0.27\\
\hline
P & 54 & 7.1932 & 2.66 & 0.37\\
\hline
\end{tabular}
\end{table}

While ultimately this scheme can be used on individual fires with
estimated or measured fuel loading and specific BUI values, as an
example, an ecozone-averaged fuel load and a multi-year average of BUI
during active fire satellite detection can also be used to provide
representative values to showcase the modelframework. Specifically, a
median BUI of detected fire hotspots in Canada from 2003-2021
\citep{barber2024} using the same data as the Canadian CFEEPS-FireWork
wildfire air quality model of \citep{chen2019} is presented in Table
\ref{tab:tab-DuffConsumpTable}, along with proportional consumption
values of the forest floor by ecozone.

Note that the maximum upland Forest Floor Fuel Load is approximately 30
kg m\textsuperscript{-2} (150 Mg C ha\textsuperscript{-1})
\citep{letang2012}; higher values are typically seen only in peat
ecosystems, where the above scheme is still valid, as approximately 33\%
of the data used to fit this model is for organic soil pool values over
30 kg m\textsuperscript{-2} (\textgreater150 Mg Mg C
ha\textsuperscript{-1}). Fully 10\% of the data used in the FFFC model
is for sites with over 60 kg m\textsuperscript{-2} of organic soil.
Absolute consumption values are similar between deep forest floor
organic layers and peatlands \citep{walker2020}, though relative
consumption rates decline rapidly in deeper organic soil layers, as
demonstrated in the model selected here. For Canadian peatlands, the
CaMP model \citep{bona2020} is instead used in spatially explicit
version of CBM-CFS3. Within CaMP, a separate peatland water model driven
by Drought Code determines the thickness of the unsaturated peat layer,
and an amount approximating 12\% of the thickness of the unsaturated
peat is consumed as smouldering consumption. The peat-specific carbon
pools and fire disturbance matrices are fully described in
\citep{bona2020}; large peatland trees will still utilize the DM scheme
described below. Since deeper forest organic soil and peat layers show
approximately similar absolute consumption rates, for the purpose of
this general model description no peatland-specific components of the
fire DMs are required.

\begin{table}
\centering
\caption{\label{tab:tab-CWDByEcozone}Coarse Woody debris consumption rates from pre/post measurements in experimental fires}
\centering
\begin{tabular}[t]{l|r|r|r}
\hline
Ecozone & Low & Mod & High\\
\hline
BSW & 0.024 & 0.163 & 0.140\\
\hline
TP & 0.000 & 0.218 & 0.238\\
\hline
TSW & 0.000 & 0.218 & 0.238\\
\hline
BP & 0.359 & 0.509 & 0.412\\
\hline
BC & 0.024 & 0.163 & 0.140\\
\hline
BSE & 0.080 & 0.131 & 0.182\\
\hline
TSE & 0.080 & 0.131 & 0.182\\
\hline
MC & 0.024 & 0.163 & 0.140\\
\hline
HP & 0.080 & 0.131 & 0.182\\
\hline
TC & 0.024 & 0.163 & 0.140\\
\hline
PM & 0.024 & 0.163 & 0.140\\
\hline
AM & 0.080 & 0.131 & 0.182\\
\hline
MP & 0.080 & 0.131 & 0.182\\
\hline
P & 0.359 & 0.509 & 0.412\\
\hline
\end{tabular}
\end{table}

Limited data is available on the fraction of woody debris consumption
alongside fire severity measurements. Coarse woody debris of overstory
stems that makes up 60-80\% of woody debris biomass in Canada's boreal
and temperate forests \citep{hanes2021}, with its moisture and
consumption patterns largely follows the moisture regime of the Drought
Code \citep{mcalpine1995}. In this modelling framework, the proportion
of coarse (\textgreater7.5 cm diameter) and medium (\textgreater0.5 cm
and \textless7.5 cm) woody debris consumption is estimated based on
detailed measurements of consumption from experimental fires. Coarse
Woody Debris is responsible for approximately 50-75\% of the total woody
debris load in most ecozones, and approximately 60\% of the total woody
debris consumption. Ecozone-level CWD consumption rates are summarized
in Table \ref{tab:tab-CWDByEcozone}.

Note that where historical burn severity data is not available, the fire
classification type of surface, intermittent crowning, and active crown
fire are used instead as proxies for low, moderate, and high severity
fire, respectively. Fine woody debris \textless0.5 cm in diameter is
consumed at the exact same rate as the litter pool (see section above).

\subsection{Calculation of Annual Direct C Emissions from Fire}

In an effort to compare model outputs against an independent,
large-scale set of fire atmospheric emissions, fire direct atmospheric
emissions from the record breaking Canadian 2023 fire season were
computed using the above framework. The classified burn severity product
from the National Burned Area Composite annual production was utilized
(see Hall et al 2020 for an algorithm description). In NBAC, an internal
tracking value ``NFIREID'' is utilized, which is the final
satellite-derived burned area polygon (allowing for multi-part polygons)
is split across any RU unit boundaries (if any). Since carbon pool sizes
vary across RU boundaries, this allows for a single NFIREID to be
present across multiple RUs.

A total of 2199 fires as small as 0.09 ha (one 30x30 Landsat pixel) were
mapped by NBAC for burn severity for a total of 14.60 Mha, but only
fires over 100 ha were utilized as a lower limit of where meaningful
per-fire estimates of the fraction of low, moderate, and high severity
burned area was available. Including only fires 100 ha and larger
reduced the total number of fires to 966 but the total area remained
largely the same at 14.58 Mha. A total of 189,704 ha of post-fire
salvage logging was also mapped in 2023 and is assigned to the moderate
severity class after consultation with provincial land managers. The
direct C emissions from fire shown here are not altered by the act of
post-fire salvage logging. Additionally, the NBAC mapping process
accounts for unburned islands (and areas with a mapped fire severity no
different than unburned) which count towards the total fire area but do
not have a disturbance matrix and direct C estimate applied.

A unique DM was then calculated for each fire using the median
area-weighted Buildup Index per fire. All thermal detection hotspots
from VIIRS that intersect a fire were extracted from the historical
hotspot archive that supports the Canadian smoke emissions model
CFFEPS-FireWork \citep{chen2019}. The median DC value across all
intersecting hotspots was used to derive a single DM per fire, no matter
the duration of burning.

To compute total direct fire emissions per fire, a single estimate of
the carbon pool size based on the Reconciliation Unit of the centroid of
the NFIREID polygon was applied. Since polygons are split across RU
boundaries, the spatial weighting of the pool size per fire is performed
automatically. While spatially explicit biomass maps are available for
some aboveground components \citep{guindon2024} and some organic soil
components \citep{hanes2022}, the majority of the required pool sizes in
CBM-CFS3 for the computation of the fire DMs are available only at the
spatially referenced RU scale.

\section{Results and Discussion}

\subsection{Direct fire carbon emissions as a function of fire severity
and drought}

Direct fire emissions per hectare by severity class and as a function of
drought condition (Buildup Index) are shown in Figure
\ref{fig:non-GWP-total-C-plot}. Direct fire emissions per hectare were
greatest (using a constant BUI of 200 as a practical maximum) in the
Pacific Maritime ecozone of British Columbia, with 83 and 71 t C/ha
released in high and moderate severity fires, respectively. This equates
to 295 and 252 t CO2e/ha and an MCE of 0.90 to 0.89, respectively. The
Pacific Maritime ecozone features some of the highest relative
contributions of canopy fuels to carbon release, with 32\% of modelled
emissions from the canopy, and 68\% from the surface (i.e.~woody debris,
litter, duff, roots etc). Next highest emissions were 65 t C/ha from
high severity fire in the Boreal Shield East of Labrador which features
far lower merchantable volume (see Appendix A) but overall softwood
foliage pools only 5\% smaller than the Pacific Maritime. Larger
aboveground very fast DOM (i.e.~litter) pools in the BSE of Labrador
offset slightly smaller aboveground slow DOM (i.e.~forest floor duff)
pools as compared to the PM.

The lowest emissions per area were estimated to be in the Boreal Shield
West of Saskatchewan, where total emissions of 16 and 22 t C/ha (56 and
76 t CO2e/ha) for low and moderate severity fire at a BUI of 200 were
77\% in surface fuels. Despite much lower overall emissions, these low
emissions fires still feature 77-79\% of the total emissions from
surface fuels. The Taiga Shield West of the Northwest Territories also
was modelled to have emissions of only 21.5 t C/ha under low severity
fire.

Across both low and high severity fires, the effect of drought
(i.e.~Buildup Index) was fairly consistent, the emissions from BUI 200
fires (i.e.~very dry conditions) were higher than those of identical
severity class but at a much lower BUI of 40. This lower BUI of 40
represents the practical lower limit of large spreading fires in forests
in Canada. Eastern regions such as the Mixedwood Plains of Ontario and
Quebec as well as the Atlantic Maritime of Nova Scotia and Prince Edward
Island see a modelled 60\% enhancement of emissions across the range of
simulated BUI here, while northwestern boreal regions such as the Taiga
Shield West of the Northwest Territories, BSW of Alberta, and Taiga
Cordillera of the NWT show a far smaller 10-13\% enhancement of
emissions per hectare. This patterns is consistent across both high and
low severity fire, with the BUI enhancement of emissions as high as 63\%
in low severity fires, while the effect is only somewhat smaller in high
severity fires at 54\%. Regions with a high emissions drought
sensitivity tend to be those with higher softwood (i.e.~broadleaf)
forest composition near 50\% (by volume) or more (i.e.~Atlantic
Maritime, Mixedwood Plains and also Boreal Plains of Manitoba). RUs with
the highest aboveground slow DOM pools also showed similarly high
drought sensitivity of \textasciitilde42\%. RUs with the lowest drought
sensitivity tend to also be the RUs with the lowest overall emissions.

\begin{figure}
\centering
\includegraphics{GMD-Thompson-et-al_files/figure-latex/non-GWP-total-C-plot-1.pdf}
\caption{\label{fig:non-GWP-total-C-plot}Direct emissions by severity
class: orange = high, yellow = moderate, green = low severity. Labelled
acronyms are Reconciliation Units (see Figure 1) as Ecozone then
Province, so BC is both Boreal Cordillera and British Columba. Labrador
is distinct from the island of Newfoundland for Reconciliation Units, so
`Lab' and `NFLD' is used. Some small Reconciliation Units without unique
NIR biomass pool data are not shown in this figure but are computed
still from adjacent RU from the same ecozone. Some RU with very small
burned area are also not shown in this figure.}
\end{figure}

\subsection{Comparison against existing fuel consumption models}

\begin{figure}
\centering
\includegraphics{GMD-Thompson-et-al_files/figure-latex/fig-DMs-vs-FBP-1.pdf}
\caption{\label{fig:fig-DMs-vs-FBP}Fire direct emissions by severity
class using the FireDMs approach from this study (lines) as compared to
Canadian Fire Behaviour Prediction (FBP) System outputs for Total Fuel
Consumption, based on experimental fires of differing fuel types C-2
(boreal spruce), C-3 (Jack and lodgepole pine) as well as M-1 (leafless
mixedwood 50\% conifer). An Initial Spread Index value of 14 is used in
all FBP calculations. Points are colourized by Crown Fraction Burned
(CFB).}
\end{figure}

The fire disturbance matrix approach is compared to the FBP System in
Figure \ref{fig:fig-DMs-vs-FBP}. The FBP used widely in fire growth
modelling as well as in Canada's national air quality model
\citep{chen2019}. The FBP fuel consumption values are based solely on
experimental (i.e.~controlled ignition) fires with fuel consumption
measured within less than an hour of the passage of the flaming front,
and therefore does not include the consumption from long-smouldering
fuel beds such as organic soils and large-diameter woody debris. The FBP
system emphasizes the prediction of fire intensity (i.e.~flame length)
and fire frontal spread rate, with no measurements of the complete fuel
consumption. Fuel consumption values provided by the FBP system are
dependent only on fuel type (related to leading tree species) and
Buildup Index, with no allowance for site-specific variations in fuel
load (both canopy and surface fuel loads are fixed in the FBP). Given
the need for safe experimental fire operations and the demand for
firefighting resources, experimental fires in the FBP with fuel
consumption data have a BUI of no more than 80 for the most common C-2
and C-3 fuels. A BUI of 80 exceeds the area-weighted BUI for most years
except in the Montane Cordillera ecozone (See Appendix C), though the
record-breaking 2023 fire season saw an area-weighted mean BUI for
burning days of 109.

The extrapolation of fuel consumption in the FBP beyond 80, coupled with
the lack of long-duration smouldering pools, likely accounts for the
systematic lower estimates of fuel consumption in the FBP compared to
this disturbance matrix approach shown above (see Figure
\ref{fig:fig-DMs-vs-FBP}). For more southern forests such as the Montane
Cordillera, Boreal Plains, or Boreal Shield East, larger canopy fuels
and woody debris volumes likely contribute to the larger fuel
consumption estimates in this approach compared to the FBP. For the
Taiga ecozones, smaller canopy fuel and woody debris fuel loads as well
as thin duff layers (i.e.~AGSlow in the CBM-CFS3 nomenclature) also
contribute to lower total fuel consumption that falls closer to observed
FBP Values.

\subsection{Comparison against observed direct fire emissions}

\subsubsection{\texorpdfstring{Forest fire observations of CO and
CO\textsubscript{2} emissions
ratios}{Forest fire observations of CO and CO2 emissions ratios}}

\begin{table}
\centering\centering
\caption{\label{tab:MCE-from-airborne}Comparison of the Modified Combustion Efficiency (MCE) of airborne gas measurements of Canadian wildfires against modelled MCE}
\centering
\resizebox{\ifdim\width>\linewidth\linewidth\else\width\fi}{!}{
\begin{tabular}[t]{l|l|l|l|l|l|l}
\hline
Study & Date & Subset & Ecozone & Buildup Index & Obs MCE & Modelled MCE\\
\hline
Hornbrook et al 2011 & 2008-07-01 & Afternoon (0.2Low 0.4Mod 0.4High) & Boreal Shield West & 61 & 0.92 & 0.91\\
\hline
Hornbrook et al 2011 & 2008-07-01 & Late Evening (100\% low severity) & Boreal Shield West & 61 & 0.82 & 0.898\\
\hline
Hornbrook et al 2011 & 2008-07-04 & After rain smouldering only low severity & Boreal Shield West & 60 & 0.83 & 0.898\\
\hline
\end{tabular}}
\end{table}

Mean observed Modified Combustion Efficiency was observed by
\citet{hornbrook2011} for distinct periods during the ARCTAS campaign
over northern Saskatchewan in 2008. MCE observed by aircraft during the
peak burn period when the majority of fuel consumption and area burned
occurs largely corresponded with modelled values (Table
\ref{tab:MCE-from-airborne}), suggesting the model provides a fair
representation of the balance of flaming and smouldering during large
active wildfires. Subsequent smoke plume observations during periods of
greatly reduced spread and intensity (late evening and after rain)
showed a substantially reduced MCE of 0.82--0.83 that indicates a near
lack of flaming combustion. Even when represented as 100\% low severity
fire in the model, the modelled MCE only declines to 0.903. Since the
fire DM model is based on area burned, fire activity such as smouldering
but with minimal actual area burned increased is going to show an MCE
far lower than areas of low severity fire spread, which are typically
still a low sub-canopy flaming front that features a mix of smouldering
duff and woody debris alongside flaming consumption of litter
\citep{mcrae1994, mcrae2017}. During the same ARCTAS campaign over a
larger sample of 39 fires, \citet{vayPatternsCO2Radiocarbon2011}
observed a mean MCE of 0.89 for smoke plumes within 2 km of a large
wildfire.

Comprehensive compilations of emissions factors from
\citet{binteshahidNEIVAv10NextgenerationEmissions2024} and
\citet{andreaeEmissionTraceGases2019} for boreal forest fuels show
N\textsubscript{2}O emissions factors of 0.205 and 0.24 g/kg,
respectively. The N\textsubscript{2}O observations from
\citet{andreaeEmissionTraceGases2019} are sourced from observations an
aggregated MCE of 0.89, similar to our mean modelled MCE of
\textasciitilde0.91. Alternatively, the widely cited emissions factors
for N\textsubscript{2}O as 0.77\% of vegetation N
\citep{lobertImportanceBiomassBurning1990, worldmeteorologicalorganizationAtmosphericOzone19851985, olivierEmissionDatabaseGlobal1994, davidsonInventoriesScenariosNitrous2014}
stems from a small number of observations over two Colorado wildfires by
\citet{crutzenBiomassBurningSource1979} that gives a similar result of
N\textsubscript{2}O:CO\textsubscript{2} \textasciitilde0.22\% by volume,
or approximately 0.2 g/kg.

\subsection{2023 Canadian wildfire season direct emissions}

To calculate the sum total of estimate direct C emission from Canadian
wildfires in 2023, severity maps were created for each fire using 2024
season satellite observations using an approach that produces severity
class maps using both classified spectral indices \citep{whitman2020}
with additional fire environment covariates to predict severity class at
30-m !!!!{[}cite Ellen's upcoming paper{]}!!!. A single representative
area-weighted Buildup Index value from fire hotspots was computed for
each fire. A single set of Fire DMs (composed of one DM per severity
class) was then applied to Reconciliation Unit carbon stock values for
the total fire area falling into Low, Moderate, and High severity
classes. The same Buildup Index was used for all severity classes per
fire event when computing the DM per fire event. Per-fire direct C
emissions (i.e.~CO\textsubscript{2}, CO, CH\textsubscript{4},
PM\textsubscript{2.5}, etc.) were then computed and summed by ecozone
and nationally.

RU-level carbon stocks were extracted from the National Forest Carbon
Monitoring Accounting and Reporting System (NFCMARS) CBM-CFS3 model
results and limited to their state at the end of 2022. Given the
extremes of the 2023 fire season, it is reasonable to assume that
wildfires occurred indiscriminately across the forested landscape rather
than being focused on specific forest types or significantly steered
away from fire suppression zones - applying regionally average carbon
stocks is correspondingly appropriate.

\begin{figure}
\centering
\includegraphics{GMD-Thompson-et-al_files/figure-latex/Emissions-2023-by-RU-1.pdf}
\caption{Comparison of CO2e emissions per hectare between this method at
the 2025 National Inventory Report for the 2023 wildfires. Dots are
labelled by RU and coloured by ecozone (see Figure 1), with dot size
proportional to 2023 area burned.}
\end{figure}

\begin{landscape}\begin{table}[!h]
\centering
\caption{\label{tab:tab-RU-CO2e-comp}Comparison of per hectare CO2e direct emissions from this study against the 2025 National Inventory Report for the 2023 wildfire season in Canada. Rows are sorted by decreasing 2023 area burned.  Note that for RUs with no managed forest area and no biomass pool data available.}
\centering
\resizebox{\ifdim\width>\linewidth\linewidth\else\width\fi}{!}{
\begin{tabular}[t]{>{\centering\arraybackslash}p{2cm}|>{\centering\arraybackslash}p{2cm}|>{\centering\arraybackslash}p{2cm}|>{\centering\arraybackslash}p{2cm}|>{\centering\arraybackslash}p{2cm}|>{\centering\arraybackslash}p{2cm}|>{\centering\arraybackslash}p{2cm}|>{\centering\arraybackslash}p{2cm}|>{\centering\arraybackslash}p{2cm}}
\hline
Jurisdiction & Ecozone & Reconciliation Unit & Area Burned (Mha) & Moderate Severity Fraction & High Severity Fraction & Emissions t C/ha & This Study Emissions t CO2e/ha & NIR 2025 Emissions tCO2e/ha\\
\hline
Lab+QC & taiga shield east & 3+14 & 2.50 & 0.39 & 0.31 & 42.3 & 148.8 & 99\\
\hline
Northwest Territories & taiga plains & 50 & 2.01 & 0.23 & 0.52 & 26.7 & 94.0 & 114\\
\hline
Alberta & boreal plains & 34 & 1.53 & 0.30 & 0.42 & 44.0 & 154.4 & 163\\
\hline
Northwest Territories & taiga shield west & 51 & 1.50 & 0.19 & 0.58 & 22.7 & 79.9 & 129\\
\hline
Quebec & boreal shield east & 15 & 1.38 & 0.39 & 0.27 & 30.2 & 106.2 & 92\\
\hline
British Columbia & taiga plains & 38 & 1.29 & 0.17 & 0.65 & 35.0 & 123.0 & 161\\
\hline
Alberta & taiga plains & 31 & 0.88 & 0.18 & 0.67 & 37.0 & 129.7 & 144\\
\hline
Saskatchewan & boreal shield west & 27 & 0.70 & 0.19 & 0.65 & 19.2 & 67.6 & 83\\
\hline
British Columbia & montane cordillera & 42 & 0.68 & 0.26 & 0.39 & 36.2 & 126.8 & 189\\
\hline
ON+QC+MB & hudson plains & 18+13+25 & 0.47 & 0.23 & 0.42 & 31.8 & 111.8 & 128\\
\hline
Saskatchewan & boreal plains & 28 & 0.31 & 0.22 & 0.41 & 29.6 & 103.9 & 91\\
\hline
AB+MB+SK & taiga shield west & 32+21+26 & 0.28 & 0.23 & 0.60 & 40.1 & 140.6 & 146\\
\hline
Ontario & boreal shield west & 16 & 0.25 & 0.23 & 0.47 & 23.8 & 83.5 & 94\\
\hline
Yukon Territory & taiga cordillera & 45 & 0.20 & 0.16 & 0.49 & 48.7 & 170.9 & 198\\
\hline
British Columbia & boreal cordillera & 40 & 0.14 & 0.12 & 0.56 & 49.5 & 173.8 & 209\\
\hline
Yukon Territory & boreal cordillera & 46 & 0.11 & 0.16 & 0.58 & 50.0 & 175.8 & 179\\
\hline
Manitoba & boreal shield west & 22 & 0.08 & 0.20 & 0.53 & 21.4 & 75.2 & 84\\
\hline
British Columbia & boreal plains & 39 & 0.08 & 0.15 & 0.50 & 34.5 & 121.2 & 176\\
\hline
Northwest Territories & boreal plains & 52 & 0.05 & 0.36 & 0.17 & 50.9 & 178.5 & 187\\
\hline
Ontario & boreal shield east & 19 & 0.04 & 0.31 & 0.41 & 26.7 & 94.0 & 107\\
\hline
\end{tabular}}
\end{table}
\end{landscape}

\begin{table}
\centering
\caption{\label{tab:tab-Model-comp-Emissions-2023-table}Comparison of total carbon emissions estimates for the Canadian 2023 fire season.  Note that the Byrne et al 2024 estimate shown here uses the emissions ratio for CO inferred from this study.  See main text for details.}
\centering
\begin{tabular}[t]{c|c|c}
\hline
Source & Approach & TotalEmission\\
\hline
This study & Severity-Inventory & 494\\
\hline
Byrne 2024 & Total column CO anomaly & 510*\\
\hline
CAMS-GFAS & Fire Radiative Power & 478\\
\hline
FireWork & Hotspot-FBP & 168\\
\hline
\end{tabular}
\end{table}

A total of 494 Mt C was estimated to be released directly by fires in
Canada in 2023. The area-weighted mean Buildup Index for all fires in
Canada in 2023 was 109, representing severe but not uniformly
exceptional drought conditions in Canada's northern forests. The
area-weighted mean total C emissions per unit area was 33.88 t C/ha,
though 90\% of the emissions per unit area were between the 5th
percentile of 14 t C/ha typical of the Taiga Plains ecozone under low
drought conditions and the 95th percentile of 54 t C/ha typical of
Pacific coastal forests.

Comparison of emissions totals against the NIR is only possible for
spatial units (i.e.~Reconciliation Units, provinces, or ecozones) where
the entire RU is composed of managed forest. In British Columbia, across
2.2 Mha burned, this FireDMs approach estimates were consistently lower
than NIR. This was driven by a lower average emissions in FireDMs for
both the Taiga Plains with 1.3 Mha burned (124 t CO2e/ha) and Montane
Cordillera with 0.68 Mha burned (127 t CO2e/ha) as compared to the NIR
(161 and 189 t CO2e/ha, respectively). For Alberta the 1.5 Mha of Boreal
Plains ecozone burned showed close agreement (155 vs 163) as well as in
the Taiga Plains (130 vs 144). The Taiga Plains fires of both Alberta
and British Columbia were largely adjacent, and the FireDMs output was
very similar (123 vs 129) across nearly identical moderate and high
severity fractions, but the NIR showed a larger contrast (144 vs 161)
that may originate from the underlying assumptions of the previous
methods of calculating fire emissions in the NIR.

The RU with the largest burned area in 2023 was in the unmanaged forest
of Quebec's Taiga Shield East, where NIR pool sizes from the adjacent
jurisdiction of Labrador TSE were used in the FireDMs estimates, as no
NIR pool sizes were available for Quebec's TSE ecozone area. Over this
2.5 Mha of burned area, the Quebec TSE FireDMs emissions estimates were
substantially larger (149 t CO2e/ha) compared to NIR (99 t CO2e/ha). In
the Quebec Boreal Shield East ecozone where NIR pools are available, the
FireDMs were still larger than NIR, though the gap between the two was
much smaller (106 vs 92). Only the Manitoba Boreal Plains and Yukon
Taiga Plains showed a similarly larger estimate in the FireDMs compared
to NIR, with all other RUs showing a typically 10-30\% reduction in
emissions in the FireDMs compared to the NIR. Overall, this new
emissions estimate method will yield more representative fire emissions
estimates that are sensitive to annual differences in drought condition
and fire-specific patterns of severity. The small overall impact is
likely due to larger soil organic layer emissions in the FireDMs being
offset by lower canopy consumption in the low and moderate severity
areas.

\citet{byrneCarbonEmissions20232024} used observed total atmospheric
column excess CO and a range of MCE values to estimate total fire C
emissions in Canada in 2023 of between 570-727 Mt C with a mean estimate
of 647 Mt. The uncertainty in the estimate from
\citet{byrneCarbonEmissions20232024} lies primarily in the uncertain
CO\textsubscript{2}:CO ratio (more commonly computed as the normalized
ratio MCE) that was estimated to be between 7.8--10.8 g
CO\textsubscript{2} / g CO. Our bottom-up estimates that partition
flaming vs smouldering shows a lower CO\textsubscript{2}:CO estimate of
6. This lower CO\textsubscript{2}:CO ratio (i.e.~more smouldering) if
applied to the data from \citet{byrneCarbonEmissions20232024} would
reduce their lower range of total C emissions to 505 Mt C, which is
comparable to the 494 Mt C computed from this bottom-up approach.
Estimates of total carbon emissions based solely on the sum of observed
Fire Radiative Power (FRP) from Copernicus-CAMS (GFASv1.2
\citep{kaiserBiomassBurningEmissions2012}) was 478 Mt C
\citep{jonesStateWildfires202320242024}. The operational air quality
model for Canada produces wildfire smoke emissions estimates using the
CFFEPS-FireWork framework \citep{chen2019} which utilizes active fire
satellite detections coupled to biomass burning rates from the Canadian
Forest Fire Danger Rating System which are known to underestimate total
emissions from forest floor combustion \citep{degroot2009}. As a result,
the CFFEPS-FireWork total emissions in 2023 with their exceptional
nation-wide moisture deficits
\citep{jainDriversImpactsRecordBreaking2024} underestimate total
emissions compared to other methods, with CFFEPS estimates of CO
emissions only \textasciitilde40\% of satellite observed values
\citep{griffinBiomassBurningCO2024, voshtaniQuantifyingCOEmissions2025}.
Even after assimilating satellite CO and tower-based total carbon
measurements, CFFEPS still retains 20\% understimating bias, likely
attributable to an estimate of peak CO (i.e.~smouldering) emissions that
are lower than reality \citep{voshtaniQuantifyingCOEmissions2025}.

\subsubsection{Relationship to other estimates}

The forest floor emissions modelling scheme used here builds upon the
CanFIRE model \citep{degroot2013} which incorporates a similar estimate
of absolute forest floor emissions based on fuel type and various Fire
Weather Index inputs (including but not limited to Buildup Index). While
the CanFIRE model is also capable of estimating tree mortality, the
premise of CanFIRE lies in the use of the Canadian Fire Behaviour
Prediction (FBP) System to model fire intensity and a
physiological-thermodynamic model of tree damage and mortality. CanFIRE
is able to run entirely in scenario or forecast mode, i.e.~no fire
severity map is needed to run the model, unlike the focus here on a
mechanism to estimate carbon fluxes and pools after fire severity is
mapped. Ultimately, models like CanFIRE can be used for near-real time
emissions estimates of direct and indirect fire carbon emissions (and
also for prescribed fire planning and scenario testing), while this
modelling framework is best used solely operational carbon accounting
and reporting given its strong dependence on severity maps. While fire
severity can be estimated empirically based on geospatial inputs without
observation using multispectral satellite data such as Landsat
!!!{[}Ellen's paper?{]}!!!, estimation using coupled fire behaviour and
ecology models such as CanFIRE is a more direct approach. Under severe
drought conditions and light winds, very large fires often create more
wind and energy than is available in gradient winds as measured at
nearby airports and fire weather observing stations
\citep{clarkAnalysisSmallScaleConvective1999}. Thus, fire behaviour and
fuel consumption models solely driven by wind inputs from distant
meteorological stations can under-predict fire intensity and therefore
likely fire severity. While fire-atmosphere models are available for
research \citep{coenGenerationForecastExtreme2018} and some forecasting
and planning purposes \citep{linn2020}, modelling resources are
typically not made available for the extensive and largely unsuppressed
fires in northern Canada. Fire severity mapping after the fact is able
to account for high fire severity \citep{whitman2018} even when fire
spread occurs under light winds more associated with lower fire
intensity \citep{whitman2024}.

While the fire DMs presented here are designed for use in NFCMARS for
GHG estimation and reporting, their simplicity and similarity to many
other forest carbon schema allows them to be used elsewhere such as in
other forest dynamics \citep{brecka2020} and earth system models
\citep{melton2020} that require estimates of Canadian forest carbon
emissions from wildfire. \citet{stenzel2019} showed that
severity-informed fire C estimates (with snags) in the US are only
30-40\% as large as fixed or variable severity data using their
LANDIS-type models. Large over-estimates of in-fire bole consumption in
models were not observed to correspond to measurements in the field. In
the model presented here, our estimates of snag consumption are based on
simple but logical relationships with Crown Fraction Burned, but more
comprehensive data collection and field observations would be required
to extend the accuracy of the snag consumption scheme.

\subsubsection{Indirect fire emissions}

While fire emissions within the year of the fire are majority of the GHG
flux, enhanced post-fire decomposition of dead biomass persists for many
years after a fire. The increase in post-fire tree mortality between the
year of the fire and the year following the fire is as much as 30\% in
low and moderate-severity stands \citep{angers2011}. This extended
mortality is not captured in year-of fire severity mapping but is likely
better assessed using fire severity data captured the year following
fire, as is operation practice in Canada !!!!{[}cite Ellen's upcoming
paper{]}!!!. Similarly, the transition of fire-killed stems (snags) from
upright to the forest floor woody debris takes between 5-8 years for
50\% of stems to fall \citep{angers2011} with stemfall rate highest in
low-severity fire. Field observations show that initial assessments of
fire severity are the primary predictors of snag fall \citet{angers2011}
and decomposition \citep{boulanger2011} rates, meaning the mapped
severity method used here provides a useful input for multi-year fire
carbon modelling of the snag pool in CBM-CFS3. For organic soil pools,
fire has been shown to suppress decomposition rates in upland soil pools
where drier post-fire conditions are present \citep{holden2015}, while
in permafrost upland \citep{odonnell2011} and peatland
\citep{gibson2018, gibson2019} systems, fire has profound impacts on
soil carbon pools by rapidly increasing thaw depths and subsequent
decomposition rates. Currently in CBM-CFS3, decomposition rates are
calibrated against field decomposition litterbag experiments
\citep{trofymow2002} and annual climate metrics that do not take into
account how disturbances such as fire (and fire of varying severity)
modify decomposition rates in the absence of a change in climate.

\subsubsection{Model gaps}

Currently, the model framework only accounts of regionally-averaged soil
carbon stocks, meaning that Reconciliation Units with high peatland
areas will show higher organic soil slow pool size, but peatlands
themselves are not spatially represented in the model, and instead
peatlands influence the mean reported Aboveground Slow pool size.
Currently, fire disturbance in peatlands is performed by the CaMP model,
which uses the closely related Drought Code fire weather metric to
estimate peat layer consumption \citep{bona2020} and uses the overstory
carbon schema of CBM-CFS3 to estimate overstory tree carbon pools and
fluxes. Direct satellite monitoring of upland forest organic soil and
peatland carbon loss via burn severity monitoring is possible but
remains a challenge to implement with accuracy for monitoring purposes
\citep{bourgeau-chavez2020}. Despite the large carbon stock (c.~500 Mg
C/ha from \citet{beilmanPeatCarbonStocks2008}), the mean C emissions per
area in peatland due to fire (65 t C/ha) as modelled by \citet{bona2024}
are comparable to regional emissions from peat-rich ecozones such as the
Boreal and Taiga Plains under moderate to severe drought conditions.
Indeed, it is under these moderate to severe drought conditions under
which peatlands burn more frequently \citep{turetsky2004, thompson2019}
and severely \citep{kuntzemann2023}.

This algorithm is currently built off regionally averaged biomass pools
by Reconciliation Unit (Appendix A). While Canadian wildfires shown an
overall selection bias towards preferentially burning older conifer
stands \citep{bernier2016}, fuel selectivity against less flammable
portions of the landscape decreases strongly under severe drought
conditions \citep{parks2018} such as those experienced in Canada in 2023
\citep{jainDriversImpactsRecordBreaking2024} where we performed the
algorithm validation against independent fire emissions observations.
Algorithm evaluation against individual fires or years of less profound
drought would be more likely to highlight the limitation of the
regionally-averaged carbon pool (i.e.~fuel load) values used here.
Ultimately, the mapped severity used here is best paired with mapped
carbon pools at a similar scale, such as the 1-ha forest carbon
modelling for analysis and scenario testing used by \citet{smyth2024}.
Future work will extend this fire disturbance algorithm to finer-scale
carbon reporting using gridded data as it becomes available for annual
reporting purposes.

Species-specific traits in trees with overlapping ranges, such as
resprouting in aspen \citep{brown1987} but not other broadleaves such as
oak or maple, are not explicitly handled in this model. Instead, the
relative abundance of trees with contrasting traits that influence
ecological outcomes such as fire survival are lumped at the ecozone
scale from the aggregation of the plot data. While ecozone-level
contrasts in key considerations such as overstory mortality do strongly
vary by ecozone (Table \ref{tab:tab-ReSproutByEcozone}), the same DM
(e.g.~hardwood root mortality) is applied in adjacent stands at the same
severity class despite the potential for strongly contrasting traits
that are readily tied back to mapping at the stand level by leading
species. Alternately, models such as CanFIRE can account for
species-level contrasts in fire ecology traits, but compiling the
relevant ecophysiological properties and traits for all tree species in
Canada remains incomplete at the current time.

\section{Conclusions}

Carbon emissions from wildfires in Canada represent a substantial pulse
input of greenhouse gases to the atmosphere: the 2023 fire emissions in
Canada were an input of approximately 494 Mt C. A modelling framework to
extend the current fire and greenhouse gas reporting in Canada is
presented. In contrast to the existing modelling system that utilizes a
fixed fire severity (i.e.~tree mortality) assumption, a field-calibrated
satellite fire severity mapping process that follows mature and
well-established scientific methods is used. Above-ground carbon pool
changes (from live to dead and in situ as well as to the gas phase) rely
on a robust field observation dataset that relates back to satellite
metrics. Evaluation of the modelling presented above against fully
independent airborne observations of CO:CO\textsubscript{2} emissions
ratios for boreal wildfires indicates the modelling of proportion of
flaming vs smouldering emissions are well-replicated by the model. With
confidence in this modelled CO:CO\textsubscript{2} ratio, the total fire
CO emissions for the 2023 wildfire season in Canada (with a record 15
Mha burned) is broadly consistent with fire season estimates using
satellite total column CO anomalies.

While this prototype carbon flux framework utilizes regionally-averaged
carbon stock estimates, future systems deployment within Canada's carbon
accounting system is likely to incorporate precisely mapped (30 m)
carbon pools alongside the finely mapped fire severity products (30 m)
used here. As both the spatial estimates of carbon stocks across
Canada's forested ecosystems improves alongside additional field
observations, it is anticipated that this modelling framework will
increase in both accuracy and precision over time.

\section{Appendix A: Mean pool size by Reconciliation Unit}

\begin{landscape}\begin{table}[!h]
\centering
\caption{\label{tab:PoolSizeByRU}Carbon pool size (Mg C/ha) for canopy fuels by Reconciliation Unit for 2023. RU are sorted in decreasing size of Softwood Merchantable, the largest average surface pool.  Columns are in decreasing mean pool size.}
\centering
\resizebox{\ifdim\width>\linewidth\linewidth\else\width\fi}{!}{
\begin{tabular}[t]{c|c|c|>{\centering\arraybackslash}p{2cm}|>{\centering\arraybackslash}p{2cm}|>{\centering\arraybackslash}p{2cm}|>{\centering\arraybackslash}p{2cm}|>{\centering\arraybackslash}p{2cm}|>{\centering\arraybackslash}p{2cm}|>{\centering\arraybackslash}p{2cm}|>{\centering\arraybackslash}p{2cm}|>{\centering\arraybackslash}p{2cm}|>{\centering\arraybackslash}p{2cm}}
\hline
Ecozone & Jurisdiction & RU & Softwood Merchantable & Softwood Other & Hardwood Merchantable & Softwood Stem Snag & Hardwood Other & Softwood Foliage & Hardwood Stem Snag & Softwood Branch Snag & Hardwood Branch Snag & Hardwood Foliage\\
\hline
PM & British Columbia & 41 & 75.7 & 33.9 & 3.1 & 10.1 & 1.2 & 9.3 & 0.4 & 2.3 & 0.1 & 0.1\\
\hline
TP & Yukon Territory & 44 & 33.3 & 13.0 & 27.5 & 5.7 & 12.1 & 3.5 & 4.7 & 0.8 & 0.7 & 1.1\\
\hline
PM & Yukon Territory & 47 & 30.3 & 17.1 & 4.1 & 5.8 & 2.9 & 4.7 & 1.3 & 1.4 & 0.3 & 0.2\\
\hline
HP & Ontario & 18 & 29.8 & 9.4 & 3.9 & 4.8 & 1.6 & 4.0 & 1.0 & 0.7 & 0.1 & 0.3\\
\hline
BC & Yukon Territory & 46 & 26.7 & 15.7 & 3.9 & 6.0 & 2.5 & 3.9 & 1.1 & 1.7 & 0.3 & 0.2\\
\hline
MC & Alberta & 36 & 25.8 & 19.2 & 3.1 & 3.4 & 2.5 & 3.8 & 0.4 & 1.6 & 0.2 & 0.2\\
\hline
BC & British Columbia & 40 & 24.5 & 22.7 & 2.8 & 4.9 & 2.1 & 4.6 & 0.4 & 2.5 & 0.2 & 0.2\\
\hline
MC & British Columbia & 42 & 23.7 & 13.7 & 2.8 & 7.2 & 1.1 & 3.6 & 0.5 & 1.9 & 0.1 & 0.1\\
\hline
BC & Northwest Territories & 54 & 21.1 & 17.0 & 1.3 & 8.4 & 1.0 & 3.6 & 0.7 & 1.4 & 0.1 & 0.1\\
\hline
TC & Yukon Territory & 45 & 20.7 & 12.2 & 4.4 & 6.9 & 4.1 & 3.4 & 2.2 & 1.4 & 0.4 & 0.3\\
\hline
BSW & Ontario & 16 & 20.3 & 7.7 & 1.2 & 4.5 & 0.5 & 3.2 & 0.4 & 0.8 & 0.1 & 0.1\\
\hline
BSE & Newfoundland & 1 & 20.1 & 16.2 & 2.7 & 2.3 & 2.9 & 9.3 & 0.4 & 1.1 & 0.2 & 0.8\\
\hline
TC & Northwest Territories & 53 & 18.6 & 11.1 & 0.4 & 3.8 & 0.2 & 2.5 & 0.1 & 0.9 & 0.0 & 0.0\\
\hline
BSE & Labrador & 4 & 17.8 & 14.8 & 0.5 & 6.2 & 0.5 & 8.8 & 0.2 & 1.3 & 0.0 & 0.1\\
\hline
BP & Alberta & 34 & 16.5 & 15.1 & 5.2 & 3.0 & 4.8 & 3.4 & 1.8 & 1.6 & 0.6 & 0.2\\
\hline
TSE & Labrador & 3 & 16.3 & 13.4 & 0.1 & 9.1 & 0.2 & 7.1 & 0.1 & 1.2 & 0.0 & 0.0\\
\hline
BSE & Quebec & 15 & 15.8 & 6.9 & 8.8 & 2.9 & 4.4 & 1.9 & 1.1 & 0.6 & 0.3 & 0.6\\
\hline
AM & Quebec & 11 & 15.7 & 7.4 & 11.0 & 2.1 & 5.7 & 1.3 & 1.2 & 0.6 & 0.4 & 0.6\\
\hline
AM & Prince Edward Island & 6 & 14.4 & 6.4 & 12.6 & 1.4 & 7.3 & 2.7 & 1.3 & 0.4 & 0.4 & 1.3\\
\hline
BSW & Manitoba & 22 & 14.0 & 4.9 & 4.2 & 4.4 & 1.4 & 1.8 & 1.2 & 0.6 & 0.2 & 0.3\\
\hline
BSE & Ontario & 19 & 13.5 & 4.9 & 11.7 & 1.8 & 4.9 & 2.0 & 1.7 & 0.3 & 0.3 & 0.9\\
\hline
TP & Alberta & 31 & 12.3 & 11.4 & 2.5 & 4.0 & 2.5 & 1.9 & 2.9 & 1.3 & 0.4 & 0.1\\
\hline
AM & New Brunswick & 7 & 12.1 & 7.6 & 9.6 & 1.3 & 7.6 & 2.2 & 1.0 & 0.5 & 0.5 & 0.8\\
\hline
TP & British Columbia & 38 & 11.8 & 14.9 & 12.4 & 1.9 & 6.8 & 2.0 & 2.1 & 1.0 & 0.5 & 0.5\\
\hline
BP & Northwest Territories & 52 & 10.8 & 20.2 & 2.1 & 2.5 & 7.2 & 1.8 & 0.5 & 2.2 & 0.8 & 0.4\\
\hline
BP & Saskatchewan & 28 & 9.2 & 3.3 & 10.8 & 2.9 & 2.7 & 1.3 & 2.6 & 0.3 & 0.3 & 0.4\\
\hline
TP & Northwest Territories & 50 & 9.0 & 6.3 & 2.9 & 2.6 & 3.4 & 1.1 & 0.9 & 0.5 & 0.3 & 0.2\\
\hline
BSW & Saskatchewan & 27 & 8.9 & 2.9 & 1.8 & 5.1 & 0.6 & 1.1 & 1.1 & 0.5 & 0.1 & 0.1\\
\hline
AM & Nova Scotia & 5 & 8.8 & 2.9 & 17.6 & 1.1 & 7.4 & 1.0 & 1.9 & 0.2 & 0.4 & 1.2\\
\hline
BP & Manitoba & 23 & 8.1 & 3.4 & 12.0 & 2.2 & 4.2 & 1.4 & 3.3 & 0.4 & 0.4 & 1.5\\
\hline
BP & British Columbia & 39 & 7.3 & 8.7 & 19.7 & 2.1 & 9.2 & 1.8 & 2.9 & 1.0 & 0.8 & 0.7\\
\hline
TSW & Alberta & 32 & 6.5 & 11.5 & 2.2 & 3.0 & 5.0 & 2.6 & 1.5 & 1.3 & 0.5 & 0.2\\
\hline
TSW & Northwest Territories & 51 & 4.6 & 10.5 & 0.7 & 2.1 & 1.3 & 1.7 & 0.4 & 1.7 & 0.1 & 0.1\\
\hline
BSW & Alberta & 33 & 1.1 & 6.2 & 1.2 & 1.0 & 7.0 & 1.5 & 1.4 & 1.2 & 1.6 & 0.3\\
\hline
\end{tabular}}
\end{table}
\end{landscape}

\begin{landscape}\begin{table}[!h]

\caption{\label{tab:PoolSizeByRU}Carbon pool (Mg C/ha) for sub-canopy fuels by Reconciliation Unit for 2023.  RU are sorted in decreasing size of Aboveground Slow DOM, the largest average surface pool.  Columns are in decreasing mean pool size.}
\centering
\resizebox{\ifdim\width>\linewidth\linewidth\else\width\fi}{!}{
\begin{tabular}[t]{c|c|c|>{\centering\arraybackslash}p{2cm}|>{\centering\arraybackslash}p{2cm}|>{\centering\arraybackslash}p{2cm}|>{\centering\arraybackslash}p{2cm}|>{\centering\arraybackslash}p{2cm}|>{\centering\arraybackslash}p{2cm}|>{\centering\arraybackslash}p{2cm}|>{\centering\arraybackslash}p{2cm}|>{\centering\arraybackslash}p{2cm}|>{\centering\arraybackslash}p{2cm}}
\hline
Ecozone & Jurisdiction & RU & Aboveground Slow DOM & Medium DOM & Aboveground Fast DOM & Aboveground Very Fast DOM & Softwood Coarse Roots & Hardwood Coarse Roots & Belowground Fast DOM & Belowground Very Fast DOM & Softwood Fine Roots & Hardwood Fine Roots\\
\hline
PM & British Columbia & 41 & 58.3 & 19.1 & 16.1 & 8.7 & 24.1 & 0.8 & 3.0 & 1.4 & 2.3 & 0.1\\
\hline
PM & Yukon Territory & 47 & 46.1 & 32.2 & 22.8 & 25.5 & 10.1 & 3.5 & 3.6 & 4.1 & 1.4 & 0.5\\
\hline
TP & Yukon Territory & 44 & 46.0 & 22.3 & 14.7 & 15.4 & 9.9 & 9.5 & 3.2 & 2.5 & 1.1 & 1.0\\
\hline
BSE & Newfoundland & 1 & 44.2 & 5.9 & 9.3 & 10.2 & 8.7 & 2.7 & 1.5 & 1.3 & 1.4 & 0.4\\
\hline
TC & Yukon Territory & 45 & 42.0 & 29.2 & 19.9 & 14.7 & 7.0 & 3.5 & 3.5 & 2.9 & 1.0 & 0.6\\
\hline
BSE & Labrador & 4 & 41.5 & 13.0 & 11.2 & 11.4 & 7.7 & 0.2 & 1.6 & 1.7 & 1.5 & 0.0\\
\hline
BP & Manitoba & 23 & 39.5 & 12.7 & 5.4 & 12.0 & 2.3 & 5.1 & 1.4 & 1.5 & 0.5 & 1.0\\
\hline
AM & New Brunswick & 7 & 38.8 & 7.0 & 9.8 & 6.9 & 4.1 & 5.4 & 1.8 & 1.1 & 0.7 & 0.9\\
\hline
BP & Northwest Territories & 52 & 38.5 & 7.4 & 20.8 & 9.9 & 6.2 & 3.8 & 2.2 & 2.3 & 1.1 & 0.7\\
\hline
BP & Alberta & 34 & 37.8 & 10.4 & 14.8 & 6.6 & 6.7 & 3.2 & 2.4 & 1.6 & 1.0 & 0.6\\
\hline
BC & British Columbia & 40 & 37.6 & 9.6 & 18.0 & 9.8 & 10.0 & 1.0 & 2.2 & 2.1 & 1.5 & 0.2\\
\hline
MC & Alberta & 36 & 35.6 & 12.2 & 15.7 & 7.7 & 9.4 & 2.6 & 2.4 & 1.8 & 1.4 & 0.4\\
\hline
AM & Prince Edward Island & 6 & 35.4 & 6.1 & 5.8 & 7.4 & 4.4 & 5.0 & 1.1 & 1.0 & 0.8 & 0.9\\
\hline
BC & Yukon Territory & 46 & 34.8 & 21.2 & 16.1 & 11.8 & 9.0 & 2.9 & 2.8 & 2.6 & 1.3 & 0.4\\
\hline
AM & Quebec & 11 & 33.8 & 9.0 & 8.7 & 7.1 & 4.6 & 5.0 & 1.8 & 1.3 & 0.8 & 0.9\\
\hline
BSE & Quebec & 15 & 32.0 & 10.7 & 8.1 & 8.4 & 4.6 & 4.1 & 1.7 & 1.5 & 0.9 & 0.7\\
\hline
AM & Nova Scotia & 5 & 31.4 & 6.0 & 4.9 & 6.1 & 2.2 & 6.3 & 1.1 & 0.9 & 0.6 & 1.2\\
\hline
TSE & Labrador & 3 & 30.7 & 15.2 & 10.3 & 8.0 & 6.7 & 0.1 & 1.9 & 1.8 & 1.5 & 0.0\\
\hline
BP & British Columbia & 39 & 29.3 & 10.1 & 10.9 & 7.8 & 3.3 & 5.9 & 1.6 & 1.4 & 0.7 & 0.8\\
\hline
MC & British Columbia & 42 & 27.8 & 12.9 & 11.5 & 5.3 & 7.9 & 0.8 & 2.3 & 1.2 & 1.2 & 0.1\\
\hline
BSE & Ontario & 19 & 27.3 & 8.1 & 5.7 & 7.9 & 3.8 & 5.0 & 1.2 & 1.3 & 0.7 & 0.9\\
\hline
HP & Ontario & 18 & 26.3 & 14.6 & 6.5 & 7.8 & 8.2 & 2.5 & 1.4 & 1.8 & 1.4 & 0.5\\
\hline
TP & British Columbia & 38 & 25.7 & 7.6 & 10.1 & 7.8 & 5.4 & 3.9 & 1.4 & 1.6 & 1.0 & 0.5\\
\hline
TP & Alberta & 31 & 24.2 & 21.9 & 11.2 & 5.4 & 4.8 & 1.9 & 2.6 & 1.7 & 0.9 & 0.5\\
\hline
BC & Northwest Territories & 54 & 23.7 & 31.7 & 15.8 & 10.3 & 7.8 & 1.6 & 2.4 & 2.7 & 1.5 & 0.3\\
\hline
TSW & Alberta & 32 & 22.2 & 19.2 & 12.7 & 6.3 & 3.8 & 3.0 & 2.4 & 1.8 & 0.8 & 0.7\\
\hline
BP & Saskatchewan & 28 & 21.4 & 13.3 & 5.0 & 6.0 & 2.5 & 4.1 & 1.5 & 1.4 & 0.6 & 0.8\\
\hline
BSW & Ontario & 16 & 20.0 & 10.7 & 6.0 & 5.2 & 5.7 & 0.5 & 1.3 & 1.3 & 1.3 & 0.1\\
\hline
BSW & Manitoba & 22 & 17.2 & 14.8 & 4.7 & 6.1 & 3.7 & 2.3 & 1.3 & 1.6 & 0.9 & 0.5\\
\hline
TP & Northwest Territories & 50 & 13.9 & 13.1 & 6.9 & 8.1 & 2.9 & 2.7 & 1.5 & 2.4 & 0.8 & 0.7\\
\hline
TC & Northwest Territories & 53 & 11.3 & 24.1 & 9.4 & 10.0 & 5.7 & 0.5 & 1.4 & 3.0 & 1.5 & 0.1\\
\hline
BSW & Alberta & 33 & 11.0 & 11.5 & 11.6 & 5.0 & 1.6 & 3.3 & 2.2 & 1.4 & 0.4 & 1.0\\
\hline
BSW & Saskatchewan & 27 & 10.9 & 13.0 & 3.7 & 4.2 & 2.2 & 1.2 & 1.4 & 1.2 & 0.6 & 0.3\\
\hline
TSW & Northwest Territories & 51 & 10.4 & 11.9 & 10.1 & 5.8 & 3.0 & 1.2 & 1.9 & 2.1 & 0.7 & 0.3\\
\hline
\end{tabular}}
\end{table}
\end{landscape}

\section{Appendix B: non-linear least squares modelling of soil organic
layer consumption}

For national annual estimates of forest organic soil layer consumption
during wildfire, implementations that only utilize Canadian experimental
fire data from the Fire Behaviour Prediction System will be limited to a
maximum consumption value of 5 kg (biomass) m\textsuperscript{-2} of
total surface fuel (woody debris, litter, and duff) of 5 kg
m\textsuperscript{-2}, or 25 Mg C ha\textsuperscript{-1}, given the
observation dataset and fitted model parameters. For the common ``C-2''
boreal spruce fuel type for instance, Surface Fuel Consumption, SFC
(biomass units in kg m\textsuperscript{-2} not kg C) is modelled as:

\begin{equation}
SFC = 5.0 \left(1-e^{-0.0115BUI} \right)^{1.0}
\end{equation}                                             

This model form has the distinct advantage of SFC being 0.0 at a BUI of
zero. The model parameters vary by fuel type (i.e.~deciduous broadleaf
fuels are limited to 1.5 kg m\textsuperscript{-2} of maximum SFC) but
are fixed within a fuel type.

More recent observations and modelling from \citet{degroot2009} extended
the FBP data with an additional 128 observations from 7 additional
wildfires, and the ABoVE project compiled over 1,000 field observations
of depth of burn and C stocks before and after wildfire in Canada and
Alaska, over 600 of which are in North American Level II ecoregions also
occurring in Canada \citep{walker2020}. \citet{degroot2009} provides a
concise and informative improvement on the FBP fuel consumption
equations, where both a Fire Weather Index System component (in this
case, Buildup Index) is used similarly to Buildup Index in the FBP, but
importantly, the site-level organic soil layer fuel load is also
accounted for, which allows for the greater absolute combustion in
deeper organic soils that is moderated by the natural logarithm
transformation:

\begin{equation}
log_{e}(FFFC) = -4.252+0.710log_{e}(DC) + 0.671log_{e}(FFFL)
\end{equation}

where FFFC is Forest Floor Fuel Consumption (SFC minus surface woody
debris) in kg (biomass) m\textsuperscript{-2} and FFFL is Forest Floor
Fuel Load in kg m\textsuperscript{-2}. The forest floor as defined here
is inclusive of the litter and duff layers, live mosses and lichens.
This model presented above fits well within the dataset and extends the
observed maximum FFFC to nearly 10 kg m\textsuperscript{-2}. The ABoVE
synthesis of FFFL and FFFC \citep{walker2020} expands upon a slightly
smaller dataset used in a modelling summary also by Walker et al
\citet{walkerFuelAvailabilityNot2020}, where structural equation
modelling was used to explore drivers of FFFC but no concise and readily
reproducible modelling is produced. The results of the SEM from Walker
et al \citet{walkerFuelAvailabilityNot2020} emphasized a greater role of
FFFL over DC, though coarse reanalysis that lacked local fire agency
weather stations was used. An analysis of just 2014 fires in the
Northwest Territories by Walker et al \citet{walkerSoilOrganicLayer2018}
showed that while the mean depth of burn across all black spruce stands
was 6-10 cm, the driest (xeric) black spruce stands with the smallest
FFFL showed upwards of 75\% soil organic consumption, while deeper
organic soils in subhygric black spruce stands showed less than 25\%
consumption.

To provide the largest possible dataset for FFFC and FFFL, the ABoVE
synthesis was combined with wildfire data from de Groot et al 2009 not
otherwise found in the ABoVE synthesis. The ABoVE synthesis sites in the
Alaska Boreal Interior ecoregion, which have equivalent Canadian ecozone
were excluded due to the presence of continuous permafrost and deep
organic soils, but Alaska Boreal Cordillera sites near the Yukon border
were utilized. Experimental fire data from the FBP data was not used, as
deeper combustion measurements resulting from hours and days of
smouldering combustion captured in wildfire data are not available in
experimental fires where extensive smouldering is not measured due to
suppression.

For the purposes of improving national estimates of the fractional soil
organic layer loss during wildfire, this framework emphasizes the
proportional C stock loss (as with all CBM-CFS3 disturbance matrices)
rather than the absolute value of combustion. In contrast to the
modelling of absolute combustion value, any analysis of proportions is
best conducted as logit- transformed data, where the logit
transformation is:

\begin{equation}
logit(p) = log \frac{p}{1-p}
\end{equation}

which effectively transforms a data of proportions of {[}0,1{]} to a
Gaussian distribution with a range of approximately -5 to +5 (in this
dataset), and a mode approximately at zero.

In the logit transformed space, saturation-type non-linear curve using
the relevant FWI component was fitted in a non-linear least squares
model, but an additive term of the natural-logarithm transformed forest
floor fuel load (FFFL) (given as AGSlow pool in Mg C/ha) was used as
well. The non-linear least squares model fit was conducted using the
Levenberg-Marquardt nonlinear least-squares algorithm found in MINPACK
\citep{elzhovMinpacklmInterfaceLevenbergMarquardt2023} R package, which
supported bounded parameter constraints.

In the abstract, the model follows the form:

\begin{equation}
logit \left( \frac{depth of burn}{pre-fire organic depth}\right) = [c (1 - e^{(a  BUI)})] + (b log_{e}(AGSlow))
\end{equation}

with fitted parameters as:

\begin{equation}
logit \left( \frac{depth of burn}{pre-fire organic depth}\right) = [3.50 (1 - e^{(-0.008 BUI)})] + (-0.53 log_{e}(AGSlow))
\end{equation}

Note the ``b'' coefficient on the parameter associated with the FFFL
(AGSlow) of -0.53, which results in larger organic layer fuel loads
leading to smaller proportional consumption values, which follows the
patterns shown by Walker 2018 for NWT fires of 2014.

The a parameter term that forms the exponent of e alongside the Buildup
Index is related to the BUI value at which half of the maximum possible
asymptotal consumption value is observed (for a given FFFL value). The
NLS fitting was given a minimum value of -0.008 such that a 50\%
consumption rate was modelled as occurring at or around a BUI value of
70 for typical upland spruce forest floor (FFFL of 8 kg/m2 or AGSlow of
40 t C/ha) following the wildfire data of de Groot et al 2008 as well as
experimental fire observations in spruce forests from the FBP System.
The other parameters were fit to the best possible value with no
constraint.

For example, using a moderately thick \textasciitilde12 cm thick organic
soil layer, the proportion of consumption as a function of Buildup Index
using the model above:

\begin{figure}
\centering
\includegraphics{GMD-Thompson-et-al_files/figure-latex/example-FFFC-vs-DC-plot-nls-1.pdf}
\caption{Example of the forest floor consumption model for a duff layer
(AGSlow) of 60 Mg C ha-1 (12 kg m-2)}
\end{figure}

With the parameter constrained NLS fitting, the proportional consumption
model for the forest floor has a leave-one-out (conducted at the
fire-level, not plot) cross validated r\textsuperscript{2} of 0.52 and a
Mean Percent Error of 45\%

\begin{figure}
\centering
\includegraphics{GMD-Thompson-et-al_files/figure-latex/show-biplot-FFFC-new-1.pdf}
\caption{Leave one out cross-validation of the forest floor consumption
model. Where multiple measurements were conducted within a single fire,
the entire fire was excluded from the training data used in the model
training.}
\end{figure}

Across the entire parameter space of Buildup Index and AGSlow pool size,
the following isolines of proportional consumption in the model can be
plotted:

\begin{figure}
\centering
\includegraphics{GMD-Thompson-et-al_files/figure-latex/FFFC-surface-plot-1.pdf}
\caption{Isoline contours of equal organic soil layer consumption
fraction as a function of BUI and the organic soil layer (Medium DOM)
Carbon pool in Mg C/ha. Field observations of burn from 547 field
measurements across Canada from the Yukon-Alaska border to Manitoba. Rug
plots at the axis margins show the marginal density of the data.}
\end{figure}

\begin{figure}
\centering
\includegraphics{GMD-Thompson-et-al_files/figure-latex/FFFC-plot-next-1.pdf}
\caption{Isoline contours of equal organic soil layer consumption
fraction as a function of BUI and the organic soil layer (Medium DOM)
Carbon pool in Mg C/ha. Field observations of burn from 547 field
measurements across Canada from the Yukon-Alaska border to Manitoba.}
\end{figure}

\begin{figure}
\centering
\includegraphics{GMD-Thompson-et-al_files/figure-latex/FFFC-plot-next2-1.pdf}
\caption{Isoline contours of equal organic soil layer consumption in
kg/m2 as a function of BUI and the organic soil layer fuel load in
kg/m2. Field observations of burn from 547 field measurements across
Canada from the Yukon-Alaska border to Manitoba. Note that peatland
ecosystems begin at approximately 50 kg/m2 of forest floor fuel load, or
approximately 40 cm organic soil thickness.}
\end{figure}

\begin{figure}
\centering
\includegraphics{GMD-Thompson-et-al_files/figure-latex/FFFC-plot-next4-1.pdf}
\caption{Forest floor consumption as a function of BUI across varying
levels of forest floor fuel load (FFFL), as shown by coloured dots.}
\end{figure}

\section{Appendix C: annual variability in observed Buildup Index during
wildfire spread, and impact on modelled ecozone-level forest floor
emissions}

\begin{figure}
\centering
\includegraphics{GMD-Thompson-et-al_files/figure-latex/AnnualDC-Vis-BUI50-1.pdf}
\caption{Annual variability in area-weighted median BUI of active fire
spread days by ecozone.}
\end{figure}

\begin{figure}
\centering
\includegraphics{GMD-Thompson-et-al_files/figure-latex/AnnualDC-Vis-Consump-1.pdf}
\caption{Annual variability in forest floor median consumption values by
ecozone.}
\end{figure}

\section{Appendix D: Representative photos}

Photos of: (1) partial litter consumption; (2) partial vs full duff
consumption; (3) mortality but not consumption of understory trees with
live overstory; (4) mortality but not consumption of overstory trees;
(5) mixedwood severity example showing consumption of broadleaf foliage;
(6) woody debris consumption; (7) snag preferential consumption relative
to little to no bole consumption in live trees

Give lat/long, year, ecozone, severity class, and leading spp for each
photo, maybe other relevant metrics? From some of the experimental fires
mostly??

\section{Appendix E: List of Resprouting Hardwoods of Canada}

Alnus spp. Arbutus men. Betula all. Betula pap. Betula pop. Fraxinus
ame. Fraxinus nig. Fraxinus pen. Populus bal. Populus gra. Populus tre.
Populus tri. Quercus spp. Salix spp.

\section{Appendix F: Disturbance Matrix Examples}

\begin{landscape}\begin{table}[!h]
\centering
\caption{\label{tab:example-DM-tables}Low severity Disturbance Matrix in BP.  Note: only some biomass and atmospheric pools are shown.}
\centering
\resizebox{\ifdim\width>\linewidth\linewidth\else\width\fi}{!}{
\begin{tabular}[t]{>{\raggedright\arraybackslash}p{3cm}|>{\centering\arraybackslash}p{2cm}|>{\centering\arraybackslash}p{2cm}|>{\centering\arraybackslash}p{2cm}|>{\centering\arraybackslash}p{2cm}|>{\centering\arraybackslash}p{2cm}|>{\centering\arraybackslash}p{2cm}|>{\centering\arraybackslash}p{2cm}|>{\centering\arraybackslash}p{2cm}|>{\centering\arraybackslash}p{2cm}}
\hline
  & Softwood Merchantable & Softwood Stem Snag & Medium DOM
 & Softwood Foliage & Aboveground Very Fast DOM & CO2 & CH4 & CO & PM25\\
\hline
Softwood Merchantable & 1 & 0.000 &  &  &  &  &  &  & \\
\hline
Softwood Stem Snag &  & 0.475 & 0.475 &  &  & 0.043 & 0.000 & 0.004 & 0.001\\
\hline
Medium DOM &  &  & 0.551 &  &  & 0.315 & 0.006 & 0.072 & 0.018\\
\hline
Softwood Foliage &  &  &  & 0.55 & 0.45 & 0.000 & 0.000 & 0.000 & 0.000\\
\hline
Aboveground Very Fast DOM &  &  &  &  & 0.14 & 0.746 & 0.004 & 0.060 & 0.016\\
\hline
CO2 &  &  &  &  &  &  &  &  & \\
\hline
CH4 &  &  &  &  &  &  &  &  & \\
\hline
CO &  &  &  &  &  &  &  &  & \\
\hline
PM25 &  &  &  &  &  &  &  &  & \\
\hline
\end{tabular}}
\end{table}
\end{landscape}

\begin{landscape}\begin{table}[!h]
\centering
\caption{\label{tab:example-DM-tables}Mod severity Disturbance Matrix in BP.  Note: only some biomass and atmospheric pools are shown.}
\centering
\resizebox{\ifdim\width>\linewidth\linewidth\else\width\fi}{!}{
\begin{tabular}[t]{>{\raggedright\arraybackslash}p{3cm}|>{\centering\arraybackslash}p{2cm}|>{\centering\arraybackslash}p{2cm}|>{\centering\arraybackslash}p{2cm}|>{\centering\arraybackslash}p{2cm}|>{\centering\arraybackslash}p{2cm}|>{\centering\arraybackslash}p{2cm}|>{\centering\arraybackslash}p{2cm}|>{\centering\arraybackslash}p{2cm}|>{\centering\arraybackslash}p{2cm}}
\hline
  & Softwood Merchantable & Softwood Stem Snag & Medium DOM
 & Softwood Foliage & Aboveground Very Fast DOM & CO2 & CH4 & CO & PM25\\
\hline
Softwood Merchantable & 0.19 & 0.81 &  &  &  &  &  &  & \\
\hline
Softwood Stem Snag &  & 0.00 & 0.545 &  &  & 0.395 & 0.002 & 0.032 & 0.009\\
\hline
Medium DOM &  &  & 0.462 &  &  & 0.379 & 0.007 & 0.087 & 0.022\\
\hline
Softwood Foliage &  &  &  & 0.19 & 0.00 & 0.703 & 0.004 & 0.057 & 0.015\\
\hline
Aboveground Very Fast DOM &  &  &  &  & 0.06 & 0.816 & 0.005 & 0.066 & 0.018\\
\hline
CO2 &  &  &  &  &  &  &  &  & \\
\hline
CH4 &  &  &  &  &  &  &  &  & \\
\hline
CO &  &  &  &  &  &  &  &  & \\
\hline
PM25 &  &  &  &  &  &  &  &  & \\
\hline
\end{tabular}}
\end{table}
\end{landscape}

\begin{landscape}\begin{table}[!h]
\centering
\caption{\label{tab:example-DM-tables}High severity Disturbance Matrix in BP.  Note: only some biomass and atmospheric pools are shown.}
\centering
\resizebox{\ifdim\width>\linewidth\linewidth\else\width\fi}{!}{
\begin{tabular}[t]{>{\raggedright\arraybackslash}p{3cm}|>{\centering\arraybackslash}p{2cm}|>{\centering\arraybackslash}p{2cm}|>{\centering\arraybackslash}p{2cm}|>{\centering\arraybackslash}p{2cm}|>{\centering\arraybackslash}p{2cm}|>{\centering\arraybackslash}p{2cm}|>{\centering\arraybackslash}p{2cm}|>{\centering\arraybackslash}p{2cm}|>{\centering\arraybackslash}p{2cm}}
\hline
  & Softwood Merchantable & Softwood Stem Snag & Medium DOM
 & Softwood Foliage & Aboveground Very Fast DOM & CO2 & CH4 & CO & PM25\\
\hline
Softwood Merchantable & 0 & 1 &  &  &  &  &  &  & \\
\hline
Softwood Stem Snag &  & 0 & 0.450 &  &  & 0.477 & 0.003 & 0.039 & 0.010\\
\hline
Medium DOM &  &  & 0.576 &  &  & 0.298 & 0.006 & 0.068 & 0.017\\
\hline
Softwood Foliage &  &  &  & 0 & 0.00 & 0.868 & 0.005 & 0.070 & 0.019\\
\hline
Aboveground Very Fast DOM &  &  &  &  & 0.02 & 0.851 & 0.005 & 0.069 & 0.019\\
\hline
CO2 &  &  &  &  &  &  &  &  & \\
\hline
CH4 &  &  &  &  &  &  &  &  & \\
\hline
CO &  &  &  &  &  &  &  &  & \\
\hline
PM25 &  &  &  &  &  &  &  &  & \\
\hline
\end{tabular}}
\end{table}
\end{landscape}

\begin{landscape}\begin{table}[!h]
\centering
\caption{\label{tab:example-DM-tables}High severity Disturbance Matrix in TSE.  Note: only some biomass and atmospheric pools are shown.}
\centering
\resizebox{\ifdim\width>\linewidth\linewidth\else\width\fi}{!}{
\begin{tabular}[t]{>{\raggedright\arraybackslash}p{3cm}|>{\centering\arraybackslash}p{2cm}|>{\centering\arraybackslash}p{2cm}|>{\centering\arraybackslash}p{2cm}|>{\centering\arraybackslash}p{2cm}|>{\centering\arraybackslash}p{2cm}|>{\centering\arraybackslash}p{2cm}|>{\centering\arraybackslash}p{2cm}|>{\centering\arraybackslash}p{2cm}|>{\centering\arraybackslash}p{2cm}}
\hline
  & Softwood Merchantable & Softwood Stem Snag & Medium DOM
 & Softwood Foliage & Aboveground Very Fast DOM & CO2 & CH4 & CO & PM25\\
\hline
Softwood Merchantable & 0 & 1 &  &  &  &  &  &  & \\
\hline
Softwood Stem Snag &  & 0 & 0.450 &  &  & 0.477 & 0.003 & 0.039 & 0.010\\
\hline
Medium DOM &  &  & 0.777 &  &  & 0.157 & 0.003 & 0.036 & 0.009\\
\hline
Softwood Foliage &  &  &  & 0 & 0.00 & 0.868 & 0.005 & 0.070 & 0.019\\
\hline
Aboveground Very Fast DOM &  &  &  &  & 0.05 & 0.825 & 0.005 & 0.066 & 0.018\\
\hline
CO2 &  &  &  &  &  &  &  &  & \\
\hline
CH4 &  &  &  &  &  &  &  &  & \\
\hline
CO &  &  &  &  &  &  &  &  & \\
\hline
PM25 &  &  &  &  &  &  &  &  & \\
\hline
\end{tabular}}
\end{table}
\end{landscape}



\codedataavailability{This article was produced from an RMarkdown
document with underlying data, available at
https://github.com/nrcan-cfs-fire/FireDMs} %% use this section when having data sets and software code available



%%%%%%%%%%%%%%%%%%%%%%%%%%%%%%%%%%%%%%%%%%
%% optional

%%%%%%%%%%%%%%%%%%%%%%%%%%%%%%%%%%%%%%%%%%

%%%%%%%%%%%%%%%%%%%%%%%%%%%%%%%%%%%%%%%%%%

%%%%%%%%%%%%%%%%%%%%%%%%%%%%%%%%%%%%%%%%%%
\competinginterests{The authors declare no competing
interests.} %% this section is mandatory even if you declare that no competing interests are present

%%%%%%%%%%%%%%%%%%%%%%%%%%%%%%%%%%%%%%%%%%
\disclaimer{The algorithm and results presented only apply to boreal and
temperate forest ecosystems where sufficient ground plots of fire
severity are available. As a data-driven model, this framework is not
suitable for other ecosystems nor agricultural or forestry biomass
burning practices.} %% optional section

%%%%%%%%%%%%%%%%%%%%%%%%%%%%%%%%%%%%%%%%%%

%% REFERENCES
%% DN: pre-configured to BibTeX for rticles

%% The reference list is compiled as follows:
%%
%% \begin{thebibliography}{}
%%
%% \bibitem[AUTHOR(YEAR)]{LABEL1}
%% REFERENCE 1
%%
%% \bibitem[AUTHOR(YEAR)]{LABEL2}
%% REFERENCE 2
%%
%% \end{thebibliography}

%% Since the Copernicus LaTeX package includes the BibTeX style file copernicus.bst,
%% authors experienced with BibTeX only have to include the following two lines:
%%
\bibliographystyle{copernicus}
\bibliography{references.bib}
%%
%% URLs and DOIs can be entered in your BibTeX file as:
%%
%% URL = {http://www.xyz.org/~jones/idx_g.htm}
%% DOI = {10.5194/xyz}


%% LITERATURE CITATIONS
%%
%% command                        & example result
%% \citet{jones90}|               & Jones et al. (1990)
%% \citep{jones90}|               & (Jones et al., 1990)
%% \citep{jones90,jones93}|       & (Jones et al., 1990, 1993)
%% \citep[p.~32]{jones90}|        & (Jones et al., 1990, p.~32)
%% \citep[e.g.,][]{jones90}|      & (e.g., Jones et al., 1990)
%% \citep[e.g.,][p.~32]{jones90}| & (e.g., Jones et al., 1990, p.~32)
%% \citeauthor{jones90}|          & Jones et al.
%% \citeyear{jones90}|            & 1990


\end{document}
