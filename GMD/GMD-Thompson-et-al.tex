%% Copernicus Publications Manuscript Preparation Template for LaTeX Submissions
%% ---------------------------------
%% This template should be used for copernicus.cls
%% The class file and some style files are bundled in the Copernicus Latex Package, which can be downloaded from the different journal webpages.
%% For further assistance please contact Copernicus Publications at: production@copernicus.org
%% https://publications.copernicus.org/for_authors/manuscript_preparation.html

%% copernicus_rticles_template (flag for rticles template detection - do not remove!)

%% Please use the following documentclass and journal abbreviations for discussion papers and final revised papers.

%% 2-column papers and discussion papers
\documentclass[, manuscript]{copernicus}



%% Journal abbreviations (please use the same for preprints and final revised papers)

% Advances in Geosciences (adgeo)
% Advances in Radio Science (ars)
% Advances in Science and Research (asr)
% Advances in Statistical Climatology, Meteorology and Oceanography (ascmo)
% Annales Geophysicae (angeo)
% Archives Animal Breeding (aab)
% Atmospheric Chemistry and Physics (acp)
% Atmospheric Measurement Techniques (amt)
% Biogeosciences (bg)
% Climate of the Past (cp)
% DEUQUA Special Publications (deuquasp)
% Drinking Water Engineering and Science (dwes)
% Earth Surface Dynamics (esurf)
% Earth System Dynamics (esd)
% Earth System Science Data (essd)
% E&G Quaternary Science Journal (egqsj)
% EGUsphere (egusphere) | This is only for EGUsphere preprints submitted without relation to an EGU journal.
% European Journal of Mineralogy (ejm)
% Fossil Record (fr)
% Geochronology (gchron)
% Geographica Helvetica (gh)
% Geoscience Communication (gc)
% Geoscientific Instrumentation, Methods and Data Systems (gi)
% Geoscientific Model Development (gmd)
% History of Geo- and Space Sciences (hgss)
% Hydrology and Earth System Sciences (hess)
% Journal of Bone and Joint Infection (jbji)
% Journal of Micropalaeontology (jm)
% Journal of Sensors and Sensor Systems (jsss)
% Magnetic Resonance (mr)
% Mechanical Sciences (ms)
% Natural Hazards and Earth System Sciences (nhess)
% Nonlinear Processes in Geophysics (npg)
% Ocean Science (os)
% Polarforschung - Journal of the German Society for Polar Research (polf)
% Primate Biology (pb)
% Proceedings of the International Association of Hydrological Sciences (piahs)
% Safety of Nuclear Waste Disposal (sand)
% Scientific Drilling (sd)
% SOIL (soil)
% Solid Earth (se)
% The Cryosphere (tc)
% Weather and Climate Dynamics (wcd)
% Web Ecology (we)
% Wind Energy Science (wes)

% Pandoc citation processing

% The "Technical instructions for LaTex" by Copernicus require _not_ to insert any additional packages.
% 
% tightlist command for lists without linebreak
\providecommand{\tightlist}{%
  \setlength{\itemsep}{0pt}\setlength{\parskip}{0pt}}


%%\usepackage{booktabs}
\usepackage{longtable}
\usepackage{array}
\usepackage{multirow}
\usepackage{wrapfig}
\usepackage{float}
\usepackage{colortbl}
\usepackage{pdflscape}
\usepackage{tabu}
\usepackage{threeparttable}
\usepackage{threeparttablex}
\usepackage[normalem]{ulem}
\usepackage{makecell}
\usepackage{xcolor}
%
\begin{document}


\title{Incorporating fire severity for refined carbon emissions
estimates of boreal and temperate forest fires in the Generic Carbon
Budget Model (GCBM)}


\Author[1]{Dan K.}{Thompson}
\Author[2]{Ellen}{Whitman}
\Author[1]{Chelene}{Hanes}
\Author[3]{PFC Carbon Team (exactly who TBD)}{Copernicus}


\affil[1]{Natural Resources Canada, Canadian Forest Service, Great Lakes
Forestry Centre, Sault Ste. Marie, Canada}
\affil[2]{Natural Resources Canada, Canadian Forest Service, Northern
Forestry Centre, Edmonton, Canada}
\affil[3]{Natural Resources Canada, Canadian Forest Service, Pacific
Forestry Centre, Victoria, Canada}

\runningtitle{Fire Severity and Forest Carbon Budget Models}

\runningauthor{Thompson et al.}


\correspondence{Dan K.\ Thompson\ (daniel.thompson@nrcan-rncan.gc.ca)}



\received{}
\pubdiscuss{} %% only important for two-stage journals
\revised{}
\accepted{}
\published{}

%% These dates will be inserted by Copernicus Publications during the typesetting process.


\firstpage{1}

\maketitle


\begin{abstract}
Wildfire is the most impactful natural disturbance to Canada's boreal
and temperate forest biomes. Current representations of fire impact on
forest carbon stocks is limited to a single parameterization of fire
severity (i.e.~the fraction of biomass consumed) that assumes only high
severity fires, despite a large and increasing evidence base of
widespread mixed-severity wildfire. In this submodel of the larger
Generic Carbon Budget Model for forest carbon accounting, field
measurements of biomass consumption as related to satellite-derived burn
severity maps are interpretted from a fire physics and ecology
perspective to derive algorithms to describe forest carbon fluxes in the
immediate aftermath of fires. Compared to the baseline high
severity-only representation, this mixed severity modelling framework
changes modelled total Carbon flux to the atmosphere by {[}\#\#\#{]}\%,
with the largest changes in {[}pool{]}. Per fire energy release rates as
detected by satellite yield a favourable rank correlation with this
severity-only method, and regional estimates of atmospheric CO release
by satellite compare favourably to outputs from this severity-driven
model.
\end{abstract}


\copyrightstatement{His Majesty the King in Right of Canada as
represented by the Minister of Natural Resources Canada. This work is
distributed under the Creative Commons Attribution 4.0 License.}


\section{Introduction}

\begin{itemize}
\item
  general introduction on forest carbon accounting in Canada (1
  paragraph)
\item
  then talk about how fire in many years is the largest disturbance by
  area, and that current estimates broadly assume full severity when
  mapped by NBAC. But, we know that \textless100\% of burned area is
  high severity from Ellen's work and others.
\item
  to support recent advances in operational burn severity mapping for
  Canada, the CBM DMs also need to be upgraded.
\item
  something about dynamic veg models like landis and how they represent
  fire disturbance -then how most MRV models and quickly how they do
  fire disturbance -even a bit on the GFED/GFAS world too
\end{itemize}

that ESSD frames it well, worth reading the intro part in detail)
(recent Smyth and Campbell papers too) (other recent Canadian
fire-carbon things, Walker - esp is supports consistent patterns in
proportional carbon loss)

\begin{itemize}
\item
  In this document, we outline the evidence-based fire DMs proposed.
\item
  From a blend of aggregated field data linked to remotely sensed
  severity, as well as insights from fire physics and experimental
  fires.
\item
  Key knowledge gaps are also highlighted, with interim solutions
  presented until further quantification can be done in field studies
  (could be wildfire, experimental fires, or prescribed fires).
\end{itemize}

\section{Methods}

\subsection{Biomass pools of the Generic Carbon Budget Model}

Short section explaining the pool definitions most relevant to fire.

\subsection{Axioms of forest carbon budget after fire}

To simplify the process of the creation of the DMs as a distillation of
the complexities of fire severity and combustion patterns, the following
logical axioms are proposed and maintained throughout:

\begin{enumerate}
\def\labelenumi{\arabic{enumi}.}
\item
  Disturbance matrices are to be in terms of mortality, not survival
\item
  Crown Fraction Burned (CFB) is a mass-based estimate of the portion of
  foliage consumed in flaming, and is inclusive of merchantable and
  submerchantable trees, both broadleaf and needleleaf
\item
  Snags are inclusive of both those killed by prior fire as well as
  those killed by all other causes
\item
  In submerchantable trees, mortality = CFB
\item
  In submerchantable trees, mortality is \textless= 1
\item
  In merchantable stands, CFB \textless{} mortality
\item
  Survival = 1 - mortality
\item
  CFB \textless{} survival
\item
  The girdled fraction of trees = mortality - CFB
\item
  Survival \textless= 1 and also \textgreater= 0
\end{enumerate}

Of these, Crown Fraction Burned (CFB) is both highly critical and a
concept used primarily in fire behaviour science but not carbon
accounting nor fire ecology. CFB was introduced in the 1992 Fire
Behaviour Prediction System documentation, and provides a simple
continuous 0-100 variable for only the consumption of foliage (inclusive
of both conifer and broadleaf), as opposed to ordinal and less precise
systems like Crown Fire Severity Index that allows the user to specify
moreso which pools of canopy biomass are consumed, but not the degree to
which a given pool is consumed.

\subsection{Ground plot and remotely sensed fire severity data}

!!!Ellen to insert methods here - including the figure of where the
samples are from etc.

\subsection{Combustion gas emission ratios}

Certain variables, like the fractionation of CO2:CH4:CO, are constant
throughout ecozones, but vary by flaming vs smouldering. They are
defined in a global variables table:

\begin{table}

\caption{\label{tab:globalVarsEmissions}Emissions ratios in flaming and smouldering phase, updated to reflect values used in Canada's operational wildfire smoke emissions model, CFFEPS-Firework}
\centering
\begin{tabular}[t]{r|r|r|r|r|r}
\hline
FlamingCO2 & FlamingCH4 & FlamingCO & SmoulderingCO2 & SmoulderingCH4 & SmoulderingCO\\
\hline
0.9 & 0.01 & 0.09 & 0.9 & 0.01 & 0.09\\
\hline
\end{tabular}
\end{table}

where CO\textsubscript{2} is responsible for 90\% of emissions in the
flaming phase, but only 90\% of emissions in the smouldering phase, with
a doubling of CO emissions and tripling of CH\textsubscript{4}
emissions. With a Global Warming Potential of CO equal to 1.9 and
CH\textsubscript{4} of 25, the Global Warming Potential per unit of
biomass consumption in the smouldering phase is 1 times higher in global
warming potential compared to flaming, not including differential
aerosol production and injection heights, however. Note that these
proposed emissions factors for flaming vs smouldering are aligned with
those currently used in Canada's operational wildfire smoke air quality
model, FireWork \citep{chen2019}. With flaming and smouldering each
contributing roughly equally to wildfire emissions, these distinct
flaming and smouldering emissions rates correspond well with aircraft
smoke chemistry observations by \citep{simpson2011} and
\citep{hayden2022} and are themselves very similar to prior emissions
factors used in CBM. Note that as current described, the sum of
CO\textsubscript{2}, CH\textsubscript{4}, and CO emissions from
wildfires only represent approximately 95\% of the fire carbon mass
emitted to the atmosphere, with 0.5-2.0\% of biomass emitted as
particulate matter (e.g.~PM2.5, but also PM1 and PM10 classes of
particulates at 1 and 10 um diameters, respectively), and an additional
5\% \citep{hayden2022} to as little as 1\% \citep{simon2010} composed of
non-methane organic gases that have a large range in global warming
potentials as compared to CH\textsubscript{4}.

\subsection{Litter layer area-wise consumption by severity class}

The litter layer forms the first biomass pool in which a spreading fire
consumes fuel. In low-severity fires, the litter layer may be consumed
little to no underlying duff material consumed, nor any tree mortality
\citep{hessburg2019}. Logically, since litter consumption is required
for the ignition of the underlying duff layer, this litter area-wise
fractional consumption also informs and constrains duff consumption.

\begin{table}

\caption{\label{tab:UnBurnedForestFloorByEcozone}Unburned litter area by ecozone and severity class.  The majority of the data comes from studies in the Boreal Plains and Boreal Shield West, and so values are extrapolated from those two well-observed ecozones to all others.}
\centering
\begin{tabular}[t]{l|r|r|r}
\hline
Ecozone & Low & Mod & High\\
\hline
AM & 0.14 & 0.06 & 0.02\\
\hline
BC & 0.14 & 0.06 & 0.02\\
\hline
BP & 0.14 & 0.06 & 0.02\\
\hline
BSE & 0.20 & 0.08 & 0.05\\
\hline
BSW & 0.20 & 0.08 & 0.05\\
\hline
HP & 0.20 & 0.08 & 0.05\\
\hline
MC & 0.14 & 0.06 & 0.02\\
\hline
MP & 0.14 & 0.06 & 0.02\\
\hline
P & 0.14 & 0.06 & 0.02\\
\hline
PM & 0.14 & 0.06 & 0.02\\
\hline
TC & 0.14 & 0.06 & 0.02\\
\hline
TP & 0.14 & 0.16 & 0.03\\
\hline
TSE & 0.20 & 0.08 & 0.05\\
\hline
TSW & 0.20 & 0.08 & 0.05\\
\hline
\end{tabular}
\end{table}

\subsection{Duff Consumption}

While consumption of fine fuels in the litter layer of the forest floor
is nearly complete for any given fire intensity, consumption of deeper
organic soil horizons (F+H layers in upland forests and upper peat
layers in wetlands) is more drought dependent. In this scheme, we
utilize the Forest Floor Fuel Consumption (FFFC) model of
\citep{degroot2009}, modified to only account for fuel horizons below
the litter layer:

\begin{equation}
FFFC = 0.016872 DC^{0.71}(FFFL-LL)^{0.671}-LL
\end{equation}

where DC is the Fire Weather Index Drought Code and FFFL is the Forest
Floor Fuel Load (with ecozone averages given in \citep{letang2012} or
site-level data). LL is the Litter Load, and is typically on the order
of \unit{0.2\,kg\,m^{-2}} for most boreal forest upland and peatland
sites \citep{thompson2017}. This distinction is necessary due to the
flaming phase consumption of the litter layer as opposed to the
smouldering phase consumption of deeper horizons (see previous section).
While ultimately this scheme can be used on individual fires with
estimated or measured fuel loading and specific Drought Code values, for
the purposes of this first assessment, an ecozone-averaged fuel load and
decadal composites of Drought Code is used to provide representative
values. Specifically, a median Drought Code of detected fire hotspots in
Canada from 2003-2021 using the same data as the Canadian
CFEEPS-FireWork wildfire air quality model of \citep{chen2019} is
presented below, along with proportional consumption values of the
forest floor by ecozone:

\begin{table}

\caption{\label{tab:DuffConsumpTable}Fire Weather, fuel loading, and duff consumption values per ecozone}
\centering
\begin{tabular}[t]{l|l|l|l|l}
\hline
Ecozone & Median.DC.of.burning & Median.Duff.Load.kg.m2 & Duff.consump.kg.m2 & Duff.consump.frac\\
\hline
AM & 270 & 10.65 & 4.14 & 0.4\\
\hline
TP & 369 & 14.75 & 6.57 & 0.45\\
\hline
TSW & 297 & 1.45 & 0.98 & 0.82\\
\hline
BSW & 239 & 8.55 & 3.23 & 0.39\\
\hline
BP & 242 & 9.55 & 3.53 & 0.38\\
\hline
P & 242 & 9.55 & 3.53 & 0.38\\
\hline
TC & 254 & 8.06 & 3.24 & 0.41\\
\hline
BC & 250 & 8.06 & 3.2 & 0.41\\
\hline
PM & 268 & 14.95 & 5.24 & 0.36\\
\hline
MC & 452 & 5.75 & 3.94 & 0.72\\
\hline
HP & 204 & 7.65 & 2.63 & 0.36\\
\hline
TSE & 98 & 1.45 & 0.31 & 0.26\\
\hline
BSE & 123 & 10.65 & 2.26 & 0.22\\
\hline
\end{tabular}
\end{table}

Little data is available on the fraction of woody debris consumption
alongside fire severity measurements. Coarse woody debris of overstory
stems that makes up 60-80\% of woody debris biomass in Canada's boreal
and temperate forests \citep{hanes2021}, with its moisture and
consumption patterns largely follows the moisture regime of the Drought
Code \citep{mcalpine1995}. As a result, in this modelling framework, the
proportion of woody debris consumption (!!give diameter range of Med
DOM) is assumed proportional to that of the duff consumption given in
the equation above.

\subsection{Drivers of C losses in the tree canopy}

\subsubsection{Overstory tree mortality and consumption}

Numerous process-driven \citep{michaletz2006} or empirical
\citep{hood2017} tree mortality models are present and show significant
skill in predicting tree mortality based on fire behaviour (i.e.~flame
length, rate of spread). Since the driving data in this model is
satellite-derived fire severity over the landscape scale, fire behaviour
metrics such as flame length or scorching height of bark are not
available as a continuous mapped product. Instead, softwood and hardwood
overstory mortality is calculated per ecozone as a function of
satellite-observed fire severity using aggregated ground plot data:

\begin{table}

\caption{\label{tab:MortByEcozone}Softwood fractional mortality by ecozone, as dervied from median values from field studies}
\centering
\begin{tabular}[t]{l|r|r|r}
\hline
Ecozone & Low & Mod & High\\
\hline
AM & 0.28 & 0.34 & 0.95\\
\hline
BC & 0.24 & 0.65 & 0.98\\
\hline
BP & 0.45 & 0.81 & 1.00\\
\hline
BSE & 0.45 & 0.81 & 1.00\\
\hline
BSW & 0.45 & 0.81 & 1.00\\
\hline
HP & 0.45 & 0.81 & 1.00\\
\hline
MC & 0.28 & 0.74 & 0.98\\
\hline
MP & 0.28 & 0.34 & 0.95\\
\hline
P & 0.45 & 0.81 & 1.00\\
\hline
PM & 0.13 & 0.38 & 0.97\\
\hline
TC & 0.24 & 0.65 & 0.98\\
\hline
TP & 0.45 & 0.81 & 1.00\\
\hline
TSE & 0.10 & 0.81 & 1.00\\
\hline
TSW & 0.10 & 0.81 & 1.00\\
\hline
\end{tabular}
\end{table}

And since large-diameter, live trees killed by fire do not experience
significant live stemwood consumption, the entirity of the live stemwood
biomass pool that is killed is transferred to the snag pool. Note that
the field data and disturbance modelling undertaken here only accounts
for tree mortality within the calender year of the fire, and delayed
mortality of over one year has been documented in boreal low and
moderate severity fires \citep{angers2011}where less than half of total
mortality occurs after the year of the fire. The modelling here does not
account for delayed mortality, but rather than routine is a function of
the larger mortality schemes in GCBM.

Crown Fraction Burned (CFB) speaks to the fraction of the live canopy
that is itself consumed in the flaming front. The alternate outcomes
being survival of the foliage, or the mortality of the tree without
canopy consumption, resulting in the dropping of foliage onto the forest
floor. From the axioms stated earlier, the CFB must be lower than or
equal to the mortality rate, using field studies that show any partial
crown consumption is likely sufficient to result in high rates if not
complete mortality \citep{hood2017}, which is the case in Canada's trees
with primarily thin bark. From field studies (!Ellen to provide details
here\ldots.) , the following ecozone-specific CFB values are found:

\begin{table}

\caption{\label{tab:CFBByEcozone}Softwood crown fraction burned by ecozone, as dervied from median values from field studies}
\centering
\begin{tabular}[t]{l|r|r|r}
\hline
Ecozone & Low & Mod & High\\
\hline
AM & 0.0 & 0.34 & 0.95\\
\hline
BC & 0.0 & 0.65 & 0.98\\
\hline
BP & 0.0 & 0.81 & 1.00\\
\hline
BSE & 0.0 & 0.81 & 1.00\\
\hline
BSW & 0.0 & 0.81 & 1.00\\
\hline
HP & 0.0 & 0.81 & 1.00\\
\hline
MC & 0.0 & 0.74 & 1.00\\
\hline
MP & 0.0 & 0.34 & 0.95\\
\hline
P & 0.0 & 0.81 & 1.00\\
\hline
PM & 0.0 & 0.38 & 0.97\\
\hline
TC & 0.0 & 0.65 & 1.00\\
\hline
TP & 0.0 & 0.81 & 1.00\\
\hline
TSE & 0.1 & 0.81 & 1.00\\
\hline
TSW & 0.1 & 0.81 & 1.00\\
\hline
\end{tabular}
\end{table}

The consumption of live bark biomass is a pool in the model, and
consumption rates can be defined by severity class. At the moment,
lacking robust field data on bark biomass consumption rates across
ecozones and severity classes (which are a small portion of the overall
biomass), the bark proportional consumption rate is set to 34\% of the
overstory mortality rate, based only on a single well-observed high
severity fire in Taiga Plains by Santin 2015.

A major distinction is made between softwood and hardwood trees, where
in Canada's boreal forests, a large fraction of hardwood trees
(specifically of the genii Populous and Betula) are able to resprout
even when the main stem has been killed by an intense forest fire
\citep{brown1987}. Accordingly, the root mortality rates differ greatly
between softwoods and hardwoods, with softwood root mortality largely
equal to stem mortality, while in resprouting hardwoods, little root
mortality is observed even after intense fire
\citep{pérez-izquierdo2019}. Concurrently, the fraction of fine roots
contained within the combustible forest floor layers can be a close to
or exceeding 50\% of the fine root biomass \citep{strong1985}, and burns
alongside the organic soils \citep{benscoter2011}. As a result, the
cgculation for softwood fine root consumption and mortality are as
follows, using Softwood as an example:

\begin{equation}
SWFineRootConsump = SW.Mort \times SW.Prop.Fine.Root.duff \times Duff.Consump.Fract
\end{equation}

\begin{equation}
SWFineRootMort.AG = SW.Mort \times SW.Prop.Fine.Root.duff \times (1-Duff.Consump.Fract) \times (1-ReSproutFactor)
SWFineRootMort.BG = SW.Mort \times (1-SW.Prop.Fine.Root.duff)
\end{equation}

In contrast, the larger diameter of the coarse root biomass pool
prevents its consumption during any smouldering of the duff layer, and
the mortality rate of coarse roots is simply proportional to that of the
stemwood overall.

\subsubsection{Understory tree mortality and consumption}

Understory (or small diameter overstory) tree mortality is defined
seperately in the model, but given the lack of data on diameter classes
in the severity data, robust field data on differing mortality rates of
smaller diameter trees is not available, and so the understory tree
mortality rate is set equal to the overstory rate as defined in the
table above. Note that trees with a top height less than 1.4 m are not
considered in this pool, and instead are lumped into the ``other'' pool.

\subsubsection{Snag and stump consumption}

Compared to live stemwood of the same diameter, the low moisture content
of standing dead stemwood (snags) allows for much greater consumption
during the passage of an intense flaming front. The snag branch pool
experiences almost complete combustion, while the largest biomass pool
of the main standing dead stemwood

\subsection{Construction fire disturbance matrices}

Give total number of global parameters, and parameters per ecozone, and
then total parameters, and what \% of total parameters we have so far
filled with data

\section{Results and example applications}

\subsection{Field case studies}

ICFME has detailed, for high severity.

Another option would be Lafoe Creek mixedwood burns, which would need
some sort of CFB and mortality estimate. Is the pre-fire fuel load
documented somewhere?

Pelican would work too, less on the bark consumption etc though.

And last, maybe something from Carrot lake or similarly well-observed BC
fire?

\subsection{2021 Fires in British Columbia, Canada}

Give quick summary here, show overview map of all the fires, and also an
example of the GCBM biomass pools and also severity classes on a fire or
two from Ellen

Then compare old vs new fire DM scheme for individual fires as a biplot

Then show sum of FRP per fire (GOES) as compared to our severity-based
scheme here

Can also look for area-weighted differences in intensity between fires,
using both sum FRP and FireDM approach

MOPITT?

\section{Discussion}

\section{Conclusions}

The conclusion goes here.

\section{Appendix A: list of fluxes and corresponding fire-related
plain-language summary.}

\begin{table}[!h]
\centering
\resizebox{\linewidth}{!}{
\begin{tabular}[t]{l|l|l}
\hline
Source & Sink & ProcessSynonym\\
\hline
Softwood Merchantable & Softwood Merchantable & Survival rate of large conifers\\
\hline
Softwood Merchantable & Softwood Stem Snag & Mortality rate of large conifers\\
\hline
Softwood Merchantable & Black Carbon & Live conifer to BC rate\\
\hline
Softwood Foliage & Softwood Foliage & Green fraction of canopy remaining intact after fire\\
\hline
Softwood Foliage & Aboveground Very Fast DOM & Post-fire litterfall (heat-killed but not burned)\\
\hline
Softwood Foliage & CO2 & Crown Fraction Burned\\
\hline
Softwood Foliage & CH4 & Crown Fraction Burned\\
\hline
Softwood Foliage & CO & Crown Fraction Burned\\
\hline
Softwood Other & Softwood Other & unconsumed Live branches, stumps and small trees including bark\\
\hline
Softwood Other & Softwood Branch Snag & Portion of "other" pool as killed by fire but unconsumed branches\\
\hline
Softwood Other & CO2 & Proportional Combustion sum of branches, stumps, small trees and bark\\
\hline
Softwood Other & CH4 & Proportional Combustion sum of branches, stumps, small trees and bark\\
\hline
Softwood Other & CO & Proportional Combustion sum of branches, stumps, small trees and bark\\
\hline
Softwood Submerchantable & Softwood Submerchantable & Understory conifer survival rate\\
\hline
Softwood Submerchantable & Softwood Branch Snag & Understory conifer branches killed but not consumed\\
\hline
Softwood Submerchantable & CO2 & Understory Conifer consumption rate\\
\hline
Softwood Submerchantable & CH4 & Understory Conifer consumption rate\\
\hline
Softwood Submerchantable & CO & Understory Conifer consumption rate\\
\hline
Softwood Coarse Roots & Softwood Coarse Roots & Surviving coarse roots in conifers\\
\hline
Softwood Coarse Roots & Aboveground Fast DOM & Coarse Roots killed in fire but not combusted in organic soil\\
\hline
Softwood Coarse Roots & Belowground Fast DOM & Coarse Roots killed in fire but not combusted in mineral soil\\
\hline
Softwood Fine Roots & Softwood Fine Roots & Surviving fine roots\\
\hline
Softwood Fine Roots & Aboveground Very Fast DOM & Fine roots killed but not burned in organic soil\\
\hline
Softwood Fine Roots & Belowground Very Fast DOM & Fine roots killed but not burned in mineral soil\\
\hline
Softwood Fine Roots & CO2 & Fine roots combusted alongside duff\\
\hline
Softwood Fine Roots & CH4 & Fine roots combusted alongside duff\\
\hline
Softwood Fine Roots & CO & Fine roots combusted alongside duff\\
\hline
Hardwood Merchantable & Hardwood Merchantable & Survival rate of broadleaf trees\\
\hline
Hardwood Merchantable & Hardwood Stem Snag & Mortality rate of broadleaves\\
\hline
Hardwood Merchantable & Black Carbon & Live broadleaf stemwood to black carbon (incomplete combustion) rate\\
\hline
Hardwood Foliage & Hardwood Foliage & Green fraction of canopy\\
\hline
Hardwood Foliage & Aboveground Very Fast DOM & Post-fire litterfall\\
\hline
Hardwood Foliage & CO2 & Crown Fraction Burned\\
\hline
Hardwood Foliage & CH4 & Crown Fraction Burned\\
\hline
Hardwood Foliage & CO & Crown Fraction Burned\\
\hline
Hardwood Other & Hardwood Other & Surviving Live branches, stumps and small trees including bark\\
\hline
Hardwood Other & Hardwood Branch Snag & Portion of "other" pool as dead but unburned large branches\\
\hline
Hardwood Other & CO2 & Proportional Combustion sum of branches, stumps, small trees and bark\\
\hline
Hardwood Other & CH4 & Proportional Combustion sum of branches, stumps, small trees and bark\\
\hline
Hardwood Other & CO & Proportional Combustion sum of branches, stumps, small trees and bark\\
\hline
Hardwood Submerchantable & Hardwood Submerchantable & Understory broadleaf survival rate\\
\hline
Hardwood Submerchantable & Hardwood Branch Snag & Understory broadleaf mortality rate\\
\hline
Hardwood Submerchantable & CO2 & Understory Broadleaf consumption rate\\
\hline
Hardwood Submerchantable & CH4 & Understory Broadleaf consumption rate\\
\hline
Hardwood Submerchantable & CO & Understory Broadleaf consumption rate\\
\hline
Hardwood Coarse roots & Hardwood Coarse roots & Surviving deciduous coarse roots\\
\hline
Hardwood Coarse roots & Aboveground Fast DOM & Deciduous coarse roots in the duff that are killed but unconsumed\\
\hline
Hardwood Coarse roots & Belowground Fast DOM & Deciduous coarse roots in the mineral soil that are killed but unconsumed\\
\hline
Hardwood Fine Roots & Hardwood Fine Roots & Surviving fine roots\\
\hline
Hardwood Fine Roots & Aboveground Very Fast DOM & Fine roots killed but not burned in organic soil\\
\hline
Hardwood Fine Roots & Belowground Very Fast DOM & Fine roots killed but not burned in mineral soil\\
\hline
Hardwood Fine Roots & CO2 & Fine roots combusted alongside duff\\
\hline
Hardwood Fine Roots & CH4 & Fine roots combusted alongside duff\\
\hline
Hardwood Fine Roots & CO & Fine roots combusted alongside duff\\
\hline
Aboveground Very Fast DOM & Aboveground Very Fast DOM & Unburned fraction of The L horizonb comprised of foliar litter plus dead fine roots, approximately <5 mm diameter\\
\hline
Aboveground Very Fast DOM & CO2 & Combusted fraction of litter layer\\
\hline
Aboveground Very Fast DOM & CH4 & Combusted fraction of litter layer\\
\hline
Aboveground Very Fast DOM & CO & Combusted fraction of litter layer\\
\hline
Belowground Very Fast DOM & Belowground Very Fast DOM & Uncombusted Fraction of Dead fine roots in the mineral soil, approximately <5 mm diameter\\
\hline
Aboveground Fast DOM & Aboveground Fast DOM & Uncombusted Fine and small woody debris plus dead coarse roots in the forest floor, approximately <75 mm diameter\\
\hline
Aboveground Fast DOM & CO2 & Combusted FWD+MWD fraction (dead course roots accounted for in AG Slow DOM)\\
\hline
Aboveground Fast DOM & CH4 & Combusted FWD+MWD fraction (dead course roots accounted for in AG Slow DOM)\\
\hline
Aboveground Fast DOM & CO & Combusted FWD+MWD fraction (dead course roots accounted for in AG Slow DOM)\\
\hline
Belowground Fast DOM & Belowground Fast DOM & Uncombusted fraction of Dead coarse roots in the mineral soil, approximately <f2>5 diameter\\
\hline
Medium DOM & Medium DOM & Uncombusted Coarse woody debris on the ground\\
\hline
Medium DOM & CO2 & Combusted fraction of CWD\\
\hline
Medium DOM & CH4 & Combusted fraction of CWD\\
\hline
Medium DOM & CO & Combusted fraction of CWD\\
\hline
Medium DOM & Black Carbon & Medium DOM converted to Black Carbon\\
\hline
Aboveground Slow DOM & Aboveground Slow DOM & Uncombusted fraction of duff and peat horizons\\
\hline
Aboveground Slow DOM & CO2 & Combusted Fraction of duff and peat horizons\\
\hline
Aboveground Slow DOM & CH4 & Combusted Fraction of duff and peat horizons\\
\hline
Aboveground Slow DOM & CO & Combusted Fraction of duff and peat horizons\\
\hline
Belowground Slow DOM & Belowground Slow DOM & Uncombusted fraction of Humified organic matter in the mineral soil\\
\hline
Softwood Stem Snag & Medium DOM & Fraction of conifer snags that fall to ground but not combusted\\
\hline
Softwood Stem Snag & Softwood Stem Snag & Uncombusted fraction of conifer snags (still upright)\\
\hline
Softwood Stem Snag & CO2 & Combusted fraction of conifer snags\\
\hline
Softwood Stem Snag & CH4 & Combusted fraction of conifer snags\\
\hline
Softwood Stem Snag & CO & Combusted fraction of conifer snags\\
\hline
Softwood Stem Snag & Black Carbon & Portion of conifer snags that are converted to Black Carbon upon burning\\
\hline
Softwood Branch Snag & Aboveground Fast DOM & Portion of conifer snag branches that falls onto the ground?\\
\hline
Softwood Branch Snag & Softwood Branch Snag & Unaltered fraction\\
\hline
Softwood Branch Snag & CO2 & Portion of conifer branch snags that are combusted\\
\hline
Softwood Branch Snag & CH4 & Portion of conifer branch snags that are combusted\\
\hline
Softwood Branch Snag & CO & Portion of conifer branch snags that are combusted\\
\hline
Hardwood Stem Snag & Medium DOM & Portion of deciduous snags that falls onto the ground?\\
\hline
Hardwood Stem Snag & Hardwood Stem Snag & Unaltered stem snag fraction\\
\hline
Hardwood Stem Snag & CO2 & Combusted fraction of deciduous snags\\
\hline
Hardwood Stem Snag & CH4 & Combusted fraction of deciduous snags\\
\hline
Hardwood Stem Snag & CO & Combusted fraction of deciduous snags\\
\hline
Hardwood Stem Snag & Black Carbon & Portion of deciduous snags that are converted to Black Carbon upon burning\\
\hline
Hardwood Branch Snag & Aboveground Fast DOM & Portion of deciduous snag branches that falls onto the ground?\\
\hline
Hardwood Branch Snag & Hardwood Branch Snag & Portion of deciduous snag branches that survive a fire\\
\hline
Hardwood Branch Snag & CO2 & Combusted fraction of deciduous snag branches\\
\hline
Hardwood Branch Snag & CH4 & Combusted fraction of deciduous snag branches\\
\hline
Hardwood Branch Snag & CO & Combusted fraction of deciduous snag branches\\
\hline
Black Carbon & Black Carbon & Unburned black carbon fraction\\
\hline
Black Carbon & CO2 & Black carbon combustion fraction\\
\hline
Black Carbon & CH4 & Black carbon combustion fraction\\
\hline
Black Carbon & CO & Black carbon combustion fraction\\
\hline
\end{tabular}}
\end{table}

\section{Appendix B: annual variability in observed Drought Code during
wildfire spread, and impact on ecozone-level DM calculations}

\includegraphics{GMD-Thompson-et-al_files/figure-latex/AnnualDC-Vis-DC50-1.pdf}

\includegraphics{GMD-Thompson-et-al_files/figure-latex/AnnualDC-Vis-Consump-1.pdf}

\includegraphics{GMD-Thompson-et-al_files/figure-latex/AnnualDC-Vis-frac-1.pdf}

\section{Appendix C: DM template (can delete in final draft)}

First, a generic template for a fire DM is loaded, that can represent
any ecozone. It comes in two parts: (1) a list of variables, some
biophysical and not relating to fire severity (such as the portion of
live branchwood that falls into the smaller size fraction); or (2)
severity-specific variables (such as Crown Fraction Burn) for a severity
class. The template is loaded, and replicated across the list of
ecozones (or any spatial unit) desired. Other processes, such as the
analysis of field data, can then be used to fill in ecozone-specific
variables in severity classes.

An example of the variable definition template is as follows:

\begin{table}[!h]

\caption{\label{tab:varDefsExample}Example of stored fire disturbance matrix precursor variable information}
\centering
\resizebox{\linewidth}{!}{
\begin{tabular}[t]{l|l|l|l|r|l|l|l}
\hline
Ecozone & Pool & Plain.Language.Name & Variable.Name & Value & SeverityClass & InterimValue & Notes\\
\hline
AM & Branch & Softwood small branch fraction of total branchwood & SW.SmBranch.frac.of.tot.BW & 0.5 &  & TRUE & \\
\hline
AM & Branch & Hardwood small branch fraction of total branchwood & HW.SmBranch.frac.of.tot.BW & 0.5 &  & TRUE & \\
\hline
AM & Other & Softwood branchwood as portion of total "other" pool & SW.BW.frac.of.other & 0.4 &  & TRUE & \\
\hline
AM & Other & Hardwood branchwood as portion of total "other" pool & HW.BW.frac.of.other & 0.4 &  & TRUE & \\
\hline
AM & Other & Hardwood Bark as portion of "other" pool & HW.bark.frac.of.other & 0.1 &  & TRUE & \\
\hline
AM & Other & Softwood Bark as portion of "other" pool & SW.bark.frac.of.other & 0.1 &  & TRUE & \\
\hline
\end{tabular}}
\end{table}

A plain language name for each variable is provided right in the data,
as well.

A second template defines each flux in a Disturbance Matrix, with Source
and Sink defined as precise character variables, and a plain language
summary (``Process Synonym'') included to tie this flux back to language
used in the fire science literature. Pseudocode and notes are included
in each flux, which is repeated for each fire severity class and
ecozone. There are 3900 total fluxes, though many do not have sufficient
information to describe differences between ecozones. Many of these are
computed automatically, tying back into variables such as Crown Fraction
Burned.

\begin{table}[!h]

\caption{\label{tab:SinkSourceExample}Sample of disturbance matrix data file}
\centering
\resizebox{\linewidth}{!}{
\begin{tabular}[t]{l|r|l|l|l|l|l|l|r|l}
\hline
Ecozone & FluxID & Source & Sink & ProcessSynonym & Pseudocode & Notes & SeverityClass & Value & InterimValue\\
\hline
AM & 1 & Softwood Merchantable & Softwood Merchantable & Survival rate of large conifers & 1-mortality rate & Ellen provides & Low & 1.0 & TRUE\\
\hline
TP & 1 & Softwood Merchantable & Softwood Merchantable & Survival rate of large conifers & 1-mortality rate & Ellen provides & Low & 1.0 & TRUE\\
\hline
TSW & 1 & Softwood Merchantable & Softwood Merchantable & Survival rate of large conifers & 1-mortality rate & Ellen provides & Low & 0.9 & TRUE\\
\hline
BSW & 1 & Softwood Merchantable & Softwood Merchantable & Survival rate of large conifers & 1-mortality rate & Ellen provides & Low & 1.0 & TRUE\\
\hline
BP & 1 & Softwood Merchantable & Softwood Merchantable & Survival rate of large conifers & 1-mortality rate & Ellen provides & Low & 1.0 & TRUE\\
\hline
P & 1 & Softwood Merchantable & Softwood Merchantable & Survival rate of large conifers & 1-mortality rate & Ellen provides & Low & 1.0 & TRUE\\
\hline
\end{tabular}}
\end{table}

With both generic ecozone variables as well as severity-specific
variables defined and the pseudocode for each flux included, actual DM
values are computed as references to tables, subset by ecozone and
severity class.

Note that rather than defining softwood crown fraction burned as a
variable called ``SW.CFB.Boreal.Plains'' for each ecozone, there is a
row in the VarDefs table that represents SW.CFB in each ecozone and for
each severity class, thus avoiding the creation of large lists of
manually entered variable names and values in the R environment.
Instead, these values can be programmatically entered via external
analysis of plot data (not covered here).

\section{Appendix D: Representative photos}

Photos of: (1) partial litter consumption; (2) partial vs full duff
consumption; (3) mortality but not consumption of understory trees with
live overstory; (4) mortality but not consumption of overstory trees;
(5) mixedwood severity example showing consumption of broadleaf foliage;
(6) woody debris consumption; (7) snag preferential consumption relative
to little to no bole consumpotion in live trees



\codedataavailability{This article was produced from an RMarkdown
document with underlying data, available at
https://github.com/nrcan-cfs-fire/FireDMs} %% use this section when having data sets and software code available



%%%%%%%%%%%%%%%%%%%%%%%%%%%%%%%%%%%%%%%%%%
%% optional

%%%%%%%%%%%%%%%%%%%%%%%%%%%%%%%%%%%%%%%%%%
\appendix
\section{List of fluxes and corresponding fire-related plain-language summary}

Regarding figures and tables in appendices, the following two options
are possible depending on your general handling of figures and tables in
the manuscript environment:

\subsection{Option 1}

If you sorted all figures and tables into the sections of the text,
please also sort the appendix figures and appendix tables into the
respective appendix sections. They will be correctly named
automatically.

\subsection{Option 2}

If you put all figures after the reference list, please insert appendix
tables and figures after the normal tables and figures.

To rename them correctly to A1, A2, etc., please add the following
commands in front of them: \texttt{\textbackslash{}appendixfigures}
needs to be added in front of appendix figures
\texttt{\textbackslash{}appendixtables} needs to be added in front of
appendix tables

Please add \texttt{\textbackslash{}clearpage} between each table and/or
figure. Further guidelines on figures and tables can be found below.
\noappendix

%%%%%%%%%%%%%%%%%%%%%%%%%%%%%%%%%%%%%%%%%%
\authorcontribution{Thompson and Whitman contributed to the concept and
code design with the assistance of Hanes.} %% optional section

%%%%%%%%%%%%%%%%%%%%%%%%%%%%%%%%%%%%%%%%%%
\competinginterests{The authors declare no competing
interests.} %% this section is mandatory even if you declare that no competing interests are present

%%%%%%%%%%%%%%%%%%%%%%%%%%%%%%%%%%%%%%%%%%
\disclaimer{The algorithm and results presented only apply to boreal and
temperate forest ecosystems where sufficient ground plots of fire
severity are available. As a data-driven model, this framework is not
suitable for other ecosystems nor agricultural or forestry biomass
burning practices.} %% optional section

%%%%%%%%%%%%%%%%%%%%%%%%%%%%%%%%%%%%%%%%%%
\begin{acknowledgements}
Thanks to (insert names here)
\end{acknowledgements}

%% REFERENCES
%% DN: pre-configured to BibTeX for rticles

%% The reference list is compiled as follows:
%%
%% \begin{thebibliography}{}
%%
%% \bibitem[AUTHOR(YEAR)]{LABEL1}
%% REFERENCE 1
%%
%% \bibitem[AUTHOR(YEAR)]{LABEL2}
%% REFERENCE 2
%%
%% \end{thebibliography}

%% Since the Copernicus LaTeX package includes the BibTeX style file copernicus.bst,
%% authors experienced with BibTeX only have to include the following two lines:
%%
\bibliographystyle{copernicus}
\bibliography{references.bib}
%%
%% URLs and DOIs can be entered in your BibTeX file as:
%%
%% URL = {http://www.xyz.org/~jones/idx_g.htm}
%% DOI = {10.5194/xyz}


%% LITERATURE CITATIONS
%%
%% command                        & example result
%% \citet{jones90}|               & Jones et al. (1990)
%% \citep{jones90}|               & (Jones et al., 1990)
%% \citep{jones90,jones93}|       & (Jones et al., 1990, 1993)
%% \citep[p.~32]{jones90}|        & (Jones et al., 1990, p.~32)
%% \citep[e.g.,][]{jones90}|      & (e.g., Jones et al., 1990)
%% \citep[e.g.,][p.~32]{jones90}| & (e.g., Jones et al., 1990, p.~32)
%% \citeauthor{jones90}|          & Jones et al.
%% \citeyear{jones90}|            & 1990


\end{document}
