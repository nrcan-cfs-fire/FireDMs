%% Copernicus Publications Manuscript Preparation Template for LaTeX Submissions
%% ---------------------------------
%% This template should be used for copernicus.cls
%% The class file and some style files are bundled in the Copernicus Latex Package, which can be downloaded from the different journal webpages.
%% For further assistance please contact Copernicus Publications at: production@copernicus.org
%% https://publications.copernicus.org/for_authors/manuscript_preparation.html

%% copernicus_rticles_template (flag for rticles template detection - do not remove!)

%% Please use the following documentclass and journal abbreviations for discussion papers and final revised papers.

%% 2-column papers and discussion papers
\documentclass[, manuscript]{copernicus}



%% Journal abbreviations (please use the same for preprints and final revised papers)

% Advances in Geosciences (adgeo)
% Advances in Radio Science (ars)
% Advances in Science and Research (asr)
% Advances in Statistical Climatology, Meteorology and Oceanography (ascmo)
% Aerosol Research (ar)
% Annales Geophysicae (angeo)
% Archives Animal Breeding (aab)
% Atmospheric Chemistry and Physics (acp)
% Atmospheric Measurement Techniques (amt)
% Biogeosciences (bg)
% Climate of the Past (cp)
% DEUQUA Special Publications (deuquasp)
% Earth Surface Dynamics (esurf)
% Earth System Dynamics (esd)
% Earth System Science Data (essd)
% E&G Quaternary Science Journal (egqsj)
% EGUsphere (egusphere) | This is only for EGUsphere preprints submitted without relation to an EGU journal.
% European Journal of Mineralogy (ejm)
% Fossil Record (fr)
% Geochronology (gchron)
% Geographica Helvetica (gh)
% Geoscience Communication (gc)
% Geoscientific Instrumentation, Methods and Data Systems (gi)
% Geoscientific Model Development (gmd)
% History of Geo- and Space Sciences (hgss)
% Hydrology and Earth System Sciences (hess)
% Journal of Bone and Joint Infection (jbji)
% Journal of Micropalaeontology (jm)
% Journal of Sensors and Sensor Systems (jsss)
% Magnetic Resonance (mr)
% Mechanical Sciences (ms)
% Natural Hazards and Earth System Sciences (nhess)
% Nonlinear Processes in Geophysics (npg)
% Ocean Science (os)
% Polarforschung - Journal of the German Society for Polar Research (polf)
% Primate Biology (pb)
% Proceedings of the International Association of Hydrological Sciences (piahs)
% Safety of Nuclear Waste Disposal (sand)
% Scientific Drilling (sd)
% SOIL (soil)
% Solid Earth (se)
% State of the Planet (sp)
% The Cryosphere (tc)
% Weather and Climate Dynamics (wcd)
% Web Ecology (we)
% Wind Energy Science (wes)

% Pandoc citation processing

% The "Technical instructions for LaTex" by Copernicus require _not_ to insert any additional packages.
% 
% tightlist command for lists without linebreak
\providecommand{\tightlist}{%
  \setlength{\itemsep}{0pt}\setlength{\parskip}{0pt}}


%%\usepackage{booktabs}
\usepackage{longtable}
\usepackage{array}
\usepackage{multirow}
\usepackage{wrapfig}
\usepackage{float}
\usepackage{colortbl}
\usepackage{pdflscape}
\usepackage{tabu}
\usepackage{threeparttable}
\usepackage{threeparttablex}
\usepackage[normalem]{ulem}
\usepackage{makecell}
\usepackage{xcolor}
%
%% \usepackage commands included in the copernicus.cls:
%\usepackage[german, english]{babel}
%\usepackage{tabularx}
%\usepackage{cancel}
%\usepackage{multirow}
%\usepackage{supertabular}
%\usepackage{algorithmic}
%\usepackage{algorithm}
%\usepackage{amsthm}
%\usepackage{float}
%\usepackage{subfig}
%\usepackage{rotating}

\begin{document}


\title{Incorporating fire severity for refined data-drive carbon
emissions estimates of boreal and temperate forest fires in the Carbon
Budget Model for Canadian forests}


\Author[]{}{}
\Author[1][daniel.thompson@nrcan-rncan.gc.ca]{Dan}{Thompson}
\Author[2]{Ellen}{Whitman}
\Author[3]{Hafer}{Mark}
\Author[2]{Oleksandra}{Hararuk}
\Author[1]{Chelene}{Hanes}
\Author[3]{Vinicius}{Manvailer Goncalves}
\Author[3]{Ben}{Hudson}


\affil[1]{Natural Resources Canada, Canadian Forest Service, Great Lakes
Forestry Centre, Sault Ste. Marie, Canada}
\affil[2]{Natural Resources Canada, Canadian Forest Service, Northern
Forestry Centre, Edmonton, Canada}
\affil[3]{Natural Resources Canada, Canadian Forest Service, Pacific
Forestry Centre, Victoria, Canada}

\runningtitle{Fire Severity and Canadian Forest Carbon Budget Models}

\runningauthor{Thompson et al.}


\correspondence{Dan\ Thompson\ (daniel.thompson@nrcan-rncan.gc.ca)}



\received{}
\pubdiscuss{} %% only important for two-stage journals
\revised{}
\accepted{}
\published{}

%% These dates will be inserted by Copernicus Publications during the typesetting process.


\firstpage{1}

\maketitle


\begin{abstract}
Wildfire is the most impactful natural disturbance to Canada's boreal
and temperate forest biomes. Current representations of fire impact on
forest carbon stocks is limited to a single parameterization of fire
severity (i.e.~the fraction of biomass consumed) that assumes only high
severity fires, despite a large and increasing evidence base of
widespread mixed-severity wildfire. In this submodel of the larger
Generic Carbon Budget Model for forest carbon accounting, field
measurements of biomass consumption as related to satellite-derived burn
severity maps are interpretted from a fire physics and ecology
perspective to derive algorithms to describe forest carbon fluxes in the
immediate aftermath of fires. Model outputs indicate total carbon
emissions range from a 11 t C/ha in Boreal Shield West forests of
Saskatchewan following low severity fire to over 70 t C/ha in Taiga
Cordillera and Pacific Maritime forests of Yukon under high severity
fire. Pacific Maritime forest showed the largest fraction of carbon
release in the canopy biomass pools (67\%), while Taiga Shield West and
Boreal Plains of the Northwest Territories and Manitoba were estimated
of having 82\% of carbon emissions in the surface and organic soil
biomass pools of litter, duff, and roots
\end{abstract}


\copyrightstatement{His Majesty the King in Right of Canada as
represented by the Minister of Natural Resources Canada. This work is
distributed under the Creative Commons Attribution 4.0 License.}


\section{Introduction}

Wildfire is on par with insects as the largest stand-replacing
disturbance process in Canada's forest, impacting \textasciitilde1-3 Mha
of Canada's 355 Mha forested area in a typical year \citep{hanes2019}.
In Canada's reliable 53-year burned area record, nine years have
exceeded 4 Mha of burned area (or approximately 1\% of Canada's forest
area) \citep{skakun2022}. The 2023 fire season in Canada burned a
remarkable 15 Mha owing to extreme drought, severe fire weather
conditions, and a prolonged fire season length \citep{jain2024}.

Of this, publicly-owned managed forest under long or short-term tenure
accounts for 23\% of the area burned between 1972 and 2024;
publicly-owned managed forest under long-term licence to private timber
companies forms 40\% of Canada's forest area \citep{stinson2019}.
Privately-owned forest constitutes only 6\% of the forest area (and
0.5\% of area burned), with the remainder of the forest area being a mix
of formally protected areas, remote unmanaged forest, Indigenous reserve
lands and other uses without large-scale harvesting. Managed forest
areas adjacent to communities have historically shown some meaningful
local fire suppression effects with a bias towards older forest nearby
boreal forest communities in Canada, though the effect is limited to a
25 km radius between these widely dispersed communities
\citep{parisien2020}.

Burned area in Canada is dominated by a relatively small number of very
large fires, with 3\% of fires constituting 97\% of the burned area
\citep{stocks2002}. Lightning-caused fires account for approximately
half of all ignitions and approximately 80\% of burned area, but no
distinction is made between human and lightning ignition for carbon
accounting purposes. Annual burned area mapping for carbon reporting in
Canada is conducted using a composite of satellite and aerial mapping at
30 metre resolution; the relatively small number of large fires, and
their slow vegetation regeneration \citep{white2017} allows for reliable
mapping using multi-spectral imagery such as Landsat within one year of
the fire \citep{whitman2018}.

{[}2 sentences on CBM-CFS3 and simple stuff about it{]}\ldots.. In
CBM-CFS, spatially referenced stand lists representing large homogenous
stands that fall within spatial units (ecozone-provincial intersections)
\citep{kurz2002}, even when applying precisely mapped burned areas
\citep{hall2020}. CBM-CFS currently assesses fire impacts to carbon
pools only as representing high-severity fire, which is the most common
of three severity classes of burned forest in Canada \citep{hall2008}.
Currently, biomass consumption estimates are based on the and the
assumption of complete crown mortality, with additional biomass
consumption following a spatially-referenced aggregated estimate at the
ecozone-province of annualized drought conditions. Quantification of the
change in carbon stock in CBM is made via a ``Disturbance Matrix''
(hereafter referred to as a DM) which is simply a matrix of the
proportional mass flux of carbon between each pair of pools in the model
for a site experiencing a given disturbance. This proportional mass flux
is independent of the size of the C pool. Pools in the DM include all
the above and below-ground pools tracked in CBM, as well as an
atmospheric sink pool. Importantly, unlike an emissions-only fire models
commonly used in air quality modelling in Canada \citep{chen2019}, DM
definitions in CBM also track the transfer of live biomass to dead, but
still uncombusted pools such as live stemwood (which are killed but not
burned) to standing deadwood, also known as snags.

In Canada's forests, a combination of disturbance history, soils, and
less frequently topographic variables determine leading tree species; at
local scales (1-100 ha), tree species plays a major role in determining
ecosystem susceptibility to fire \citep{bernier2016} where older,
conifer-dominated forests burn at very high rates relative to adjacent
deciduous or mixed stands. Even when deciduous and mixed forests do
burn, they do so at consistently lower severity compared to all but the
most xeric conifer forests \citep{whitman2018}. Thus, important local
biases in fire activity towards older and moderate to poorly-drained
forests are not resolved in the spatially-referenced CBM-CFS3 when only
a regionally-averaged fire severity is applied.

To support recent advances in operational burn severity mapping for
Canada \citep{whitman2020} alongside multi-decade reliable burned area
records that provide certainty on fire start and end dates
\citep{hall2020}, this paper describes a local scale (30-m) method for
defining a per-pixel proportional carbon flux measurement via a locally
calculated fire DM. In this document, we outline the evidence-based fire
Disturbance Matrices updated and designed for a spatially-explicit
update to the CBM, anchored in a three severity class paradigm. These
fire carbon flux models are built from a blend of aggregated field data
linked to remotely sensed severity, as well as insights from fire
physics and experimental fires. Key knowledge gaps are also highlighted,
with interim solutions presented until further quantification can be
done in field studies, such as from further wildfire observations,
experimental fires, or prescribed fires.

Simplified fire DMs (i.e.~a scalar reduction on 100\% mortality
assumption) have been used in scenario exercises using CBM for
assessment of future fire and harvest scenarios \citep{smyth2022}; the
algorithm development shown here provides an important data-driven and
regionally-adjusted framework that better reflects the ecological
nuances of moderate and low-severity fire across Canada's diverse
ecozones.

\section{Methods}

\subsection{Carbon Modelling}

\subsection{Carbon Modelling Spatial Units}

\subsection{Biomass pools of the Generic Carbon Budget Model}

Short section explaining the pool definitions most relevant to fire.

\subsection{Axioms of forest carbon budget after fire}

To simplify the process of the creation of the DMs as a distillation of
the complexities of fire severity and combustion patterns, the following
logical axioms are proposed and maintained throughout:

\begin{enumerate}
\def\labelenumi{\arabic{enumi}.}
\item
  Disturbance matrices are to be in terms of mortality, not survival.
  Mortality here is defined as tree death by the end of the calendar
  year of the fire's occurrence. Tree mortality in subsequent years is
  not modelled here.
\item
  Crown Fraction Burned (CFB) is a mass-based estimate of the portion of
  foliage consumed in the flaming passage of a fire, and is inclusive of
  merchantable and submerchantable trees, both broadleaf and needleleaf.
  Needles that are heat-killed but otherwise not consumed in the fire
  are not considered part of CFB, and are instead considered part of the
  foliage to litter biomass transfer.
\item
  The heat-killed but unconsumed fraction of the canopy is equal to
  (mortality - CFB).
\item
  In submerchantable trees, mortality == CFB.
\item
  In merchantable trees, CFB \textless= mortality.
\item
  Snags are inclusive of both those killed by prior fire as well as
  those killed by all other causes.
\end{enumerate}

Of these, Crown Fraction Burned (CFB) is an important concept used
primarily in fire behaviour science but not carbon accounting nor fire
ecology. CFB was introduced in the 1992 Fire Behaviour Prediction System
documentation \citep{forestrycanadafiredangerratinggroup1992}, and
provides a simple continuous 0-100 variable for only the consumption of
foliage (inclusive of both conifer and broadleaf). For our purposes, CFB
is the desirable metric as opposed to ordinal and less precise systems
like Canopy Fire Severity Index \citep{kasischke2000} that allows the
user to specify which pools of canopy biomass are consumed, but not the
precise fraction of each given pool that is consumed.

\subsection{Ground plot and remotely sensed fire severity data}

!!!Ellen to insert methods here - including the figure of where the
samples are from etc.

\subsection{Combustion gas emission ratios}

Certain variables, like the partitioning of
CO\textsubscript{2}:CH\textsubscript{4}:CO gas emissions, are constant
throughout ecozones, but vary by flaming vs smouldering combustion
modes. The precise emissions ratios vary slightly between models and
field studies, but for this initial algorithm assessment, we define
these emissions ratios as being identical to those used in Canada's
operational wildfire smoke forecasting system, FireWork
\citep{chen2019}. They are defined in a global variables table:

\begin{table}
\centering
\caption{\label{tab:globalVarsEmissions}Emissions factors in flaming and smouldering phase, expressed as portion of unburned biomass carbon content}
\centering
\begin{tabular}[t]{l|r|r}
\hline
Spp & Flaming & Smouldering\\
\hline
CO2 & 0.868 & 0.703\\
\hline
CO & 0.070 & 0.161\\
\hline
PM10 & 0.022 & 0.048\\
\hline
NMOG & 0.016 & 0.035\\
\hline
PM25 & 0.019 & 0.040\\
\hline
CH4 & 0.005 & 0.013\\
\hline
BC & 0.000 & 0.000\\
\hline
\end{tabular}
\end{table}

where CO\textsubscript{2} is responsible for 86.8\% of emissions in the
flaming phase, but only 70.3\% of emissions in the smouldering phase,
with a doubling of CO emissions and tripling of CH\textsubscript{4}
emissions. With a Global Warming Potential of CO equal to 1.9 and
CH\textsubscript{4} of 25, the Global Warming Potential per unit of
biomass consumption in the smouldering phase is 1.18 times higher in
global warming potential compared to flaming, not including differential
aerosol production and injection heights, however. With flaming and
smouldering each contributing roughly equally to wildfire emissions,
these distinct flaming and smouldering emissions ratios correspond well
prior emissions factors used in CBM. Note that as current described, the
sum of CO\textsubscript{2}, CH\textsubscript{4}, and CO emissions from
wildfires only represent approximately 95\% of the fire carbon mass
emitted to the atmosphere, with 0.5-2.0\% of biomass emitted as
particulate matter (e.g.~PM2.5, but also PM1 and PM10 classes of
particulates at 1 and 10 um diameters, respectively), and an additional
3\% \citep{hayden2022} to as little as 1\% \citep{simon2010} composed of
non-methane organic gases that have a large range in global warming
potentials as compared to CH\textsubscript{4}.

\subsection{Litter layer area-wise consumption by severity class}

The litter layer forms the first biomass pool in which a spreading fire
consumes fuel. In low-severity fires, the litter layer may be consumed
little to no underlying duff material consumed, nor any tree mortality
\citep{hessburg2019}. Logically, since litter consumption is largely
required for the ignition of the underlying duff layer, this litter
area-wise fractional consumption also informs and constrains duff
consumption.

\begin{table}
\centering
\caption{\label{tab:UnBurnedForestFloorByEcozone}Unburned litter area by ecozone and severity class.  The majority of the data comes from studies in the Boreal Plains and Boreal Shield West, and so values are extrapolated from those two well-observed ecozones to all others.}
\centering
\begin{tabular}[t]{l|r|r|r}
\hline
Ecozone & Low & Mod & High\\
\hline
BSW & 0.20 & 0.08 & 0.05\\
\hline
TP & 0.14 & 0.16 & 0.03\\
\hline
TSW & 0.20 & 0.08 & 0.05\\
\hline
BP & 0.14 & 0.06 & 0.02\\
\hline
BC & 0.14 & 0.06 & 0.02\\
\hline
BSE & 0.20 & 0.08 & 0.05\\
\hline
TSE & 0.20 & 0.08 & 0.05\\
\hline
MC & 0.14 & 0.06 & 0.02\\
\hline
HP & 0.20 & 0.08 & 0.05\\
\hline
TC & 0.14 & 0.06 & 0.02\\
\hline
PM & 0.14 & 0.06 & 0.02\\
\hline
AM & 0.14 & 0.06 & 0.02\\
\hline
MP & 0.14 & 0.06 & 0.02\\
\hline
P & 0.14 & 0.06 & 0.02\\
\hline
\end{tabular}
\end{table}

\subsection{Forest Floor Consumption}

While consumption of fine fuels in the litter layer of the forest floor
is nearly complete for any given fire intensity, consumption of deeper
organic soil horizons (F+H layers in upland forests and upper peat
layers in wetlands) is more drought dependent. In the fire literature in
Canada, the soil organic layer is sometimes termed the Forest Floor Fuel
Load (\emph{FFFL}) \citep{letang2012} and is dominated by the equivalent
organic soil Slow pool (Aboveground Slow Dead Organic Matter, or
AGSlowDOM) in CBM. Typically attention has been paid to the absolute
value of Forest Floor Fuel Consumption (\emph{FFFC}); however in the
case of carbon modelling, it is the relative fraction of consumption
(\emph{FFFC/FFFL}) that is of interest. In this scheme, we utilize a
composite of wildfire data from \citep{degroot2009} alongside the ABoVE
duff consumption data \citep{walker2020}, with an alternative modelling
approach to compute the relative amount of depth of consumption (scalar
from 0 to 1) rather than an absolute value in kg m\textsuperscript{-2}
or cm as otherwise done in the literature. A logit transform is used on
the scalar data to make it suitable for the fitted non-linear
least-squares modelling:

\begin{equation}
logit \left( \frac{depth of burn}{prefire depth}\right) = [3.83(1 - e^{(-0.005DC)})] + (-0.718log_{e}(AGSlow))
\end{equation}

where DC is the Fire Weather Index Drought Code, and AGSlow in the CBM
(given in Mg C/ha in this equation), and also synonymous with the the
Forest Floor Fuel Load (with ecozone averages given in
\citep{letang2012} or site-level data where observed). The modelling of
relative depth of burn had a higher skill than modelling of the relative
mass of consumption, given natural variability in soil density with
depth.

A conversion factor is then applied to the relative depth of burn data
to convert it to a relative mass consumption value. Since organic soil
density always increases non-linearly with depth, this conversion factor
from depth to mass is less than one:

\begin{equation}
\left( \frac{mass consumed}{prefire mass}\right) = \left( \frac{depth of burn}{prefire depth}\right) * CF
\end{equation}

where the Correction Factor is defined for boreal spruce fuel types as:

\begin{equation}
CF_{spruce} = 1.018(RelativeDepth)^{0.250}
\end{equation}

and for all other fuels as

\begin{equation}
CF_{nonspruce} = 0.13(RelativeDepth)+0.87
\end{equation}

with RelativeDepth as a value from 0-1.

While ultimately this scheme can be used on individual fires with
estimated or measured fuel loading and specific Drought Code values, for
the purposes of this first assessment, an ecozone-averaged fuel load and
decadal composites of Drought Code can also be used to provide
representative values. Specifically, a median Drought Code of detected
fire hotspots in Canada from 2003-2021 \citep{barber2024} using the same
data as the Canadian CFEEPS-FireWork wildfire air quality model of
\citep{chen2019} is presented below, along with proportional consumption
values of the forest floor by ecozone:

\includegraphics{GMD-Thompson-et-al_files/figure-latex/computeFixed-DM-DuffConsump-1.pdf}

\begin{table}
\centering
\caption{\label{tab:DuffConsumpTable}Fire Weather, fuel loading, and duff consumption values per ecozone.  Note that FFFL values 
 are from Letang et al., (2012) and not directly from the Reporting Unit values provided in the National Inventory Report. 
 Median Drought Code of burning is from Barber et al., (2024).}
\centering
\begin{tabular}[t]{l|l|l|l|l}
\hline
Ecozone & Median Drought Code of Burning & Median FFFL kg m-2 & FFFC kg m-2 & \% consumption\\
\hline
BSW & 239 & 6.864 & 3.22 & 0.47\\
\hline
TP & 369 & 11.97 & 6.12 & 0.51\\
\hline
TSW & 297 & 1.7556 & 1.37 & 0.78\\
\hline
BP & 242 & 7.1932 & 3.34 & 0.46\\
\hline
BC & 250 & 7.67844 & 3.56 & 0.46\\
\hline
BSE & 123 & 9.3522 & 1.85 & 0.2\\
\hline
TSE & 98 & 4.9764 & 1.15 & 0.23\\
\hline
MC & 452 & 4.32 & 3.21 & 0.74\\
\hline
HP & 204 & 6.1462 & 2.64 & 0.43\\
\hline
TC & 254 & 7.7616 & 3.63 & 0.47\\
\hline
PM & 268 & 13.5584 & 5.15 & 0.38\\
\hline
AM & 270 & 6.33 & 3.35 & 0.53\\
\hline
MP & 123 & 9.3522 & 1.85 & 0.2\\
\hline
P & 242 & 7.1932 & 3.34 & 0.46\\
\hline
\end{tabular}
\end{table}

Note that the maximum upland Forest Floor Fuel Load is approximately 30
kg m-2 (150 Mg C/ha) \citep{letang2012}; higher values are typically
seen only in peat ecosystems, where the above scheme is less
representative at a local scale since the fuel load (pool size) in
peatlands is much larger, but absolute consumption values are similar
between deep forest floor organic layers and peatlands
\citep{walker2020}. For Canadian peatlands, the CaMP model
\citep{bona2020} is instead used in CBM. Within CaMP, a separate
peatland water model driven by Drought Code determines the thickness of
the unsaturated peat layer, and an amount approximating 12\% of the
thickness of the unsaturated peat is consumed as smouldering
consumption. The peat-specific carbon pools and fire Disturbance
Matrices are fully described in \citep{bona2020}; large peatland trees
will still utilize the DM scheme described below. Since deeper forest
organic soil and peat layers show approximately similar absolute
consumption rates, for the purpose of this general model description no
peatland-specific components of the fire DMs are required.

Limited data is available on the fraction of woody debris consumption
alongside fire severity measurements. Coarse woody debris of overstory
stems that makes up 60-80\% of woody debris biomass in Canada's boreal
and temperate forests \citep{hanes2021}, with its moisture and
consumption patterns largely follows the moisture regime of the Drought
Code \citep{mcalpine1995}. In this modelling framework, the proportion
of coarse (\textgreater7.5 cm diameter) and medium (\textgreater0.5 cm
and \textless7.5 cm) woody debris consumption is estimated based on
detailed measurements of consumption from experimental fires. Coarse
Woody Debris is responsible for approximately 50-75\% of the total woody
debris load in most ecozones, and approximately 60\% of the total woody
debris consumption. Ecozone-level CWD consumption rates are summarized
as:

\begin{table}
\centering
\caption{\label{tab:CWDByEcozone}Coarse Woody debris consumption rates from pre/post measurements in experimental fires}
\centering
\begin{tabular}[t]{l|r|r|r}
\hline
Ecozone & Low & Mod & High\\
\hline
BSW & 0.024 & 0.163 & 0.140\\
\hline
TP & 0.000 & 0.218 & 0.238\\
\hline
TSW & 0.000 & 0.218 & 0.238\\
\hline
BP & 0.359 & 0.509 & 0.412\\
\hline
BC & 0.024 & 0.163 & 0.140\\
\hline
BSE & 0.080 & 0.131 & 0.182\\
\hline
TSE & 0.080 & 0.131 & 0.182\\
\hline
MC & 0.024 & 0.163 & 0.140\\
\hline
HP & 0.080 & 0.131 & 0.182\\
\hline
TC & 0.024 & 0.163 & 0.140\\
\hline
PM & 0.024 & 0.163 & 0.140\\
\hline
AM & 0.080 & 0.131 & 0.182\\
\hline
MP & 0.080 & 0.131 & 0.182\\
\hline
P & 0.359 & 0.509 & 0.412\\
\hline
\end{tabular}
\end{table}

Note that where historical burn severity data is not available, and
instead the fire classification type of surface, intermittent crowning,
and active crown fire are used as proxies for low, moderate, and high
severity fire, respectively. Fine woody debris \textless0.5 cm in
diameter is consumed at the exact same rate as the litter pool (see
section above).

\subsection{Drivers of C losses in the tree canopy}

\subsubsection{Overstory tree mortality and consumption}

Numerous process-driven \citep{michaletz2006} or empirical
\citep{hood2017} tree mortality models are present and show significant
skill in predicting tree mortality based on fire behaviour (i.e.~flame
length, rate of spread). Since the driving data in this model is
satellite-derived fire severity over the landscape scale, fire behaviour
metrics such as flame length or scorching height of bark are not
available as a continuous mapped product. Instead, softwood and hardwood
overstory mortality is calculated per ecozone as a function of
satellite-observed fire severity using aggregated ground plot data:

\begin{table}
\centering
\caption{\label{tab:MortByEcozone}Softwood fractional mortality by ecozone, as dervied from median values from field studies}
\centering
\begin{tabular}[t]{l|r|r|r}
\hline
Ecozone & Low & Mod & High\\
\hline
BSW & 0.45 & 0.81 & 1.00\\
\hline
TP & 0.45 & 0.81 & 1.00\\
\hline
TSW & 0.10 & 0.81 & 1.00\\
\hline
BP & 0.45 & 0.81 & 1.00\\
\hline
BC & 0.24 & 0.65 & 0.98\\
\hline
BSE & 0.45 & 0.81 & 1.00\\
\hline
TSE & 0.10 & 0.81 & 1.00\\
\hline
MC & 0.28 & 0.74 & 0.98\\
\hline
HP & 0.45 & 0.81 & 1.00\\
\hline
TC & 0.24 & 0.65 & 0.98\\
\hline
PM & 0.13 & 0.38 & 0.97\\
\hline
AM & 0.28 & 0.34 & 0.95\\
\hline
MP & 0.28 & 0.34 & 0.95\\
\hline
P & 0.45 & 0.81 & 1.00\\
\hline
\end{tabular}
\end{table}

And since large-diameter, live trees killed by fire do not experience
significant live stemwood consumption, the entirety of the live stemwood
biomass pool that is killed is transferred to the snag pool. Note that
the field data and disturbance modelling undertaken here only accounts
for tree mortality within the calender year of the fire, and delayed
mortality of over one year has been documented in boreal low and
moderate severity fires \citep{angers2011}where less than half of total
mortality occurs after the year of the fire. Thus, the modelling here
does not account for delayed mortality that may extend upwards of 5
years after fire.

Crown Fraction Burned (CFB) speaks to the fraction of the live canopy
that is itself consumed in the flaming front. The alternate outcomes
being survival of the foliage, or the mortality of the tree without
canopy consumption, resulting in the dropping of foliage onto the forest
floor. From the axioms stated earlier, the CFB must be lower than or
equal to the mortality rate, using field studies that show any partial
crown consumption is likely sufficient to result in high rates if not
complete mortality \citep{hood2017}, which is the case in Canada's trees
with primarily thin bark. Due to the structure of the CBM, all High
Severity fires have their mortality in the merchantable and smaller
trees set to exactly 1.0, which is no more than a 5\% variance from
observed values. From field studies, the following ecozone-specific CFB
values are found:

\begin{table}
\centering
\caption{\label{tab:CFBByEcozone}Softwood crown fraction burned by ecozone, as dervied from median values from field studies}
\centering
\begin{tabular}[t]{l|r|r|r}
\hline
Ecozone & Low & Mod & High\\
\hline
BSW & 0.0 & 0.81 & 1.00\\
\hline
TP & 0.0 & 0.81 & 1.00\\
\hline
TSW & 0.1 & 0.81 & 1.00\\
\hline
BP & 0.0 & 0.81 & 1.00\\
\hline
BC & 0.0 & 0.65 & 0.98\\
\hline
BSE & 0.0 & 0.81 & 1.00\\
\hline
TSE & 0.1 & 0.81 & 1.00\\
\hline
MC & 0.0 & 0.74 & 1.00\\
\hline
HP & 0.0 & 0.81 & 1.00\\
\hline
TC & 0.0 & 0.65 & 1.00\\
\hline
PM & 0.0 & 0.38 & 0.97\\
\hline
AM & 0.0 & 0.34 & 0.95\\
\hline
MP & 0.0 & 0.34 & 0.95\\
\hline
P & 0.0 & 0.81 & 1.00\\
\hline
\end{tabular}
\end{table}

The consumption of live bark biomass is a pool in the model, and
consumption rates can be defined by severity class. At the moment,
lacking robust field data on bark biomass consumption rates across
ecozones and severity classes (which are a small portion of the overall
biomass), the bark proportional consumption rate is set to 34\% of the
overstory mortality rate, based only on a single set well-observed high
severity fires in the Taiga Plains by {[}santín2015{]}.

A major distinction is made between softwood and hardwood trees, where
in Canada's boreal forests, a large fraction of hardwood trees (see
Appendix E) are able to resprout even when the main stem has been killed
by an intense forest fire \citep{brown1987}. Accordingly, the root
mortality rates differ greatly between softwoods and hardwoods, with
softwood root mortality equal precisely to stem mortality, while in
resprouting hardwoods, little root mortality is observed even after
intense fire \citep{pérez-izquierdo2019}. Though GCBM can resolve a
species list down to the pixel level, currently an ecozone-level
regional average composition of hardwood species with resprouting traits
is used and is shown below:

\begin{table}
\centering
\caption{\label{tab:ReSproutByEcozone}Ecozone-level average fraction of hardwood overstory species that do not suffer extensive belowgroud biomass mortality after fire}
\centering
\begin{tabular}[t]{l|r}
\hline
Ecozone & Resprout Fraction\\
\hline
BSW & 0.75\\
\hline
TP & 0.75\\
\hline
TSW & 0.94\\
\hline
BP & 0.99\\
\hline
BC & 0.76\\
\hline
BSE & 0.67\\
\hline
TSE & 0.78\\
\hline
MC & 0.97\\
\hline
HP & 0.80\\
\hline
TC & 0.27\\
\hline
PM & 0.39\\
\hline
AM & 0.76\\
\hline
MP & 0.32\\
\hline
P & 0.99\\
\hline
\end{tabular}
\end{table}

Concurrently, the fraction of fine roots contained within the
combustible forest floor layers can be a close to or exceeding 50\% of
the fine root biomass \citep{strong1985}, and burns alongside the
organic soils \citep{benscoter2011}. As a result, the calulation for
softwood fine root consumption and mortality are as follows, using
Softwood as an example:

\begin{equation}
SWFineRootConsump = SW.Mort \times SW.Prop.Fine.Root.duff \times Duff.Consump.Fract
\end{equation}

\begin{align*}
\text{SWFineRootMort.AG} = \text{SW.Mort} \times SW.Prop.Fine.Root.duff \times (1-Duff.Consump.Fract) \times (1-ReSproutFactor)
& \text{SWFineRootMort.BG} = SW.Mort \times (1-SW.Prop.Fine.Root.duff)
\end{align*}

In contrast, the larger diameter of the coarse root biomass pool
prevents its consumption during any smouldering of the duff layer, and
the mortality rate of coarse roots is simply proportional to that of the
stemwood overall.

\subsubsection{Understory tree mortality and consumption}

Understory (or small diameter overstory) tree mortality is defined
seperately in the model, but given the lack of data on diameter classes
in the severity data, robust field data on differing mortality rates of
smaller diameter trees is not available, and so the understory tree
mortality rate is set equal to the overstory rate as defined in the
table above. Note that trees with a top height less than 1.4 m are not
considered in this pool, and instead are lumped into the ``other'' pool.

\subsubsection{Snag and stump consumption}

Dead standing trees and branches on average is a biomass pool much
smaller than organic soils or coarse woody debris on the ground, but in
areas of extensive insect killed trees, recent fires, and blowdown, is a
substantial biomass pool. In this initial algorithm description, only
typical boreal forest stands where snags are a small fraction of total
stems are considered. Compared to live stemwood of the same diameter,
the low moisture content of snags and snag branches allows for much
greater consumption during the passage of an intense flaming front
\citep{stocks2004}. The snag branch pool experiences almost complete
combustion in high severity fires \citep{talucci2019}, and is highly
correlated with live foliage consumption (i.e Crown Fraction Burned)
\citep{degroot2022}. The largest biomass pool of the main standing dead
stemwood remains at most approximately 50-60\% consumed. In the absence
of extensive field data from CBI plots to quantify snag and snag branch
consumption, snag (dead stemwood) consumption fraction is estimated as:

\begin{equation}
Snag Consumption fraction = \left( 0.5 \times Crown Fraction Burned \right) + 0.05
\end{equation}

And for snag branches:

\begin{equation}
Snag Branch Consumption = \left( 0.9 \times Crown Fraction Burned \right)
\end{equation}

Stumps are tracked as their own biomass pool in CBM. As they are
typically in contact with the upper forest floor, are assumed to be
consumed at the same rate as coarse woody debris.

\subsection{Calculation of Annual Direct C Emissions from Fire}

To compute an estimate of the total direct C emissions from forest fires
in Canada in 2023, the classified burn severity product from the
National Burned Area Composite annual production was utilized (see Hall
et al 2020 for an algorithm description). In NBAC, an internal tracking
value ``NFIREID'' is utilized, which is the final satellite-derived
burned area polygon (allowing for multi-part polygons) is split across
any RU unit boundaries (if any). Since carbon pool sizes vary across RU
boundaries, this allows for a single NFIREID to be present across
multiple RUs.

A total of 2199 fires as small as 0.09 ha (one 30x30 landsat pixel) were
mapped by NBAC for burn severity for a total of 14.60 Mha, but only
fires over 100 ha were utilized as a lower limit of where meaningful
per-fire estimates of the fraction of low, moderate, and high severity
burned area was available. Including only fires 100 ha and larger
reduced the total number of fires to 966 but the total area remained
largely the same at 14.58 Mha. A total of 189,704 ha of post-fire
salvage logging was also mapped in 2023 and is assigned to the moderate
severity class after consultation with provincial land managers. The
direct C emissions from fire shown here are not altered by the act of
post-fire salvage logging. Additionally, the NBAC mapping process
accounts for unburned islands (and areas with a mapped fire severity no
different than unburned) which count towards the total fire area but do
not have a disturbance matrix and direct C estimate applied.

A unique DM was then calculated for each fire using the median
area-weighted Drought Code per fire. All thermal detection hotspots from
VIIRS that intersect a fire were extracted from the historical hotspot
archive that supports the Canadian smoke emissions model CFFEPS-FireWork
\citep{chen2019}. The median DC value across all intersecting hotspots
was used to derive a single DM per fire, no matter the duration of
burning.

To compute total direct fire emissions per fire, a single estimate of
the carbon pool size based on the Reporting Unit of the centroid of the
NFIREID polygon was applied. Since polygons are split across RU
boundaries, the spatial weighting of the pool size per fire is performed
automatically. While spatially explicit biomass maps are available for
some aboveground components \citep{guindon2024} and some organic soil
components \citep{hanes2022}, the majority of the required pool sizes in
CBM for the computation of the fire DMs are available only at the
spatially referenced RU scale.

\section{Results and Discussion}

\begin{landscape}\begin{table}[!h]
\centering
\caption{\label{tab:makeTables}High severity Disturbance Matrix in BP}
\centering
\resizebox{\ifdim\width>\linewidth\linewidth\else\width\fi}{!}{
\begin{tabular}[t]{>{\raggedright\arraybackslash}p{3cm}|>{\centering\arraybackslash}p{2cm}|>{\centering\arraybackslash}p{2cm}|>{\centering\arraybackslash}p{2cm}|>{\centering\arraybackslash}p{2cm}|>{\centering\arraybackslash}p{2cm}|>{\centering\arraybackslash}p{2cm}|>{\centering\arraybackslash}p{2cm}|>{\centering\arraybackslash}p{2cm}|>{\centering\arraybackslash}p{2cm}}
\hline
  & Softwood Merchantable & Softwood Stem Snag & Medium DOM
 & Softwood Foliage & Aboveground Very Fast DOM & CO2 & CH4 & CO & PM25\\
\hline
Softwood Merchantable & 0 & 1 &  &  &  &  &  &  & \\
\hline
Softwood Stem Snag &  & 0 & 0.45000 &  &  & 0.4774000 & 0.0027500 & 0.0385000 & 0.0104500\\
\hline
Medium DOM &  &  & 0.57624 &  &  & 0.2979033 & 0.0055089 & 0.0682254 & 0.0169504\\
\hline
Softwood Foliage &  &  &  & 0 & 0.00 & 0.8680000 & 0.0050000 & 0.0700000 & 0.0190000\\
\hline
Aboveground Very Fast DOM &  &  &  &  & 0.02 & 0.8506400 & 0.0049000 & 0.0686000 & 0.0186200\\
\hline
CO2 &  &  &  &  &  &  &  &  & \\
\hline
CH4 &  &  &  &  &  &  &  &  & \\
\hline
CO &  &  &  &  &  &  &  &  & \\
\hline
PM25 &  &  &  &  &  &  &  &  & \\
\hline
\end{tabular}}
\end{table}
\end{landscape}

\subsection{Direct fire carbon emissions as a function of fire severity
and drought}

\includegraphics{GMD-Thompson-et-al_files/figure-latex/non-GWP-total-C-plot-1.pdf}

\subsection{Comparison against observed direct fire emissions}

\subsubsection{Forest fire observations of CO and CO2 emissions ratios}

\begin{table}
\centering\centering
\caption{\label{tab:MCE-from-airborne}Comparison of the Modified Combustion Efficiency (MCE) of airborne gas measurements of Canadian wildfires against modelled MCE}
\centering
\resizebox{\ifdim\width>\linewidth\linewidth\else\width\fi}{!}{
\begin{tabular}[t]{l|l|l|l|l|l|l}
\hline
Study & Date & Subset & Ecozone & Drought Code & Obs MCE & Modelled MCE\\
\hline
Hornbrook et al 2011 & 2008-07-01 & Afternoon (0.2Low 0.4Mod 0.4High) & Boreal Shield West & 250 & 0.92 & 0.912\\
\hline
Hornbrook et al 2011 & 2008-07-01 & Late Evening (100\% low severity) & Boreal Shield West & 250 & 0.82 & 0.901\\
\hline
Hornbrook et al 2011 & 2008-07-04 & After rain smouldering only low severity & Boreal Shield West & 275 & 0.83 & 0.901\\
\hline
\end{tabular}}
\end{table}

Mean observed Modified Combustion Efficiency was observed by
\citet{hornbrook2011} for distinct periods during the ARCTAS campaign
over northern Saskatchewan in 2008. MCE observed by aircraft during the
peak burn period when the majority of fuel consumption and area burned
occurs (0.92) largely corresponded with modelled values (0.91),
suggesting the model provides a fair representation of the balance of
flaming and smouldering during large active wildfires. Subsequent smoke
plume observations during periods of greatly reduced spread and
intensity (late evening and after rain) showed a substantially reduced
MCE of 0.82-0.83 that indicates a near lack of flaming combustion. Even
when represented as 100\% low severity fire in the model, the modelled
MCE only declines to 0.903. Since the fire DM model is based on area
burned, fire activity such as smouldering but with minimal actual area
burned increased is going to show an MCE far lower than areas of low
severity fire spread, which are typically still a low sub-canopy flaming
front that features a mix of smouldering duff and woody debris alongside
flaming consumption of litter \citep{mcrae1994, mcrae2017}.

\subsection{2023 Canadian wildfire season total direct emissions}

!!! results table to be completed for the 2023 emissions !!!

A total of 501 Mt C was estimated to be released directly by fires in
Canada in 2023. The total mass of PM2.5 emissions was estimated as
\ensuremath{3.2284828\times 10^{7}} Mt of total mass, assuming a 50\%
carbon content.. The area-weighted mean Drought Code for all fires in
Canada in 2023 was 444, representing severe but not uniformly
exceptional drought conditions in Canada's northern forests. The
area-weighted mean total C emissions per unit area was 34.36, though
90\% of the emissions per unit area were between the 5th percentile of
14 t C/ha typical of the Taiga Plains ecozone under low drought
conditions and the 95th percentile of 14 t C/ha typical of Pacific
coastal forests.

Byrne et al 2024 used observed total atmospheric column excess CO and a
range of MCE values to estimate total fire C emissions in Canada in 2023
of between 570-727 Mt C with a mean estimate of 647 Mt. The uncertainty
in the estimate from Byrne et al 2024 lies primarily in the uncertain
CO\textsubscript{2}:CO ratio (more commonly computed as the normalized
ratio MCE). Our bottom-up estimates that partition flaming vs
smouldering shows a lower CO\textsubscript{2}:CO estimate of 6.06. This
lower CO\textsubscript{2}:CO ratio (i.e.~more smouldering) if applied to
the data from Byrne et al 2024 would reduce their lower range of total C
emissions to 508.81 Mt C, which is comparable to the 501 Mt C computed
from this bottom-up approach. Estimates of total carbon emissions based
solely on the sum of observed Fire Radiative Power (FRP) from
Copernicus-CAMS was approximately 490 Mt C
(\url{https://atmosphere.copernicus.eu/smoke-canadian-wildfires-reaches-europe})

\subsubsection{Relationship to other fire carbon models}

The forest floor emissions modelling scheme used here builds upon the
CanFIRE model \citep{degroot2013} which incorporates a similar estimate
of absolute forest floor emissions based on fuel type and various Fire
Weather Index inputs (including but not limited to Drought Code). While
the CanFIRE model is also capable of estimating tree mortality, the
premise of CanFIRE lies in the use of the Canadian fire Behaviour
Prediction (FBP) System to model fire intensity and a
physiological-thermodynamic model of tree damage and mortality. CanFIRE
is able to run entirely in scenario or forecast mode, i.e.~no fire
severity map is needed to run the model, unlike the focus here on a
mechanism to estimate carbon fluxes and pools after fire severity is
mapped. Ultimately, models like CanFIRE can be used for near-real time
emissions estimates of direct and indirect fire carbon emissions (and
also for prescribed fire planning and scenario testing), while this
modelling framework is best used solely operational carbon accounting
and reporting given its strong dependence severity maps. While fire
severity can be estimated empirically based on geospatial inputs without
observation using multispectral satellite data such as Landsat
!!!{[}Ellen's paper?{]}!!!, estimation using coupled fire behaviour and
ecology models such as CanFIRE is a more direct approach. Under severe
drought conditions and light winds, very large fires often create more
wind and energy than is available in gradient winds as measured at
nearby airports and fire weather observing stations !!!(cite Coen paper
on fire-induced winds)!!! Thus, fire behaviour and fuel consumption
models solely driven by wind inputs from distant meteorological stations
can under-predict fire intensity and therefore likely fire severity.
While fire-atmosphere models are available for research !!{[}cite coen
again{]} and some forecasting and planning purposes \citep{linn2020},
modelling resources are typically not made available for the extensive
and largely unsuppressed fires in northern Canada. Fire severity mapping
after the fact is able to account for high fire severity
\citep{whitman2018} even when fire spread occurs under light winds
\citep{whitman2024}.

While the fire DMs presented here are designed for use in the CFS-CBM
carbon reporting framework, their simplicity and similarity to many
other forest carbon schema allows them to be used elsewhere such as in
other forest dynamics \citep{brecka2020} and earth system models
\citep{melton2020} that require estimates of Canadian forest carbon
emissions from wildfire. \citet{stenzel2019} showed that
severity-informed fire C estimates (with snags) in the US are only
30-40\% as large as fixed or variable severity data using their
LANDIS-type models. Large over-estimates of in-fire bole consumption in
models were not observed to correspond to measurements in the field. In
the model presented here, our estimates of snag consumption are based on
simple but logical relationships with Crown Fraction Burned, but more
comprehensive data collection and field observations would be required
to extend the accuracy of the snag consumption scheme.

\subsubsection{Indirect fire emissions}

While fire emissions within the year of the fire are majority of the GHG
flux, enhanced post-fire decomposition of dead biomass persists for many
years after a fire. The increase in post-fire tree mortality between the
year of the fire and the year following the fire is as much as 30\% in
low and moderate-severity stands \citep{angers2011}. This extended
mortality is not captured in year-of fire severity mapping but is likely
better assessed using fire severity data captured the year following
fire, as is operation practice in Canada !!!!{[}cite Ellen's upcoming
paper{]}!!!. Similarly, the transition of fire-killed stems (snags) from
upright to the forest floor woody debris takes between 5-8 years for
50\% of stems to fall \citep{angers2011} with stemfall rate highest in
low-severity fire. Field observations show that initial assessments of
fire severity are the primary predictors of snag fall \citet{angers2011}
and decomposition \citep{boulanger2011} rates, meaning the mapped
severity method used here provides a useful input for multi-year fire
carbon modelling of the snag pool in CBM. For organic soil pools, fire
has been shown to suppress decomposition rates in upland soil pools
where drier post-fire conditions are present \citep{holden2015}, while
in permafrost upland \citep{odonnell2011} and peatland
\citep{gibson2018, gibson2019} systems, fire has profound impacts on
soil carbon pools by rapidly increasing thaw depths and subsequent
decomposition rates. Currently in CBM, decomposition rates are
calibrated against field decomposition litterbag experiments
\citep{trofymow2002} and annual climate metrics that do not take into
account how disturbances such as fire (and fire of varying severity)
modify decomposition rates in the absence of a change in climate.

\subsubsection{Model gaps}

Currently, the model framework only accounts of regionally-averaged soil
carbon stocks, meaning that Reporting Units with high peatland areas
will show higher organic soil slow pool size (see Table \#\#\#), but
peatlands themselves are not spatially represented in the model.
Currently, fire disturbance in peatlands is performed by the CaMP model,
which uses the same Drought Code fire weather metric to estimate peat
layer consumption \citep{bona2020} and uses the overstory carbon schema
of CBM to estimate overstory tree carbon pools and fluxes. Direct
satellite monitoring of upland forest organic soil and peatland carbon
loss via burn severity monitoring is possible but remains a challenge to
implement with accuracy for monitoring purposes
\citep{bourgeau-chavez2020}. Despite the large carbon stock, the mean C
emissions per area in peatland due to fire (65 t C/ha) as modelled by
\citet{bona2024} are comparable to regional emissions from peat-rich
ecozones such as the Boreal and Taiga Plains under moderate to severe
drought conditions. Indeed, it is under these moderate to severe drought
conditions under which peatlands burn more frequently
\citep{turetsky2004, thompson2019} and severely \citep{kuntzemann2023}.

This algorithm is currently built off of regionally averaged biomass
pools by Reporting Unit (Figure 1; Appendix \#\#\#). While Canadian
wildfires shown an overall selection bias towards preferentially burning
older conifer stands \citep{bernier2016}, fuel selectivity against less
flammable portions of the landscape decreases strongly under severe
drought conditions \citep{parks2018} such as those experienced in Canada
in 2023 \citep{jain2024} where we performed the algorithm validation
against independent fire emissions observations. Algorithm evaluation
against individual fires or years of less profound drought would be more
likely to highlight the limitation of the regionally-averaged carbon
pool (i.e.~fuel load) values used here. Ultimately, the mapped severity
used here is best paired with mapped carbon pools at a similar scale,
such as the 1-ha forest carbon modelling for analysis and scenario
testing used by \citet{smyth2024}.

Species-specific traits in trees with overlapping ranges, such as
resprouting is aspen \citep{brown1987} but not other broadleaves such as
oak or maple, are not explicitly handled in this model. Instead, the
relative abundance of trees with contrasting traits that influence
ecological outcomes such as fire survival are lumped at the ecozone
scale from the aggregation of the plot data. While ecozone-level
contrasts in key considerations such as overstory mortality do strongly
vary by ecozone !!!!{[}Table \#\#\#{]}, the same DM (e.g.~hardwood root
mortality) is applied in adjacent stands at the same severity class
despite the potential for strongly contrasting traits that are readily
tied back to mapping at the stand level by leading species. Alternately,
models such as CanFIRE can account for species-level contrasts in fire
ecology traits, but compiling the relevant ecophysiological properties
and traits for all tree species in Canada remains incomplete at the
current time.

\section{Conclusions}

Carbon emissions from wildfires in Canada represent a substantial pulse
input of greenhouse gases to the atmosphere: the 2023 fire emissions in
Canada were an input of approximately 501 Mt C. A modelling framework to
extend the current fire and greenhouse gas reporting in Canada is
presented. In contrast to the existing modelling system that utilizes a
fixed fire severity (i.e.~tree mortality) assumption, a field-calibrated
satellite fire severity mapping process that follows mature and
well-established scientific methods is used. Above-ground carbon pool
changes (from live to dead and in situ as well as to the gas phase) rely
on a robust field observation dataset that relates back to satellite
metrics. Evaluation of the modelling presented above against fully
independent airborne observations of CO:CO\textsubscript{2} emissions
ratios for boreal wildfires indicates the modelling of proportion of
flaming vs smouldering emissions are well-replicated by the model. With
confidence in this modelled CO:CO\textsubscript{2} ratio, the total fire
CO emissions for the 2023 wildfire season in Canada (with a record 15
Mha burned) is broadly consistent with fire season estimates using
satellite total column CO anomalies.

While this prototype carbon flux framework utilizes regionally-averaged
carbon stock estimates, future systems deployment within Canada's carbon
accounting system is likely to incorporate precisely mapped (1 ha)
carbon pools alongside the finely mapped fire severity products (30 m)
used here. As both the spatial estimates of carbon stocks across
Canada's forested ecosystems improves alongside additional field
observations, it is anticipated that this modelling framework will
increase in both accuracy and precision over time.

\section{Appendix A: Mean pool size by Reporting Unit}

\section{Appendix B: non-linear least squares modelling of soil organic
layer consumption}

For national annual estimates of forest organic soil layer consumption
during wildfire, implementations that only utilize Canadian experimental
fire data from the Fire Behaviour Prediction System will be limited to a
maximum consumption value of 5 kg (biomass) m\textsuperscript{-2} of
total surface fuel (woody debris, litter, and duff) of 5 kg
m\textsuperscript{-2}, or 25 Mg C ha\textsuperscript{-1}, given the
observation dataset and fitted model parameters. For the common ``C-2''
boreal spruce fuel type for instance, Surface Fuel Consumption, SFC
(biomass units in kg m\textsuperscript{-2} not kg C) is modelled as:

\begin{equation}
SFC = 5.0 \left(1-e^{-0.0115BUI} \right)^{1.0}
\end{equation}                                             

This model form has the distinct advantage of SFC being 0.0 at a BUI of
zero. The model parameters vary by fuel type (i.e.~deciduous broadleaf
fuels are limited to 1.5 kg m\textsuperscript{-2} of maximum SFC) but
are fixed within a fuel type.

More recent observations and modelling from \citet{degroot2009} extended
the FBP data with an additional 128 observations from 7 additional
wildfires, and the ABoVE project compiled over 1,000 field observations
of depth of burn and C stocks before and after wildfire in Canada and
Alaska, over 600 of which are in North American Level II ecoregions also
occurring in Canada \citep{walker2020}. \citet{degroot2009} provides a
concise and informative improvement on the FBP fuel consumption
equations, where both a Fire Weather Index System component (in this
case, Drought Code) is used similarly to Buildup Index in the FBP, but
importantly, the site-level organic soil layer fuel load is also
accounted for, which allows for the greater absolute combustion in
deeper organic soils that is moderated by the natural logarithm
transformation:

\begin{equation}
log_{e}(FFFC) = -4.252+0.710log_{e}(DC) + 0.671log_{e}(FFFL)
\end{equation}

where FFFC is Forest Floor Fuel Consumption (SFC minus surface woody
debris) in kg (biomass) m\textsuperscript{-2} and FFFL is Forest Floor
Fuel Load in kg m\textsuperscript{-2}. The forest floor as defined here
is inclusive of the litter and duff layers, live mosses and lichens.
This model presented above fits well within the dataset and extends the
observed maximum FFFC to nearly 10 kg m\textsuperscript{-2}. The ABoVE
synthesis of FFFL and FFFC (Walker et al 2020 ORNL) expands upon a
slightly smaller dataset used in a modelling summary also by Walker et
al (2020 NCC), where structural equation modelling was used to explore
drivers of FFFC but no concise and readily reproducible modelling is
produced. The results of the SEM from Walker et al (2020 NCC) emphasized
a greater role of FFFL over DC, though coarse reanalysis that lacked
local fire agency weather stations was used. An analysis of just 2014
fires in the Northwest Territories by (walker2018) showed that while the
mean depth of burn across all black spruce stands was 6-10 cm, the
driest (xeric) black spruce stands with the smallest FFFL showed upwards
of 75\% soil organic consumption, while deeper organic soils in
subhygric black spruce stands showed less than 25\% consumption.

To provide the largest possible dataset for FFFC and FFFL, the ABoVE
synthesis was combined with wildfire data from de Groot et al 2009 not
otherwise found in the ABoVE synthesis. The ABoVE synthesis sites in the
Alaska Boreal Interior ecoregion, which have equivalent Canadian ecozone
were excluded, but Alaska Boreal Cordillera sites near the Yukon border
were utilized. Experimental fire data from the FBP data was not used, as
deeper combustion measurements resulting from hours and days of
smouldering combustion captured in wildfire data are not available in
experimental fires where extensive smouldering is not measured due to
suppression. In order to best represent on-the-ground Fire Weather Index
values, the Drought Code and other FWI values from Walker et al 2020
reanalysis were substituted with interpolated weather station (both
Environment and Climate Change Canada as well fire agency stations).
This data also has the benefit of being properly overwintered for
Drought Code (Hanes DC overwinter) and capturing small rain events not
captured in reanalysis that meaningfully impact the Duff Moisture Code
in particular.

For the purposes of improving national estimates of the fractional soil
organic layer loss during wildfire, this framework emphasizes the
proportional C stock loss (as with all CBM disturbance matrices) rather
than the absolute value of combustion. In contrast to the modelling of
absolute combustion value, any analysis of proportions is best conducted
as logit- transformed data, where the logit transformation is:

\begin{equation}
logit(p) = log\frac{p}{1-p}
\end{equation}

which effectively transforms a data of proportions of {[}0,1{]} to a
Gaussian distribution with a range of approximately -5 to +5 (in this
dataset), and a mode approximately at zero. Within the logit-transformed
data, exploratory analysis of ecozones as a factor alongside other
non-linear splines of FWI values and FFFL was conducted:

\begin{verbatim}
## 
## Family: gaussian 
## Link function: identity 
## 
## Formula:
## prop_sol_combusted_logit ~ s(drought_code, k = 4, bs = "tp") + 
##     ecozone + s(AGSlow.Mg.C.ha, k = 4, bs = "tp")
## 
## Parametric coefficients:
##             Estimate Std. Error t value Pr(>|t|)  
## (Intercept)  0.11939    0.09318   1.281   0.2005  
## ecozoneBP   -0.29433    0.15506  -1.898   0.0581 .
## ecozoneBSW  -0.19928    0.15988  -1.246   0.2131  
## ecozoneTP   -0.23049    0.11476  -2.008   0.0450 *
## ecozoneTSW  -0.29887    0.12157  -2.458   0.0142 *
## ---
## Signif. codes:  0 '***' 0.001 '**' 0.01 '*' 0.05 '.' 0.1 ' ' 1
## 
## Approximate significance of smooth terms:
##                     edf Ref.df       F p-value    
## s(drought_code)   1.352  1.618   2.862  0.0631 .  
## s(AGSlow.Mg.C.ha) 2.950  2.998 232.720  <2e-16 ***
## ---
## Signif. codes:  0 '***' 0.001 '**' 0.01 '*' 0.05 '.' 0.1 ' ' 1
## 
## R-sq.(adj) =  0.628   Deviance explained = 63.3%
## -REML = 869.04  Scale est. = 0.81207   n = 651
\end{verbatim}

(could also show boxplots?) which shows that the Boreal Cordillera
ecozone has a meaningfully higher proportional soil organic layer
consumption rate compared to BP, BSW, TP, and TSW, all of which are not
significantly different in consumption rates once DC and FFFC (shown as
AGSlow.Mg.C.ha) are accounted for. Boreal Cordillera data were set aside
for a distinct model.

Similar to \citet{degroot2009}, Drought Code was a better predictor of
consumption rates than Buildup Index as used in the FBP. In the logit
transformed space, saturation-type non-linear curve using the relevant
FWI component was fitted in a non-linear least squares model, but an
additive term of the natural-logarithm transformed FFFC (given as AGSlow
pool in Mg C/ha) was used as well. In the end, a superior model was
found using the proportional depth of burn, rather than the proportional
loss of the mass of the soil organic layer. The non-linear least squares
model fit was conducted using the Levenberg-Marquardt nonlinear
least-squares algorithm found in MINPACK (Elzhov et al 2023) R package,
which supported bounded parameter constraints.

In the abstract, the model follows the form:

\begin{equation}
logit \left( \frac{depth of burn}{pre-fire organic depth}\right) = [c (1 - e^{(a  DC)})] + (b log_{e}(AGSlow))
\end{equation}

with fitted parameters as:

\begin{equation}
logit \left( \frac{depth of burn}{pre-fire organic depth}\right) = [3.83 (1 - e^{(-0.005 DC)})] + (-0.718 log_{e}(AGSlow))
\end{equation}

Note the ``b'' coefficient on the parameter associated with the FFFL
(AGSlow) of -0.718, which results in larger organic layer fuel loads
leading to smaller proportional consumption values, which follows the
patterns shown by Walker 2018 for NWT fires of 2014.

The a parameter term that forms the exponent of e alongside Drought Code
is related to the DC value at which half of the maximum possible
asymptotal consumption value is observed (for a given FFFL value). The
NLS fitting was given a minimum value of -0.06 such that half of the
asymptotal maximum consumption rate was modelled as occurring at or
around a DC value of 300. The other parameters were fit to the best
possible value with no constraint.

Importantly, since fire behaviour, emissions modelling, and carbon
accounting all operate with the calculation of mass loss and depth of
burn, a correction factor was applied that corrects for the trends in
bulk density with depth for C-2 fuels as given by \citet{degroot2009},
so that a model of proportional depth of burn is then converted into a
proportional mass loss term:

\begin{equation}
\left( \frac{FFFC}{FFFL}\right) = 1.017 \left( \frac{depth of burn}{pre-fire organic depth}\right)^{0.250}
\end{equation}

which for example means that the median proportional depth of burn in
the AboVE/de Groot training data of 0.40 corresponds to 0.32 of the
proportional mass loss (since shallow organic soil is less dense), or a
correction factor of 0.80. For non-spruce-dominated fuels, this
correction is much smaller but still meaningful:

\begin{equation}
\left( \frac{FFFC}{FFFL}\right) = 0.13 \left( \frac{depth of burn}{pre-fire organic depth}\right)+0.87
\end{equation}

(show perhaps what a few different a values look like?) (also compute
the DC value of about half the maximum FBP SFC values too, just for
kicks)

For example, using a moderately thick \textasciitilde12 cm thick organic
soil layer, the proportion of consumption as a function of Drought Code
using the model above

\includegraphics{GMD-Thompson-et-al_files/figure-latex/example-FFFC-vs-DC-plot-nls-1.pdf}

With the parameter constrained NLS fitting, the proportional consumption
model for the forest floor has a leave-one-out (conducted at the
fire-level, not plot) cross validated r2 of \#\#\#, and a Mean Percent
Error of \#\#\#\%

\begin{verbatim}
## Scale for y is already present.
## Adding another scale for y, which will replace the existing scale.
## Scale for y is already present.
## Adding another scale for y, which will replace the existing scale.
## Scale for x is already present.
## Adding another scale for x, which will replace the existing scale.
\end{verbatim}

\includegraphics{GMD-Thompson-et-al_files/figure-latex/show-biplot-FFFC-new-1.pdf}

Across the entire parameter space of Drought Code and AGSlow pool size,
the following isolines of proportional consumption in the model can be
plotted:

\includegraphics{GMD-Thompson-et-al_files/figure-latex/FFFC-surface-plot-1.pdf}

\includegraphics{GMD-Thompson-et-al_files/figure-latex/correction-factor-1.pdf}

\subsubsection{Boreal Cordillera modelling}

For the Boreal Cordillera, a simple linear model on the
logit-transformed data was found to be the best performing model for
estimating soil organic consumption proportion:

\begin{equation}
logit \left( \frac{FFFC}{FFFL}\right) = (0.00257 * DC) + (-0.54 * log_{e}(AGSlow)) + 2.17
\end{equation}

Despite the presence of a fixed intercept term, the strong inverse
dependence on the natural logarithm of the AGSlow pool size (FFFL)
results in zero proportional consumption when Drought Code is equal to
zero.

\section{Appendix C: annual variability in observed Drought Code during
wildfire spread, and impact on ecozone-level average forest floor
emissions}

\includegraphics{GMD-Thompson-et-al_files/figure-latex/AnnualDC-Vis-DC50-1.pdf}

\includegraphics{GMD-Thompson-et-al_files/figure-latex/AnnualDC-Vis-Consump-1.pdf}

\section{Appendix D: Representative photos}

Photos of: (1) partial litter consumption; (2) partial vs full duff
consumption; (3) mortality but not consumption of understory trees with
live overstory; (4) mortality but not consumption of overstory trees;
(5) mixedwood severity example showing consumption of broadleaf foliage;
(6) woody debris consumption; (7) snag preferential consumption relative
to little to no bole consumption in live trees

Give lat/long, year, ecozone, severity class, and leading spp for each
photo, maybe other relevant metrics? From some of the experimental fires
mostly??

\section{Appendix E: List of Resprouting Hardwoods of Canada}

Alnus spp. Arbutus men. Betula all. Betula pap. Betula pop. Fraxinus
ame. Fraxinus nig. Fraxinus pen. Populus bal. Populus gra. Populus tre.
Populus tri. Quercus spp. Salix spp.



\codedataavailability{This article was produced from an RMarkdown
document with underlying data, available at
https://github.com/nrcan-cfs-fire/FireDMs} %% use this section when having data sets and software code available



%%%%%%%%%%%%%%%%%%%%%%%%%%%%%%%%%%%%%%%%%%
%% optional

%%%%%%%%%%%%%%%%%%%%%%%%%%%%%%%%%%%%%%%%%%

%%%%%%%%%%%%%%%%%%%%%%%%%%%%%%%%%%%%%%%%%%

%%%%%%%%%%%%%%%%%%%%%%%%%%%%%%%%%%%%%%%%%%
\competinginterests{The authors declare no competing
interests.} %% this section is mandatory even if you declare that no competing interests are present

%%%%%%%%%%%%%%%%%%%%%%%%%%%%%%%%%%%%%%%%%%
\disclaimer{The algorithm and results presented only apply to boreal and
temperate forest ecosystems where sufficient ground plots of fire
severity are available. As a data-driven model, this framework is not
suitable for other ecosystems nor agricultural or forestry biomass
burning practices.} %% optional section

%%%%%%%%%%%%%%%%%%%%%%%%%%%%%%%%%%%%%%%%%%

%% REFERENCES
%% DN: pre-configured to BibTeX for rticles

%% The reference list is compiled as follows:
%%
%% \begin{thebibliography}{}
%%
%% \bibitem[AUTHOR(YEAR)]{LABEL1}
%% REFERENCE 1
%%
%% \bibitem[AUTHOR(YEAR)]{LABEL2}
%% REFERENCE 2
%%
%% \end{thebibliography}

%% Since the Copernicus LaTeX package includes the BibTeX style file copernicus.bst,
%% authors experienced with BibTeX only have to include the following two lines:
%%
\bibliographystyle{copernicus}
\bibliography{references.bib}
%%
%% URLs and DOIs can be entered in your BibTeX file as:
%%
%% URL = {http://www.xyz.org/~jones/idx_g.htm}
%% DOI = {10.5194/xyz}


%% LITERATURE CITATIONS
%%
%% command                        & example result
%% \citet{jones90}|               & Jones et al. (1990)
%% \citep{jones90}|               & (Jones et al., 1990)
%% \citep{jones90,jones93}|       & (Jones et al., 1990, 1993)
%% \citep[p.~32]{jones90}|        & (Jones et al., 1990, p.~32)
%% \citep[e.g.,][]{jones90}|      & (e.g., Jones et al., 1990)
%% \citep[e.g.,][p.~32]{jones90}| & (e.g., Jones et al., 1990, p.~32)
%% \citeauthor{jones90}|          & Jones et al.
%% \citeyear{jones90}|            & 1990


\end{document}
